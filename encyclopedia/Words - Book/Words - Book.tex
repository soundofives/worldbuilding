% universal settings
\documentclass[smalldemyvopaper,11pt,twoside,onecolumn,openright,extrafontsizes]{memoir}
\usepackage[utf8x]{inputenc}
\usepackage[T1]{fontenc}
\usepackage[osf]{Alegreya,AlegreyaSans}
\usepackage{textcomp}

% PACKAGE DEFINITION
% typographical packages
\usepackage{microtype} % for micro-typographical adjustments
\usepackage{setspace} % for line spacing
\usepackage{lettrine} % for drop caps and awesome chapter beginnings
\usepackage{titlesec} % for manipulation of chapter titles

% for placeholder text
\usepackage{lipsum} % to generate Lorem Ipsum

% other
\usepackage{calc}
\usepackage{hologo}
\usepackage[hidelinks]{hyperref}
%\usepackage{showframe}

% PHYSICAL DOCUMENT SETUP
% media settings
\setstocksize{8.5in}{5.675in}
\settrimmedsize{8.5in}{5.5in}{*}
\setbinding{0.175in}
\setlrmarginsandblock{0.211in}{0.822in}{*}
\setulmarginsandblock{0.722in}{1.545in}{*}

% defining the title and the author
%\title{\LaTeX{} ePub Template}
%\title{\textsc{How I Started to Love {\fontfamily{cmr}\selectfont\LaTeX{}}}}
\title{A Collection of Stories from the Red Moon}
\author{The Author}
%\newcommand{\ISBN}{0-000-00000-2}
\newcommand{\press}{My very own studio}

% custom second title page
\makeatletter
\newcommand*\halftitlepage{\begingroup % Misericords, T&H p 153
  \setlength\drop{0.1\textheight}
  \begin{center}
  \vspace*{\drop}
  \rule{\textwidth}{0in}\par
  {\Large\textsc\thetitle\par}
  \rule{\textwidth}{0in}\par
  \vfill
  \end{center}
\endgroup}
\makeatother

% custom title page
\thispagestyle{empty}
\makeatletter
\newlength\drop
\newcommand*\titleM{\begingroup % Misericords, T&H p 153
  \setlength\drop{0.15\textheight}
  \begin{center}
  \vspace*{\drop}
  \rule{\textwidth}{0in}\par
  {\HUGE\textsc\thetitle\par}
  \rule{\textwidth}{0in}\par
  {\Large\textit\theauthor\par}
  \vfill
  {\Large\scshape\press}
  \end{center}
\endgroup}
\makeatother

% chapter title manipulation
% padding with zero
\renewcommand*\thechapter{\ifnum\value{chapter}<10 0\fi\arabic{chapter}}
% chapter title display
\titleformat
{\chapter}
[display]
{\normalfont\scshape\huge}
{\HUGE\thechapter\centering}
{0pt}
{\vspace{18pt}\centering}[\vspace{42pt}]

% typographical settings for the body text
\setlength{\parskip}{0em}
\linespread{1.09}

% HEADER AND FOOTER MANIPULATION
  % for normal pages
  \nouppercaseheads
  \headsep = 0.16in
  \makepagestyle{mystyle} 
  \setlength{\headwidth}{\dimexpr\textwidth+\marginparsep+\marginparwidth\relax}
  \makerunningwidth{mystyle}{\headwidth}
  \makeevenhead{mystyle}{}{\textsf{\scriptsize\scshape\thetitle}}{}
  \makeoddhead{mystyle}{}{\textsf{\scriptsize\scshape\leftmark}}{}
  \makeevenfoot{mystyle}{}{\textsf{\scriptsize\thepage}}{}
  \makeoddfoot{mystyle}{}{\textsf{\scriptsize\thepage}}{}
  \makeatletter
  \makepsmarks{mystyle}{%
  \createmark{chapter}{left}{nonumber}{\@chapapp\ }{.\ }}
  \makeatother
  % for pages where chapters begin
  \makepagestyle{plain}
  \makerunningwidth{plain}{\headwidth}
  \makeevenfoot{plain}{}{}{}
  \makeoddfoot{plain}{}{}{}
  \pagestyle{mystyle}
% END HEADER AND FOOTER MANIPULATION

% table of contents customisation
\renewcommand\contentsname{\normalfont\scshape Contents}
\renewcommand\cftchapterfont{\normalfont}
\renewcommand{\cftchapterpagefont}{\normalfont}
\renewcommand{\printtoctitle}{\centering\Huge}

% layout check and fix
\checkandfixthelayout
\fixpdflayout

% BEGIN THE DOCUMENT
\begin{document}
\pagestyle{empty}
% the half title page
\halftitlepage
\cleardoublepage
% the title page
\titleM
\clearpage
% copyright page
\noindent{\small{This novel is entirely a work of fiction. The names, characters and incidents portrayed in it are the product of the author's imagination. Any resemblance to actual persons, living or dead, or events or localities is entirely coincidental.

%\par\vfill\noindent Paperback Edition\space\today\\ISBN\space\ISBN\\\copyright\space\theauthor. All rights reserved.\par\vfill\noindent\theauthor\space asserts the moral right to be identified as the author of this work. All rights reserved in all media. No part of this publication may be reproduced, stored in a retrieval system, or transmitted, in any form, or by any means, electronic, mechanical, photocopying, recording or otherwise, without the prior written permission of the author and/or the publisher.\par}}
\clearpage

% dedication
\begin{center}
\itshape{\noindent{Dedicated to my all {\fontfamily{cmr}\selectfont\LaTeX{}} users.}}
\end{center}

% begin front matter
\frontmatter
\pagestyle{mystyle}
% preface
\chapter*{Preface}
The Seldar created the multiverse, and in its center, they placed the Prime Plane. The Prime Plane held the Seldar’s canvas—the world that they left nameless. For millennia, the Seldar leave the canvas blank, painting instead the outer realms. Their gaze is shifted back to this nameless world when the Children of Ehlu appear on an island known as the Eye of Ehlu. There they live in bliss, and from among them one becomes distinguished. He is the Architect, and he is the first of the children to awaken on the Eye of Ehlu. With his mind, he holds knowledge, with his hands he holds power, and with his heart he holds emotions uncounted. In grace, he teaches to his kin his heart, gives to them his power, and shows to them his mind. The Children of Ehlu delight, and all is good. They make the world a garden of their delight, and give themselves into it, so that they become its life and it theirs. The Architect names himself their instrument, and for them he splits himself into shards. Through the life of Ehlu's children, the world wakes all other creatures. The shards of the Architect watch over them in benevolence, carrying out the destiny set before them.
% acknowledgements
\chapter*{Acknowledgements}
% table of contents
\clearpage
\tableofcontents*

% begin main matter
\mainmatter
%%%%%%%%%%%%%%%%%%%%%%%%%%%%%%%%%%%%%%%%%%%%%%%%%%%%%%%%%%%
%CHAPTER
%%%%%%%%%%%%%%%%%%%%%%%%%%%%%%%%%%%%%%%%%%%%%%%%%%%%%%%%%%%
\chapter{Chapter One}

\textit{Van'Yasseron}

\textit{When We Walked Beside Beings of Power}


\par\vfill The first age of the world after its creation by the Seldar. Marked by the awakening of Power and Growth, change came quickly across the waters, the continents, and the sky. The age was dark, lit in glorious colors of blue and red by the twin moons in the sky. The sun had not yet appeared, and would not be seen for many thousands of years after the shattering of Power. Now the world was young, and under the light of the moons bloomed still flowers and plants, and animals still hunted and nested in the nooks of hills and the caves of mountains. This was the way of the world before the coming of the Nalanat, the first beings.

\par\vfill\noindent 4,000,000,000 Years Before Burned Time—The first shard of the Architect awakens under the world. They are Power, and they define the movement of the world—the motives of creatures, and the emotions of the planet. Another shard of the Architect, Growth, awakens on the world. Growth defines this period of history.

\par\vfill\noindent 700,000 BBT-- A race of fey shapeshifters, the Nalanat, awaken on an island in the center of Perakor—Naa'yamen. They walk long under the twin moons of the world, for the lives of the Nalanat are long, but not everlasting. Many Nalanat walk the earth for twenty-five-thousand years before their bodies or minds fail them, but those who are powerful can live far longer.

\par\vfill\noindent 600,000 BBT—During the flourishing time of the early world, some of these shapeshifters follow the natural migrations of fishes and whales across the globe. Pockets of the fey shapeshifters reach multiple continents.

$\succ$ The fey change quickly to their environments, as if their evolution was accelerated. Many humanoids develop from the isolated niches.

$\succ$ The fey that remain on Naa’yamen take on a humanoid shape and are known as Naa'waith, also called the Elea'waith.

$\succ$ The Loss’kelvar form nomadic tribes in Numendor.

$\succ$ The Onear evolve in the seas around Numendor.

$\succ$ The Okemenel first begin to develop in Annuntol. These beings follow the migration of sharks, and from them they learn predatory habits. They take the forms of savage creatures. The grandeur and large caves of Annuntol lead the Okemenel to adapt, growing larger and larger.

\par\vfill\noindent 300,000-200,000 BBT—The shapeshifting races develop spoken language around this time. Many of the languages develop separately between the races, giving each a unique feel and distinct sound. The Naa’waith tongue sounds smooth and musical, reminiscent of modern elvish, while the Okemenel tongue is low and slow. The Okemenel are giant, but graceful. These characteristics manifest in their speech.

$\succ$ The Loss’kelvar and the Onear develop a similar tongue as they interact around Numendor. Their interactions are not typically in good spirit, however, as the Loss’kelvar are a cold race driven by the hunt. They live a pack lifestyle taken from the northern wolves of Numendor. The Loss’kelvar themselves tend to take on wolf-like characteristics or even become wolves themselves. Their society is nomadic and predatory, but pack leaders are held in high regard. Packs meet occasionally, but they tend to keep to their territory. This harsh lifestyle can be seen in their speech, which is broken and sharp.

158,000 BBT—The Okemenel’s nomadic tribes fight savagely with each other, and many of the species are killed in the battles. The many creatures of the Okemenel throw themselves at each other and tear viciously. One among them, Comenraan, takes a new form—a massive winged beast, scaled and horned. The first dragon. Comenraan dominates the other tribes, and he leads his own towards the most bountiful of locations on Annuntol, the Grove of Comenraan. The followers of Comenraan destroy any who approach, leaving the other groups of Okemenel to die. Those who do not follow Comenraan are the Rhovari, hunted by the greater Okemenel and forced to hide their tribes in the dark caves of Annuntol, or under the roots of mighty trees. Comenraan isolates his followers and promotes the destruction of the Rhovari. The generations that follow begin to worship Comenraan.

120,000 BBT—The Loss’kelvar develop a unique culture. Of the original fey beings, the Loss’kelvar are the ones who keep their shapeshifting, using it mainly as camouflage. Packs of red wolf beasts stalk the red nights under the twin moons. The Loss’kelvar believe that the two moons, Agash’Kelesh and Rikshe’Kelesh, are the two eyes of their collective ancestry. This builds an importance of family and genealogy in the Loss’kelvar society. The Loss’kelvar believe that when one dies, they go to join their ancestors in Keleshe.

100,000 BBT—The few Okemenel that are left after the savagery on Annuntol live in the Grove of Comenraan. Generations of adherent and worship have made Comenraan continue to live and gain power, feeding his life force with the power of others. Comenraan nurtures the Okemenel with his power, and he carves into the Grove of Comenraan, creating grand tunnels for his following to live in. As the Okemenel peoples grow, and their society changes, Comenraan uses his power to expand this cave system into what is akin to the world’s first city.
-While the Okemenel look down, the Naa’waith look upward. They give the names Memaska and Meval to the two moons of the world.

98,000 BBT—The Onear learn to live in and around underwater volcanoes, trapping fish that live near them. They carve honeycomb-like structures into the walls of the volcanoes. Their culture develops as well: the Onear begin to practice a ritual for the dead, sending them into the underwater volcanoes to burn and become one with the waters that they live in. The Onear begin to spread themselves throughout the oceans as groups split off. One among them, Habbakuk, leads a group to the ocean, Eanun, east of Perakor, where the warm waters around Perakor and Annuntol hold a bounty of fish. These Onear become known as Annonear. Habbakuk’s following flourishes in the waters, and they quickly meet the Okemenel of Annuntol. Habbakuk sees the grandeur of Comenraan, his mighty wings and scales of gold, and is in awe. He meets with Comenraan on the shores of Annuntol and learns with him in good faith. Habbakuk teaches Comenraan the languages of the western races and tells of the Loss’kelvar of Numendor. Comenraan is intrigued by the tales of distant lands, and the seed of expansion is planted in his head. The two beings share with each other the tales of their people, and Comenraan shows Habbakuk his power. The worshipping of Comenraan gives the dragon extraordinary power, and he gives Habbakuk the knowledge to let the Onear thrive in the waters off Perakor. Habbakuk takes this knowledge back to his people and leads them to a bountiful cove off the island to the south of Annuntol, Qas. There, Habbakuk’s people thrive. They revere him as the giver of life and the king of the seas, and the generations of Onear begin to worship him.

90,000 BBT—The Naa’waith begin to grow wheats and other crops on Naa’yamen. With this advancement, they begin to build primitive villages. Before long, the agriculture and condensing of the Naa’waith in groups causes strife. Skirmishes, raids, and battles tear the Naa’waith apart. The island was divided among the groups, each making their own small cities to protect their people. There were many groups of the Naa’waith, but few were prominent in the millennium to come.
Some groups engineer the small fishing vessels of the Naa’waith to bear the waves of the sea. Those who leave are known as the Lemba'waith, and they continue to hunt and gather or found small concentrations of villages across the globe. This separation of the Naa’waith peoples is known by them as the Keles. The Lemba’waith who travel east and land on Annuntol are known as Annun’waith. Rhunendor are known as the Rhun’waith. Those who travel west and land in Numendor are known as the Numen’waith.
-Some of those who left were followers of Nestadis, who loved the trees and the grasses of Naa’yamen but was not powerful enough to remain. Nestadis and her kin ever loved to discover new lands, and Naa’yamen was filled with others who only wanted their own lands. Nestadis took her followers on small ships that were unfit for the harsh seas, and many perished before landing upon the inner shores of the west-lands of Perakor. There they remained mainly in the lower latitudes, avoiding the north lands. The spires of that land were very much reminiscent of the steep mountains of Annuntol, but their valleys are filled with a thick jungle of tall trees. The Lemba’waith there find that the trees offer shelter and prey, but are lacking in crops. They find a technique to grow crops on the sides of the great spires, mostly lichens and edible fungi. In addition, this area of Perakor has a soft soil that allows the Lemba’waith to dig tunnels to farm subterranean crops. These Lemba’waith, closest to Naa’yamen still, were named the Neve’waith.
-Another of the houses to take ships from Naa’yamen in the Keles was the house of Saeronder the Gentle. Saeronder had a small following, but he and his many children felt that the seas could offer repute against quarrels of Naa’yamen. He led his people not far to the east, to the lands that he named Nevlonde, the Near Havens. They found the landscape to be calm and dynamic. Rolling plains and shady mountains lend themselves nicely to an agricultural lifestyle. They were contented with their lives in the Nevlonde and named themselves the Seim’waith. The Seim’waith were always detached from the dealings of their brethren, and oft kept to themselves even when called upon.
-One of the mightiest clans to leave during the Keles was that of Castion, the Son of the Sea, and his sister Rosiell. Castion was the greatest of the shipwrights, and he and his followers were always calmed and given life by the music of the waters. Castion taught his shipwrights well the arts he knew, but only to his son Arndulin did he teach everything. While Castion was tall and strong, Arndulin was taller and stronger and fairer in body and mind. He was the source of Castion’s pride, and all that Castion did was for his son. Arndulin rejoiced in the waves like his father, but also found solace in the trees and rocks and dirt. The Numen’waith ships of Castion and Arndulin land on the shores of the small peninsula off the south-east part of Numendor and Rosiell lands on the small island to the south-east. Castion name this cool region Dor Dal, and they remain on the coasts there. Those that land on the south-eastern island name it Dae Ithil. Soon, some of those Numen’waith upon Dae Ithil discover the second large island south of Dor Dal, which they name Tol Atya. While Castion and his son are detached from Rosiell, the societies still work together and interact. For many years, the followers of Castion lived together there, but Arndulin wished to explore the land of Dor Dal, the forests and mountains. He spoke to Castion of his dreams, and though Castion was saddened at his son’s wish to leave the waters, he did not forbid him to go. So Arndulin moved inland, and many of his people rested throughout the countryside, but he found his home on the shores of Luth Lirill, the Bay of Resting, where the waters reflected so brightly the light of the moons and the shapes of the trees. There Arndulin worked to craft a wondrous ship of white wood, but he was often distracted by the forests and animals and left the ship uncompleted in the bay. Those of the Numen’waith that followed Arndulin were named the Forhos, and those who made their homes upon the islands of Dae Ithil and Tol Atya were named the Aluhos. Castion and Rosiell soon encounter the Onear of Numendor. The Onear are a predatory aquatic people, their schools organized themselves under one queen, Faear. Faear was old and wise, and she saw the Aluhos and recognized that the Onear could learn from them. She insisted that the groups work in harmony. Each of the peoples learned language of the other, but the Aluhos found it difficult to learn the watery speech of the Onear. The Onear explain the best way to hunt the waters while the Numen’waith teach the Onear agriculture and construction. The Onear develop underwater farms growing seaweed and other aquatic plants. With the coming of the Numen’waith, the Onear society began to develop. They were organized in small places, some permanent while others were nomadic. These groups of Onear are led by a clan leader, and each of these clan leaders meets periodically to discuss territory and trade with the Numen’waith. These clan leaders are led by a single ruler chosen from among them, the one of them to accomplish the greatest feat by the time of the choosing. 
-Behind Castion and Rosiell in prominence were the houses of Tolas, Selor, Revion, and Farriel. The four families worked in unison and left Naa’yamen together in the Keles. Tolas and Selor were close friends to Castion, and he built for them many strong ships to bear the waters. While Castion and his folk sailed for the west, Tolas and Selor made for the lands of the south. Long they went south until they came upon the great falls at the rim of the world. They made fast upon the islands there and saw little opportunity, for the waters were stormy and rapid. Revion and his folk enjoyed the great mists cast up by the falls, and they stayed upon the southern island Tol Hith. Revion was a lover of adventure, and he wanted to look out over the falls of the world where the others shied away in fear. Revion leaned out and peered over the rim of the world and glimpsed through the mists the vast universe that the world held within its great bowl. With this vision Revion was amazed and enlightened, and made a great pier stretching from Tol Hith to the rim of the world. Often, he would look out over the rim and watch the universe move and grow, and some say he was consumed by his love for the unknown in that place. His people built great temples upon Tol Hith and the islands around it and were called the His’waith. The three remaining houses of Lemba’waith toiled hard to move back north, and they found strong currents bringing them to Annuntol. Yet even there the journey narrowed the great host of Lemba’waith, for Farriel and her folk saw the mighty towers, wondrous trees, and plunging lakes of the land and were enchanted. People of the deep rivers they became, the dark waters and high mountains of Annuntol hiding many secrets. Farriel led her people along the great river Taumuin along the stretch of Annuntol until they came to the hidden sanctuary of Varorn.
The two remaining houses of the great host that left Naa’yamen, those of Tolas and Selor, sail on from Annuntol to the far lands of the east. Their journey took them at first into the cold currents to the north, where they landed on the frozen continents north of Annuntol which they named Halcinaline. They find the land of Rhunendor, and waylay their ships upon its shores. The land is serene, and nature has taken the continent into its own hands. Mountains rise from flowing plains, and the waters off the coasts blend into the green forests. The Rhun’waith wander Rhunendor. The brilliance of the land eggs them to keep exploring, leaving few to settle down and farm. 
Tolas and Selor merge their houses on the mountaintops of Rhunendor to mark their union, and they name themselves Rhun’waith. Years they explore the land, and after twenty years they stumble upon a sanctuary at the center of Rhunendor, a cave with a large light opening it to the light of the moons. The cave had a calm pond, and birds roosted in the branches of scraggly trees growing in the cavern. Tolas and Selor felt a presence among them. Moonlight glistened on the pool and the birds flew from their branches. In that moment, a being appeared on the surface of the water—a shard of the Architect. The shard stepped forward towards Tolas and Selor, his aura of power overwhelming. The Rhun’waith fell to their knees in awe, and Tolas speaks “Ela’Hera’roilya!” (I see He Who Possesses All Power!). Hera’roilya dampens his aura and shows his benevolence to the Rhun’waith.
Hera’roilya is in conflict, as his self is a contradiction. He is Good, Evil, and Neutrality; Law and Chaos. He begins to die as he moves forward towards the Rhun’waith. In his first moments, he touches Tolas and Selor, bestowing upon them an idea of his power. To them he teaches the first magics of the world—the magic of raw power, the manipulation of reality and of non-reality. Hera’roilya fades, and tells Tolas and Selor of his internal plight. He will die, and until he does, his presence on the world is dangerous—even now, with his short time there, the world has felt his presence. Earthquakes rock the landscape, monsoons brew in the oceans, and volcanoes begin to shake.
Hera’roilya asks them to help him—he knows that his death may tear the universe apart, so he asks Tolas and Selor to entomb him in a stasis beneath the world. Tolas and Selor work with Hera’roilya to send him beneath the world, the concave bowl of the universe. He leads them to the place where he first woke, the Eye of Ehlu. The barren place with only a single oasis of time. They labor to create a tomb for Hera’roilya—a place made by the hands of those who cry for the being. As they finish, they lay Hera’roilya in the tomb and summon to them Meval, the great celestial power, twin to Memaska. Meval is the only thing that could keep the unbridled power of Hera’roilya from escaping the Eye of Ehlu. This is the Valcone. As Meval is summoned below the world, the red night dies. The Loss’kelvar cry in the night as Agash’Kelesh vanishes from the sky and their access to Keleshe is cut off. The soft blue light of Memaska shines down, mocking the red wolves of the north. While the imprisonment of Hera’roilya by Tolas and Selor saves the world the misery of his death, it kindles a powerful fire within the hearts of the Loss’kelvar. The Loss’kelvar know this moment as Sa’Kurtha. Because they believe that the moon was Keleshe, the place of their afterlife, its disappearance cuts off their connection to the afterlife. Their belief created a true Keleshe, and, now that it is gone, the souls of their dead remain on the earth. The Loss’kelvar lose the ability to truly die, though they can feel the pain of death forever after they are killed. The Loss’kelvar are quickly corrupted, unable to die but able to feel the pain of death. The fey of the north fall apart, their society crushed by the loss of Agash’Kelesh.
Tolas and Selor return to the world and find Revion of the His’waith. They tell him of Hera’roilya and his prison, and talk Revion to watch over the Eye of Ehlu from Tol Hith. They spend a time upon Tol Hith to teach Revion of the magics they have glimpsed, and the three magi work to sheath the rim of the world in the mighty storm Yvari. This is a shield against outsiders seeking Hera’roilya. The His’waith peoples upon Tol Hith sit at the eye of Yvari, and they become the guardians of Hera’roilya. Tolas and Selor are the legacy of power now, and they return to Rhunendor to try to spread this magic among the Rhun’waith. When they return, and demonstrate their powers, the Rhun’waith see Tolas and Selor as higher beings. They believe in the power of the two magi, granting long life and greater powers, but Tolas and Selor were not greedy with the power they had been granted. They take their power and teach it to others. They create a great city to house their students, Celebtal, and it floats over the first sanctuary of Hera’roilya. At the center of the city stands the temple and learning grounds of Tolas and Selor. The other Rhun’waith are not attuned to magic, and so they can only learn a weaker form of the power. Tolas and Selor continue their heritage with four children, and once their children take over the ruling of Celebtal, they return to the sanctuary of Hera’roilya beneath the world.
-The Annun’waith first land far from the Grove of Comenraan at the mouth of the Taumuin, but the Okemenel are aware of the presence. Comenraan and other Okemenel approach the new peoples as they journey up the Taumuin and greet them. The grandeur of the Okemenel at first frightens Farriel and the Annun’waith, but the ancient dragon blooms a great field of flowers around the newcomers, and they are amazed. The Okemenel and Annun’waith cannot yet communicate, but Comenraan guides the Annun’waith up the Taumuin to the Varorn, a hidden sanctuary to the east of the Grove. There the mighty dragon watches the Annun’waith and he slowly works with Farriel to learn the language of the Annun’waith. In time, the Okemenel learn the language of the Naa’waith, for the Annun’waith bodies are incapable of creating the sounds of the Okemenel tongue. Comenraan shows Farriel his power, and the Annun’waith are amazed. The two peoples form a friendship between them, and many of the Annun’waith begin to worship Comenraan as a great being of Nuinen. While the Okemenel are hunters and gatherers, the Annun’waith still rely heavily on agriculture. The crops of Naa’yamen would not grow in the wet mountain valleys of Annuntol, so Farriel learns to domesticate a new plant, rice, and grow it in the valleys. As they live near the waters and fish, they contact Habbakuk and his Annonear. Habbakuk, enlarged by the grandeur of Annuntol and the worship of his followers, seems to the Annun’waith a great being like Comenraan. They revere him, and for this, Habbakuk gives them a bounty of fish. The Annonear live alongside the Annun’waith and learn their language.

85,000 BBT—After five thousand years, the magic of Tolas and Selor has spread throughout the Rhun’waith and reaches the other Naa’waith. The children of Tolas and Selor inherit their power and create cities to train the people in magic. Tolas and Selor have two sons and two daughters—Finiel, Ralun, Renno, Eres, respectively. Each of their children they send out to the greatest of the Naa’waith peoples. The great city Halletal is raised by Renno in Annuntol, Ringwetal is raised by Finiel in the snowy north of Numendor. Templatal is created on Naa’yamen by Eres. Ralun takes over Celebtal after Tolas and Selor. The four mages of the four cities come together on the precipice of Tol Hith, and with Revion the Guardian they pull from the bowl of the world four stars and create the Anariima. The Anariima is the constellation to connect them, the society to bring together the Naa’waith peoples. Not only is the Anariima the group of mages that rule the cities, it is also the gate-like connection between the cities. The children of Tolas and Selor work together to connect the reality of the cities, creating ports to travel from one city to another. These gates are manifested around small star-like stones.
	As this magic spread across the continents, the other fey races quickly learn of it.
-Renno comes to Farriel and the Annun’waith in Varorn and raises for them the great city of Halletal with her magic. The towers of Halletal rise to challenge those of Annuntol itself, with walls of shining gold and streets of crystal and water. Renno establishes a grand temple for her teachings, and worked to spread the magic of Hera’roilya among the Annun’waith. The city is hidden within the realm of Varorn, but Comenraan was quick in discovering it. He looked down upon the city in rage, seeing that the Annun’waith had power to rival his own. Renno replaces Comenraan in their minds, and they begin to believe Renno to be of equal power as Comenraan. As the people’s belief shifts away from Comenraan, he feels a shift in his power—he weakens. Comenraan is filled with fear, and he begins to preach the danger of the Annun’waith to the Okemenel in his Grove.
-The Annonear under Habbakuk see the coming of power to the Annun’waith as a bad omen. Habbakuk, now growing powerful, sees the coming of Renno as a rivalry just as Comenraan had seen. He sees also the rage of Comenraan, and knows that the Annun’waith have made mistake far graver than they knew. He then goes to Renno and Farriel in Halletal and warns them of the jealousy of Comenraan, but reveals not of his standing. Habbakuk is torn, not wanting to betray Comenraan fully, but knowing that if Comenraan destroys the Annun’waith then Habbakuk would lose much power from their belief. The great Annonear retreats into the depths of the ocean to ponder this.
-Finiel came first to the wide cold lands of Khelonde, the realm bordering the Helkalad north of the Telkohar. He wandered from his ship until he came to the coasts of the Helkalad, and he looked down into the cold waters and fell deeply in love with the cold beauty that was reflected from them. The Khelonde he loved, the rolling hills and thick forests of tall pines. The ground was cold, but the grass that grew was soft. Among the grass, he saw many flowers also, blooming under the soft light of the moon in the sky. The flowers were of shining silver, and their dew glittered and shined as it froze upon their leaves. He knew that his city should be raised there, on the hill he named Amon Celefin. Finiel left for the south then, whence he came first to the coasts of Dor Dal. He spoke with Castion and Rosiell of the Aluhos and demonstrated his great power and his willingness to teach it, but Castion thought still of his son Arndulin, who had left north with the Forhos, and Rosiell loved only the rain and the waves, and had little need for the powers that Finiel showed. Castion told Finiel of his son to the north, and of his love for new things and places. To him Castion sent Finiel, for the old Aluhos knew that Finiel and Arndulin would become close friends. And so Finiel trekked north, and spoke to the Forhos throughout Dor Dal on his path to Arndulin by Luth Lirill, and many he convinced to follow him. At the head of a host of the Forhos Finiel strode upon the coasts of Luth Lirill and found Arndulin on the water. He spoke to him and showed him the power that he had learned, and Arndulin was amazed. Arndulin spoke from his tower to all the Forhos around Luth Lirill and many of them he persuaded to follow Finiel to Amon Celefin. Finiel led the many followers of Arndulin north by the blazing light of his Anariima, Corilya. His staff banishes shadows and welcomes warmth, and when the host reaches Amon Celefin the light of Corilya scatters off the silver flowers, and the Forhos are enraptured. Around Amon Celefin, Finiel raises Ringwetal, his glorious city. Arndulin is the Lord of the Forhos and the prince of the Numen’waith, and as he sees the wonder of Ringwetal come from the ground and Corilya’s light shine over the flowers he names those of the Naa’waith who do not harken to the Anariima as Dym’waith, and those that take the gift of Hera’roilya as the Areil. Finiel and Arndulin rule Ringwetal in harmony, Finiel as the great teacher, and Arndulin as the lord.
After many years had passed and the societies of the Forhos had taken to the schools of Finiel, one student goes out into the cold to experiment with his powers, and he finds something strange: a ragged, torn beast. The creature seems long dead, but life still burns in its eyes. It is a Loss’kelvar. The Loss’kelvar have been broken—many see no point in eating, but still cannot die. This one picks its head up as the student approaches. The student demonstrates his powers, the magic mending the Loss’kelvar wounds. They begin to speak to each other, the student’s magic allowing him to understand the creature. The beast introduces itself as Bash'Tikish, or Bash. The student regaled Bash’Tikish of the origin of the magic—a gift from a powerful being, now held in stasis under the world by the moon Meval. That story changed something within Bash’Tikish—he alone of the Loss’kelvar knew the cause of his people’s plight. These flimsy beings had taken Agash’Kelesh! Bash’Tikish immediately slays the student and begins to seek out more of his people.
-Eres came to Naa’yamen as an emissary of the east, the first of the Lemba’waith to return. The houses and clans of the Naa’waith were still divided and broken, but when Eres came and showed them of her powers, they all harkened to her call. She raised herself upon a high tower and called down to the heads of the many clans that had gathered, telling them of Hera’roilya, of Tolas and Selor, and of her power. She raised then the core of the most magnificent city of Naa’waith, Templatal. The war-torn countryside and the high walls of the clans gave way to towers and streets, to towers that rose higher than the clouds, and to a great arc upon which Eres made her temple. Great parks and terraces Eres made, and she stood upon her balcony then and looked out upon the Naa’waith. This effort brought all the clans together as houses of the Naa’waith in Templatal, and over time the people themselves expanded the city with their own magics. Eres rules as Lady of Templatal, and she is loved by all. She goes out to the Seim’waith under Saeronder, but he tells her that his people are contented under the soft light of Memaska. They are a calm and gentle people, and have no need for the powers that Eres offers. She goes then to the Neve’waith to the west, and some of them return with her to Templatal.

80,000 BBT—Another generation sees the four cities of the Naa’waith thrive. The fey learn the arts of magic and spread their society. Their agriculture and buildings are aided by the magic, and their hierarchy of respect is based on the power each person held with magic. As the Naa’waith gain power, however, the other fey races develop as well. The caves of Comenraan are rank with hatred for the Annun’waith of Halletal. Comenraan is weak and furious—the Okemenel are few and far between though they are powerful. Their belief is not enough to sustain Comenraan’s lust. In a horrid twist of his powers, he attempts to create life, more worshipers for his realm. He gathers the bones of the deceased in a blasphemous cavern and works his powers upon them. The bones are given flesh, and the flesh is given life. The caverns under Annuntol were filled with the screams of Comenraan’s child race, a species of monstrous beings that Comenraan doled over, preaching to them of his powers and prowess. The creatures were incapable of belief in Comenraan, and he was angered at his failure. Shamed, the mighty dragon destroyed many of his creations. Those that survived escaped into the deepest tunnels and reaches of the caves. Years passed with Comenraan rarely coming above ground as he sat in the caves, contemplating his failure—and how to succeed if he were to try again. The golden scales of the mighty dragon had dimmed from years below ground before he made a second attempt, this time to create another Okemenel. Through a dark ritual, Comenraan sacrifices part of himself to create Raeg, the mightiest creation Comenraan. Comenraan hides Raeg in his caves and attempts to bestow upon him his power. The caves are filled with the howls of Raeg, but their cease signals Comenraan’s success. Raeg the Beast, the Mindless, the Raging, the Undying. The monster that cannot be killed, for each death that strikes at Raeg’s false soul only manifests itself in a vile growth, making the dragon stronger.
-Habbakuk learns of the creation of Raeg and begins to fear Comenraan’s powers. Hatred 
builds inside him—Comenraan grows in power while Habbakuk knows not how to wield his. Comenraan replaces Habbakuk as his ally with Raeg—this abomination from the depths of the caves. Habbakuk turns his back on Comenraan, vowing that the Onear will not be underestimated. Habbakuk meets with Renno of Halletal and asks to learn the ways of magic. Habbakuk is the first of the non-Naa’waith to learn this raw magic. In his powerful state, Habbakuk finds it easier to learn the magic—those with power gain power more easily. Habbakuk begins to love Renno, and they live in peace in Halletal. In time, Renno draws a cord of power from Onsinta and sews it into a cloak for Habbakuk. This cloak is Avcolla, the Vanishing Cloak, for any who wear it bear no outward presence. Renno granted this to Habbakuk so he may walk with the peoples of Halletal and see their ways unnoticed, and so that he may meet with her in secret.
-Bash’Tikish stands upon Amon Maica to the north looking over Ringwetal, the great city of the Naa’waith in Numendor. Behind him are the dead Loss’kelvar, hundreds of thousands of them. These are the Gurohtar, those of the Loss’kelvar stirred by Bash’Tikish’s calls. They are here to destroy the Forhos and free Agash’Kelesh. Bash’Tikish leads a blood-curdling howl into the night as the Loss’kelvar descend on the city. Domringel is the name given to this night by the Naa’waith. The Forhos are shocked into action, but too late. The Areil could not prepare a defense for the city, and the Gurohtar poured through the streets, each howling and calling for Agash’Kelesh. Finiel awakes to the great howling, but even then Bash’Tikish was at his door. Arndulin rides through the streets, calling for those that survived the chaos to move south. The Forhos of Ringwetal are slaughtered, and Finiel is captured. The city falls, but before the fall, Finiel uses his power to close the Anariima. In each of the other three cities of the Naa’waith, the connections to Ringwetal blink out. The northern city is gone. The other children of Tolas and Selor—Ralun, Renno, and Eres—meet in Templatal. They come together and use their powers to look upon Ringwetal and upon Finiel. They see the darkness that covers Ringwetal—the Forhos that survived retreated south over the Telkohar. Arndulin meets with the Onear Hylama. Hylama is a young Onear who became ruler in the areas north of Numendor. She met with Finiel before the fall of Ringwetal and began to learn magic. In her training, she learned to create brilliant lights, beautiful enough to blind her opposition. Hylama and Arndulin emerge from their meeting with a strong alliance between their peoples. The Forhos and Onear would join forces to stop Bash’Tikish. When the children look upon Finiel, they find their vision clouded. He is lost. Finiel was their leader, the eldest among them—with him lost, they do not know what to do. The children attempt to call Tolas and Selor to them for guidance, but they receive no answer.

79,975 BBT—Arndulin and Hylama combine their powers to create defenses for Dor Dal. Arndulin leads those magi who survived the Domringel to narrow the Telkohar, and they layer it with a powerful barrier, the Tinechorn. The architects of the Tinechorn create it to wrap around Numendor proper and keep the Loss’kelvar from crossing out of the cold continent. The shield requires the constant focus of Arndulin, so he makes a high tower to watch over the shield, the Perch of Arndulin. Within the tower and around the Telkohar, the Forhos and Onear made their new villages and the first city of combined race, Harenyast. While Arndulin holds the Tinechorn, the society of the Numen’waith and Onear come together as one. The magi of the Forhos teach the Onear the magics they have been taught, and Arndulin continues to mentor his students, teaching them the primal magic and nature of their species: shapeshifting. The Forhos begin to teach this to the Onear, but some refuse to learn. They are set in their way of life, and need not for this change. Those of the Onear who pursue mastery of the land and the waves are known as the Tralonear. The Tralonear work to embrace this magic so that they may join the Forhos on the land. Hylama is the first to master this, and she is called Vantura.

79,900 BBT—The leaders of the Forhos and Tralonear clans meet in council in Harenyast. Arndulin knows that with Finiel captured, Bash’Tikish will likely try to use him to open the Anariima to the other cities. Though the Gurohtar cannot be killed, the Anariima can be stolen, and Finiel can be freed. And so, the leaders begin to prepare for their plan: Arndulin learns to create blades and arms of magic, and he teaches this art to the Forhos and Tralonear. In addition, Arndulin picks the greatest of the Areil among the Forhos and teaches them to hold the Tinechorn strong. They will remain at Harenyast when the Forhos march north. Fifty years the leaders prepare longer.

79,850 BBT—The Forhos and Tralonear finish preparations and plans to invade the Khelonde and take back Ringwetal. The weapons they held were blades of writhing magic, efficient and fit perfectly to each who used them. Armed with a high helm and plates of magic upon his breast, Arndulin strode at the head of his mighty host. The dark cold of the north parts for Arndulin, and the force lines up outside the ruins of Ringwetal. The city is dead of lights, and the Gurohtar look from the windows and from the walls. Bash’Tikish grimaces, and barks to his soldiers. As the clouds part and the light of Memaska shines down upon Amon Celefin the horns of the Numen’waith blow and the force charges the city. The mighty yell and deafening horns of the Forhos frighten the Gurohtar, and Bash’Tikish calls for his men to protect the Anariima. The gift of Hera’roilya shines brightly still in the eyes of the Areil, and their wrath crashes like waves down upon the beasts of the north.  Arndulin yells as he cuts through the bodies of the Loss’kelvar, “Aucir’telcontya! Rac’nammantya!” (Sever their legs! Break their claws!) Vantura and her Tralonear rise from the Helkalad then and fall upon Ringwetal from the north. The Tralonear enter the city as the horns of the Forhos blast once more, and Arndulin strides to the great plaza of Corilya and takes the Anariima. He turns to see Finiel, shuddering in his steps, moving towards Arndulin. The prince of the Forhos calls to Finiel only to find that the torture of Bash’Tikish had scared Finiel and left him a husk of his former self—but still just as powerful. Finiel, in his corruption, saw only an enemy in Arndulin. His body pulsed with power, and he unleashed it upon the Forhos and Tralonear. The streets were torn and the towers were toppled by Finiel’s might, and as he looked up at Memaska, he pulled forth the blood of the slain and painted its light red. The howls of the Gurohtar filled the night, and the dead awoke once again. Arndulin charge forward and dueled with Finiel then, his friend and his teacher. Vantura found Bash’Tikish and put her blade against his mighty claws. His following had made him huge and strong and had sharpened his teeth and claws. The two battles resonated throughout the city, as great auras of power emanated from the combatants. Finiel held in one hand a great bolt of fire and in the other a long whip of ice, and he faced Arndulin upon Amon Celefin at the heart of the city. Arndulin’s shield was strong, and he shattered Finiel’s bolt and caught Finiel’s whip around his sword. He threw down his sword and shield, disarming Finiel. The two battled with raw power then upon the plain of flowers. Bash’Tikish raged at Vantura, his cold blood boiling under the cursed red light above. They battled up the temple of Ringwetal, climbing its many stairs and moving along its long halls until finally they reached its peak. His claws flew quickly, and Vantura could not evade them on her new legs. Her blade parried and cut, and she severed the left hand and gauged out the eye of Bash’Tikish before being struck from the tower. Tralonear there were, waiting below, and they summoned forth a great wave to catch her. Crashes echoed throughout Ringwetal as the power of Finiel was diverted and dodged by Arndulin. The Forhos was fast, swifter than rain, and as Finiel struck down with a mighty hammer of force Arndulin stepped inward, summoning forth a spear of cold and striking it through Finiel. The mighty son of Tolas and Selor fell then, though he was not killed. Arndulin carried him, and the Forhos and Tralonear retreated to Harenyast. The flight to Harenyast was long, and it was not without skirmishes. Though the blood washed from the skies after Dagor Mehtanawen as it was called, the Gurohtar did not tarry in their attacks. Arndulin held the Anariima aloft in the skirmishes, and its light shone brightly forth blinding the attackers, but the path was yet long and many Forhos and Tralonear were slain. The final push was made by the Gurohtar as Arndulin and Vantura’s hosts neared the Tinechorn, which the Loss’kelvar knew they could not cross. Desperate in their final raid, their defenses were laid low and a good many were maimed. The hosts of the south reached Harenyast and recovered.

79,849 BBT—Arndulin placed Finiel in a prison high in the ranks of Harenyast, and he chained him fast with rock and crystal, enchanted to hold the powerful Areil. Corilya Arndulin placed upon the head of Harenyast, at the Perch where its powers could be used to hold the Tinechorn. Arndulin met often with Finiel, speaking with him and trying to break the corruption of Bash’Tikish, but it was evident that Finiel was broken in mind and body. Saddened at the loss of Finiel, Arndulin leaves his presence and rarely returns. Bash’Tikish and the Gurohtar rally outside the wall of the Tinechorn, and while they cannot cross they throw mighty stones in attempts to break it, and they corrupt evil beasts of the north to cross and watch the south Harenyast. Thus, the siege of the fortress of the Forhos began that would last many hundreds of years. Arndulin named this event to be one of a greater act: Yestoht, the First War.

79,550 BBT—Three hundred years the Areil of the Forhos and the Tralonear toiled below Harenyast to create the vast caverns of Haldarda. Using powerful magic, they carved their escape from Harenyast—but more than that, as Haldarda turned into a massive complex of innumerable caverns, abysses, and underground seas. In a massive swipe of his power, Arndulin broke from Haldarda into a wondrous realm of natural caves, the Morinuin. The expansion of the Areil and Tralonear continued into the unknown depths. Over the course of the siege of Harenyast, many of the Forhos and Tralonear became enchanted with the Morinuin below the ground and remained there, lost to the group. Those who chose to live among the caves created outside of Haldarda during their creation were called the Vanhar. Deep in a hidden part of the Haldarda the Areil carved Thuricaras, their secret city should Harenyast fall. And so, Arndulin was named Nuragon, Lord of the Deep. Arndulin Nuragon developed the Haldarda and the Morinuin attached to it, though the Morinuin was wide and deep, and its secrets had yet been untouched. Many exits he carved throughout Dor Dal, and he came to the surface to speak with his father Castion, King of the Numen’waith. He told Castion all that had befallen in the north lands, and of Haldarda and the Morinuin below. Warnings he gave of Bash’Tikish and the Gurohtar, and instructed Castion that if the Tinechorn should fall and the bells of Harenyast should ring its fall, then the Aluhos should seek the gates of the Morinuin, for each is secret and guarded. Castion sent his swiftest ships then to each of the Naa’waith cities to warn them of the coming storm. Arndulin Nuragon rode north then, and spoke to many Forhos along the mountains of Dor Dal, telling them that they must make great pyres upon the peaks of the mountains that should be lit if Harenyast falls.
	In this time the Tralonear and their Onear kin showed the Forhos their mighty forges, founded in the deepest chasms of the ocean, where fire spews great clouds of black ash. In these forges the Onear crafted their spears and bricks, and their smiths were skilled. They taught their art to the Forhos, who took it to the hot places of the Morinuin. There they found hidden volcanoes and rivers of molten rock in the center of an underground sea. This place they named Histamin, as it was shrouded often in the steam of the sea. The arts of the Onear were taught to the Forhos by the forge master Aluru, and the Forhos took quickly to it. A people of the sea they originally were, but the Forhos had long resided in the mountains and the hills of Dor Dal and were glad to take on new projects. Among them the Areil learned fastest and were the most skilled, using their magic to shape the metals and crystals in delicate patterns. 

79,549 BBT—The ships of the Aluhos reached each of the Naa’waith cities, warning them of the fury of Bash’Tikish and the Domringel. Each of the children of Selor close their Anariima to sever their connection to Corilya. The cities of the Naa’waith were silent and isolated, and they began to prepare their shores for assault. Great walls, towers, and fortresses were built around the cities of the Naa’waith to hold back the eventual tide of Gurohtar. The leaders of the cities called out to all Naa’waith to return and find refuge behind their walls. Many answered, though more still refused the call, instead lying in wait beneath the light of Memaska.

79,451 BBT—To the tall and dark forests of the north Bash’Tikish went to find his champion, the monster Hurus. Long had Hurus stalked the forests there, and his might had made him grow large and wild. Lose Loss’kelvar that did not follow him feared him, and Bash’Tikish came to him then to will the beast to aid him in breaking the Tinechorn. Slow and deceitful were Bash’Tikish’s words, but full of passion and fury too. They swayed Hurus, but when the massive wolf was free of his forests Bash’Tikish and his Gurohtar chained him and bound him strong. They brought him south, starving him and spurring his anger. Hurus had not yet tasted death as the Gurohtar had, but Bash’Tikish made him feel pain unlike any other, and made him rue the theft of Agash’Kelesh.

79,449 BBT—The mists of the north came down upon the Tinechorn, and through them the peoples of Harenyast could hear raucous thumping, cracking, and crashing. A deep howl came forth from the mist, shaking the stones of the mountains and the hearts of the defenders. The howl parted the mists, and Arndulin looked form his perch as the Anariima illuminated the champion of Bash’Tikish. A wolf of great size it was, that its chains were carved of heavy stone and weighed many tons. Its breath was hot in its anger such that it burned, and fire spewed from its jaws and smoke rose from its nostrils. Hurus, the Hound of Wrath, had come. As it rammed the Tinechorn, ripples of power moved through the shield. The Gurohtar around Hurus chanted and howled, and Bash’Tikish bared his black teeth in a smile. Arndulin Nuragon saw that the beatings of Hurus shook Corilya, and he knew the Tinechorn would not hold much longer on its own strength. He summoned Vantura and his trained Areil, and together they stood upon the brow of Harenyast and held the Tinechorn against the Gurohtar.
	From the slim windows of Harenyast many Forhos peered at the monster Hurus, and at the beasts roaming the Telkohar. One among them, Nithos, was shaken by the might of Hurus. Thoughts raced through his mind—and Bash’Tikish, grown in his power, could smell his fear. Bash’Tikish called out to the Forhos beyond the Tinechorn, “The power of Agash’Kelesh is shown here, for our bodies are dead and rotting, but our hearts still burn red with the blood of Keleshe! For your thievery of Keleshe you must repent. Keleshe will take you in your repentance, know this!” Bash’Tikish turned his lone eye to Nithos. “You too can feel the power coursing through our veins. The blood of Keleshe is hot, let its warmth wash away your shame.” The words of Bash’Tikish ate at Nithos’s mind, and he fell to the ground, his shame and fear tearing at his soul. He flew from the window and threw himself down the many stairs of Harenyast. Bloody and broken he was at the bottom, and he lay, shivering in his blood and wanted nothing but to be warm—to be safe from this siege. He lifted himself and staggered then to the depths of the Morinuin, where in the darkness he curled himself.

79,448 BBT—Long did Hurus throw himself against the Tinechorn, and with each crash upon the shield Arndulin, Vantura, and the Areil became weaker. They knew that they could not hold the Tinechorn long, and with this knowledge they began preparing for the coming battle. Vantura trained long with her blade, and in secret Arndulin met with Aluru the forge master to craft her one unlike any other. Aluru shaped the blade of an alloy and mixing of metal and hard crystal, and Arndulin wove into it the magic of the north winds and set within it a cord of Corilya which he drew, so that the blade should ever be as sharp as the cutting cold, and each strike should freeze the blood of her enemies. The blade was long and pearly white, and Aluru did name it Glosalagos, the Snow-white storm. Long the two toiled in the Histamin, and long did the Forhos all work in the heat of the forges, their hammers ringing like the bells of Ringwetal. The blades and spears of the Forhos were sharp and long, and their armor was light and shone red with the gleam of the fires.
	Nithos lifted himself from his cave in the Morinuin, and came forth secretly among the Forhos. Cloaked in shadow he was, and he strode among the halls of Harenyast and the streets of Thuricaras telling of the glory that could be felt if they embraced the blood of Keleshe. Nithos had fallen to Bash’Tikish’s words, and he spread his thoughts among the Forhos and Tralonear. Tainted with magic were his words, glossed with promises and falsities, for Nithos was a skilled Areil among the Forhos, and many began to believe his words. For themselves they forged many weapons in secret, knives, hooks, axes and chains. Nithos meets with them in the alleys of Thuricaras, the cult forming around the belief of Keleshe. The cult begins to plan their acts of repentance to Keleshe. Nithos is not driven by the freedom of Agash’Kelesh like Bash’Tikish, but instead by its everlasting imprisonment. He feels the power course through his veins, and believes in the power of Keleshe and the red moon, but wants to feel that power forever. Nithos plans to get close to Bash’Tikish, but then cast him down as leader of the Gurohtar.

79,446 BBT—The beating of Hurus never ceased, but all else was silent at the Tinechorn and at Harenyast. The time was nigh, though, that Nithos would carry out his plans. The night was dark, with Memaska’s light hidden behind thick clouds when Nithos sneaked to the Perch of Arndulin and stole from the mountain top Corilya. When the Forhos grasped the Anariima he felt the rush of power that it brought, the resonations of his heart and soul that echoed through him. In that moment, the Tinechorn fell, and the beating of Hurus was heard no more. Bash’Tikish and the Gurohtar were silent, but then they let up a howl that moved with the wind throughout Dor Dal, and over the waters to Tol Atya and Dae Ithil. The Loss’kelvar charged the fortress, and the sudden attack startled many of the Forhos, who were unprepared. Arndulin quickly donned his brilliant emerald armor and handed Glosalagos to Vantura Hylama, who drew the shining blade and called to the defenders, rallying them to the gates of Harenyast. Aluru the forge master cleared the halls and rooms of Harenyast of those working to defend: the children, mothers, elderly, and unable. They were led to the hidden city Thuricaras, where they awaited news from above. Arndulin, unknowing of Nithos’s treachery, ran to the top of the fortress. Seeing that Corilya was gone, he knew that the Forhos were betrayed. Nithos had retreated to the gate of Haldarda, where he was rallying his own men—those of the cult of Keleshe. Vantura was leading the defense of Harenyast’s main gate. Hurus’s fiery breath seeped through the cracks in the gate, and the monstrous wolf slammed himself against it, shaking the foundation and shattering the wood. The head of Hurus reached through the first hole and spewed forth fire upon the Forhos near him. Hurus retreated and made a second mighty throw at the gate of Harenyast, and with this the door broke open and the monster fell through. Gurohtar streamed through, their fur unprotected against the long pikes of the Forhos. Many of the wolves were stuck upon the spears, but each wave continued climbing over the last. The shields of the Forhos were tall and strong, and the claws of the Gurohtar could not breach them, and many of the wolves felt the pain of death an innumerable number that day. Hurus filled the hall of Harenyast with his mighty stature, and quickly pushed his way into the open rotunda. Not further could he slink, though, as Vantura’s blade caught his ankles, and he felt the Glosalagos bite into him. The freezing wrath of the waves doused themselves upon the blood of Keleshe then, and Vantura dueled with the monster in that hall. Nithos and his men came then from the depths, and they flanked the Forhos in the great hall. Their knives and swords cut into the backs of the defenders, and many of the Forhos were trapped. Reinforcements from above came down upon their new adversary, though many were confused by this betrayal. Nithos called across his forces, battling in the hall, urging them to slay those who would not repent to Keleshe. Bash’Tikish strode then through the shattered gate and reveled in the chaos. He saw Nithos and his men and smiled at the betrayal that he sewed. Nithos spied Bash’Tikish too across the sea of shining armor, and he climbed the stairs of the hall and raised Corilya high. The glorious stone was tainted red with the blood of Keleshe that ran through Nithos’s veins, and the red light gave courage to the Gurohtar. Bash’Tikish saw then that Nithos had felled the Tinechorn, and he trusted the Areil. Nithos made for the Perch of Arndulin, for he sought to slay the Lord of the Forhos. Nithos climbed the stairs and emerged on the Perch of Arndulin to the sight of the prince himself. The tower was surrounded with the smoke from below, and Arndulin looked at Nithos in fury. He yelled at him, “Why, Nithos? Why would you betray us?” He cursed him then, naming him Rhach, and dooming him to live an everlasting life of suffering, with his body rotting from him, his bones turning to dust below him, those he trusted betraying him, and living only with the blood of Keleshe that he so desired. Arndulin brandished his spear then, and the two fought. Rhach Nithos was empowered by Corilya that he held against him, and his magic came in bold strokes. Rhach Nithos was arrogant, and his steps were reckless. His power was mighty, though, and with his magic marred Arndulin’s face and blinded his left eye. The betrayer underestimated the powerful lord that he battled, and Arndulin drove his spear through his heart. The Areil felt the pain of death for the first time, and the blood of Keleshe warmed his mouth. Arndulin Nuragon pinned his body to the stones of the fortress, for he knew that the corrupt Areil would not remain dead. Arndulin took Corilya then and plunged it into his empty eye socket, and he was wracked with pain as the Anariima’s power rippled through his body. The prince channeled his power though the stone, and as the light shone from it he could see far, and in incredible detail. The floor and walls were no obstacle, and as he moved down the tower he realized that he could see the motions of the future and the past. With each stair down the spine of Harenyast, he could see the doom of his men. Battle raged in the halls of the fortress below. Vantura rallied the Forhos and Tralonear around the frozen body of Hurus, the blood of Keleshe solid and binding in its veins. Their shields they locked in a wall against the Gurohtar and the traitorous followers of Nithos. Arndulin came to the great hall then, and Corilya his eye shone with the light of the universe yet veiled. The light of the Anariima shone of the shields of the Forhos and blinded the Gurohtar and the traitors of Nithos, and Arndulin Nuragon called to Vantura to lead the Forhos to the gates of Haldarda, for there shall the Areil make their stand. The light of Corilya gave them passage, and Arndulin Nuragon spoke to the attackers. He turned first to the Gurohtar and to Bash’Tikish who was burned by the light, “Your pursuit leads only to shallow rest, and doomed are you to awaken to pay the price of your evils.” Next, he turned to the followers of Nithos, and spoke harshly, “Unfaithful are each of you, guilty of abjuring the blood of kinship for the hope of the fiery blood of Keleshe. Damnation only will you feel, and you shall be named Rotasere, forever cursed to walk Tyeluum. Loyalty shall always slip from your grasp, and your trust for all things will crumble.” Their doom he spoke, and with his power he set it in time. Arndulin then raised his hands and brought in a cloud of mist to hide his escape to the gate of Haldarda. The passage to the underground caverns were hidden in the basements of Harenyast, and the Gurohtar could not find it.
	Bash’Tikish and the Gurohtar had taken Harenyast, but Corilya yet evaded them. Bash’Tikish climbed to Finiel’s prison and smote his chains apart. The King of the Naa’waith rose and bowed before Bash’Tikish. The Lord of the Gurohtar looked upon Finiel as a pawn, a weapon in his game. Finiel’s eyes were clouded and dead—but behind them hid a fire more furious than any other. Finiel served Bash’Tikish, but seeks only the utter destruction of Keleshe and Agash’Kelesh. All ties that Finiel had were shattered and forgotten—the only thing that remained was hatred and vengeance. He followed Bash’Tikish. Nithos pried from his body Arndulin’s spear, and stood in front of his host of Rotasere. Bash’Tikish came upon the Rotasere and grabbed Nithos by the throat. He smiled as he examined the traitor of the Forhos, and Nithos spoke then: “I know the passage to Haldarda.”
	Arndulin Nuragon meets with Vantura Hylama as the soldiers of the Forhos fortify the gate to Haldarda. Those of the Forhos who cannot fight are led by Aluru through the Morinuin, along the supply routes to the southern coasts of Dor Dal.
	Vantura stops Arndulin and speaks with him, telling him to escape to the Morinuin. There Corilya will be safe, and the doom of the Nalanat can be held. Vantura and the captains of the Forhos would hold the Haldarda, and if they would fall, then at least they would delay the pursuit of Arndulin. Seeing the truth that Vantura spoke, Arndulin Nuragon left through the alleys of Thuricaras and disappeared into the darkness of the Morinuin with the host of Aluru. The battle at the gates of Haldarda was fierce and lasted many months. The Forhos destroyed the great bridge before the gate, sending many of the Gurohtar and Rotasere into the heart of the world. The bridge was slowly rebuilt, but each attempt was destroyed again by the defenders.

79,445 BBT—The long struggle at the gates of Haldarda came to an end as the Forhos and Tralonear were pushed into Thuricaras. In the streets of the city they fought and died, and the silver towers of Thuricaras were filled with death. Vantura retreated into the Histamin, where many of her offenders found themselves in molten prisons as she froze the very heart of the forge. The few remaining defenders fell there, and Vantura was struck down with the swords of the Rotasere. This became known as the Battle of Thuricaras, but it is rarely spoken of as there are few who lived to record it.
	The mountain pyres of Dor Dal had alerted the Aluhos of the fall of Harenyast, but Castion and Rosiell could not bring themselves to hide their peoples beneath the stones of the earth. Knowing that they could not remain in Numendor, for only death could the land bring them, they tarried. Castion was bound to Dor Dal by his love for Arndulin, and would not leave him to face the storm of the Gurohtar alone. He rallied his people upon the southern shores, and they build many fortresses, walls, and secret delves. The greatest of these was Erhanca, the lone and mighty castle that rose from the waves off Dor Dal. There Castion took his throne as Lord of the Numen’waith and orchestrated his protection of Arndulin. Rosiell was less bound to the lands of Numendor, for the waters of that land were the same as the waters of the wider world. She and her Aluhos took to their mighty fleet of ships. Thence the Aluhos and Aluru’s Forhos and Tralonear departed from the shores of Numendor, and the Onear swam from their colonies. The refugees sailed to the island city of Templatal and were welcomed by Eres its queen. They brought the tidings of the fall of Harenyast and the breaking of the Tinechorn in Numendor. They sing songs of the valor of Arndulin and his companions, of the sacrifice of Vantura Hylama, and of the betrayal of Rhach Nithos. Aluru and the Forhos brought their mastery in the craft of weapons and armor to the Naa’waith of Templatal, and forged for them an array of arms.

79,000 BBT—And thence it came to pass that Arndulin Nuragon wandered long in the dark of the Morinuin with only two companions, those of the Areil most courageous: Carelwen the Swift and Tilmore the Tower. The trio had no means been idle, however, for they set it upon themselves to torment Bash’Tikish for his hatred against the Naa’waith. They turned first to the ancient forges of the Onear in the north and crafted for each of them arms worthy of legend, for each was infused with the light and power of Corilya. For Carelwen was made the short sword Alca, that would shine the brighter with each swift stroke that it made. Ramalin, a great and broad sword, was crafted for Tilmore. The tall and dark Areil was ever a shaper of the earth, and as he wielded Ramalin he could shape the very light of Corilya with his strokes. A final weapon he made—the spear Lasafen. The spear Lasafen Arndulin made for himself, and into it he sewed his fury, his sadness and suffering, his longing, and his hope. Lasafen he made the deadliest and most dreadful of all creations of the Forhos, for it struck the souls of those it slew, destroying them entirely. Those souls could know no afterlife, no ending, no finality. They were simply snuffed out.
	The three companions mastered the roads of the Morinuin, and with their arms they fought the Gurohtar scouts and searches from all directions. They seemed at all places at once, striking then melting into the shadows once again. Never still were they, for Arndulin knew that Finiel could detect their hideout each night, and the Gurohtar moved closer. Many dark places they hid, but never tarrying long before striking out once more to take the fight to the Rotasere and Gurohtar. Bash’Tikish and Finiel they did never find, for the two commanders never delved in search of the outlaws. Many fell to the blades of the trio, among them Hurus the monster, whose soul was blinded and crushed by Lasafen. Led by the foresight of Corilya, Arndulin remained one step ahead of those in search of them, and each tunnel was an ambush for the undying wolves of Bash’Tikish. After long, Bash’Tikish knew it to be folly to search further for Corilya. Ships would take them to the cities of the Naa’waith, but with each fleet they raised Carelwen the Swift burned the ships under the protection of Tilmore the Tower, for Carelwen could move quicker than even the northern winds, and Tilmore stood taller and mightier than any of the Gurohtar. Bash’Tikish howled in anger as he realized that his fury was confined to the shores of Numendor.
	The three companions found themselves not without aid—the Aluhos under Castion the Shipwright brought further confusion to the forces of Bash’Tikish. The Aluhos would raid the coasts and venture throughout the land, feigning the presence of Arndulin Nuragon. These ghosts distracted the Gurohtar and made their search for Arndulin Nuragon ever the more futile.
	For hundreds of years did the fortresses of the Aluhos hold against the Gurohtar, spread thin by their search. In time, though, Finiel unified the disheveled Gurohtar and brought them in force with the Rotasere against Castion. The stones could not hold back the tides of dead, and Arndulin Nuragon watched with his companions as the fortresses of the Aluhos fell one by one. Finally, the black stones of Erhanca were stained red, and Castion the Shipwright was slain on his throne. Finiel took that seat as his own, and the search for Arndulin Nuragon, now Lord of the Numen’waith, continued.

78,500 BBT—Long the fellows of Arndulin Nuragon fought in the Morinuin, sometimes even coming to the light of the moon to evade the eyes of the Gurohtar. Little did they find in Dor Dal, however, as the armies of Bash’Tikish and the lies of Nithos had corrupted the trees and rivers. The Dym’waith of Dor Dal had fallen prey to the coercion of Nithos, and many had joined his cult. It was to one of these hidden cultists that Arndulin and his followers finally fell to.
	When nigh on one thousand years had passed from the fall of Harenyast, Arndulin and his companions happened upon a secluded cottage in Dor Dal. Coming into the building carefully, they spoke to its owner, an old Dym’waith named Sogor. After they spoke to him, they decided he was not corrupted by Nithos but worthy of their trust. They slept there for three days, but unknown to them Sogor slipped out in the dark of the first night and whispered of their coming to informants of Nithos, for Sogor was a deceiver and in the employ of the Rotasere. On the third night, the cottage of Sogor was beset upon by many Rotasere and Gurohtar as their targets were sleeping. The Areil escaped from the cottage but where hunted close through the forest. In a valley they slipped, but their enemies drew too near. Tilmore turned and yelled to Arndulin and Carelwen to flee. The dark Forhos turned to the Gurohtar creeping closer and bellowed as he swung Ramalin. The broad blade cast out great waves of light, and before he was felled Tilmore cut down higher than one hundred of his attackers.
	Carelwen and Arndulin retreated north then, hiding in the trees and the rivers. But Nithos had spies there, and the two were followed. At the crossing of a wide river they were ambushed. The Rotasere and Gurohtar fired arrows at them from high ridges, but Carelwen moved Alca with a speed unseen before, cutting the darts from the skies. The forces and arrows were too numerous, and Carelwen was caught with four before she was killed. Arndulin took her body in his arms and escaped down the river, blinding the attackers in the night with Corilya. The river was low and swift, and he followed its banks to Luth Lirill, his old home. There he hides for a time, and buries Carelwen in a sepulcher in his hidden grove. In his time in Luth Lirill Arndulin works upon his old ship, that lay still unfinished in the bay. As he finished the mighty ship, the greatest that would ever grace the seas and the skies, he named it Enyalis, and he sent it out onto the bay. Arndulin turned to see the hosts of Nithos and Bash’Tikish before him. Finiel stood at their van, and walked over the white sands to his old friend. Arndulin was weary, and he knew his time had come. The greatest of the Forhos considered Finiel’s eyes long as he approached, but turned towards the water. He spoke softly, “I see light of the moon in the waters of the bay like the light of the flowers upon Amon Celefin. Do you remember those flowers, how they shined?” Those were the last words of Arndulin Nuragon as Finiel, King of the Naa’waith struck him down and took Corilya from his eye. The valor of Arndulin is sung in the epic song Vahalis, which is the telling of the Naa’waith peoples.
	On the beaches of Luth Lirill something happened that Arndulin had not accounted for, and that the Naa’waith had not prepared for—Finiel, in his mighty power and drive, channeled his being through Corilya, and spread his fingers of power across the world. His power stretched far, and it felt the presence of Onsinta in Halletal. He forced the Anariima open.
	

78,500 BBT—The children of Tolas and Selor prepare for the worst. Finiel and Numendor are silent as the world prepares for war. The Rhun’waith and Ralun raise the Iantar to cover Rhunendor and hide Celebtal. Renno calls to the Okemenel and the Annonear for aid. Comenraan and Habbakuk answer in like, both agreeing to help the Annun’waith. The heart of Comenraan was deceiving, though, for he plotted then to destroy the wondrous city of the Annun’waith should it come under siege. Habbakuk was true in his words, but his fear for Comenraan seeded still in his heart. Should Comenraan’s wrath show, Habbakuk knew not if he could endure.
-Naa’waith culture is still developing during this time, and with the raising of the Iantar, some of the Rhun’waith take to artwork. Rhun’waith art takes the form of beautiful landscapes, as they used their magic to shape the world around them. Among these artworks is Lanta’aluyosto.
-Renno sits in wait in her council room in Halletal. She looks out over the city—the tall pillars and smooth walls of marble hide the bustling Annun’waith. Suddenly, the city flashes with a blue glow—the gate of Ringwetal is reopened. The sky seems to go dark as Finiel steps through the gate. The Annun’waith bow before him, their king. Finiel unleashes a storm of magic, ripping and tearing the reality of those around him apart. Loss’kelvar flood through the gate, Bash’Tikish howling to his lost moon. Renno knows she does not have the power of Finiel—she cannot best him, not can she close the portal with the Loss’kelvar bearing through. She meditates and sends a message to Eres and Ralun of the fall of Halletal. Storm clouds roll over the city, and in a horrible extermination, Finiel calls thousands upon thousands of lightning bolts down on the peoples of the city. The light show can be seen for miles, and Comenraan takes notice. He breaks open the cage of Raeg the Beast and lets the fell dragon fly for the first time. Renno stands upon the balcony of her temple and manifests a great force of protection—a shield to push back the attackers. Her grand barrier is a sight to behold, its unbreakable force emanating like a wave on to the Loss’kelvar. Raeg the Beast dives down towards the temple, shattering Renno’s wall with sheer force of body. The dragon crashes down on the temple in Halletal and battles with Renno. The daughter of Tolas and Selor summons forth great blades of power and plunges them into the creature, but Raeg seems to only grow each time he is dealt a fatal blow. Renno falls to the claws of Raeg, and Halletal is destroyed. Farriel is slain by Finiel, and the great house of the Annun’waith is cut short. Comenraan revels in the destruction, and he congratulates the Loss’kelvar. Bash’Tikish stands on the body of Renno in the heights of Halletal and announces himself to be Vash Bash’Tikish, ruler of the Keveshkek nation. Comenraan meets with Vash Bash’Tikish and declares his intent to work with him in destroying the Naa’waith. Comenraan’s jealously of the Naa’waith magic knows no bounds, and he will not rest until he slays Tolas and Selor, the two who dared match his power. He seizes the Anariima Onsinta and uses its power to swell to unprecedented size. His form dwarfs mountains and seas, and all that see him name him the Infinite Jaws. Comenraan’s pride and envy drove him towards one goal: the consumption and possession of all power. In this he was most afeard of Finiel, for Comenraan could see behind the dead of Finiel and knew that there was an impossible power there.
-Habbakuk watches the destruction of Halletal from the seas. Renno, his teacher and love, is dead. Habbakuk approaches Raeg the Beast in the cover of Avcolla as Comenraan meets with Vash Bash’Tikish. Habbakuk seeds doubt in the mind of Raeg, and begins to turn him away from Comenraan. The mighty creature Raeg may be the only being able to defeat Comenraan.
-Eres hears Renno’s message. Immediately, she knows to hide the city. The Anariima must be moved, hidden in some place inhospitable to life—if the Loss’kelvar pass through it, they will die. Eres throws the Anariima of Templatal into the sky, and it studs the black as the first star.  When Ralun sees the Anariima in the sky, he does the same with the Anariima of Celebtal. The twin artifacts are pinpoints of light. The two cities are hidden now, and in their sorrows, the Naa’waith of the two cities create points of light on their fingers and hold them into the night. Eres shades the mighty city of Templatal in a magical fog, and Ralun pulls the Iantar close around the city of Celebtal, shrouding it from the outside.
-Vash Bash’Tikish sends out some Loss’kelvar in the shape of white hawks to find Templatal and Celebtal. Comenraan sends Raeg to fly the world as well. Finiel retreats to the fortress Erhanca and surrounds himself with the petty Rotasere. Nithos stands at Finiel’s right hand.

78,100 BBT—The spies of Comenraan find the southern islands of Tol Hith, home of the Gatekeeper Revion. Finiel’s joy is unbounded—Tol Hith is the one place where the prison of Hera’roilya can be seen and reached, there may be the one place that Finiel can destroy Meval and curse the Gurohtar to everlasting damnation. The combined forces of Finiel, Vash Bash’Tikish, and Comenraan made to slay Revion and take Tol Hith. The winds of Yvari were relentless, and the waves brought many of the attackers to lie forever beneath the waves. Even mighty Okemenel could not withstand the winds and waters of the World Storm. The pursuit of Tol Hith was ended, though, as Revion turned from the endless cosmos over the rim of the world and did the one thing that Finiel could not predict—he ripped the rim of the world from its very plane of existence. Revion, taught the ways of magic by Tolas and Selor themselves, had long studied the manipulation of existence. Years he stared into the endless cosmos, and even called at times to Hera’roilya. Only once was his call answered, but the words of Hera’roilya forever echoed in the ears of Revion. The Gatekeeper of the Endless Realm unlocked the power of the multiverse, and in this moment showed his full power. He ripped the rim of the world and the image of the cosmos from the Prime plane and set it apart. There the island of Hera’roilya remains, the one place that it can be physically reached. Tol Hith can be found no longer on the waters of Tyeluum, and Revion stands still over the gates of the universe. The assault on Tol Hith ended there, with many of the attackers sent spiraling into the bowl of the world, over the falls at its new rim. Finiel, Comenraan, and Vash Bash’Tikish return to their broods and continue the search for the hidden Naa’waith cities.

78,000 BBT— The isolation of the Naa’waith is over as Templatal is discovered by Raeg. Eres knows that war is soon to come. Eres tells her people that Templatal will not fall as the other cities did—but she knows the truth. The city and its people can hold for generations, but their foes are lifeless, deathless, timeless.  Long she thought on this problem, and her research led to one solution—the creation of a fifth Anariima, this one different from the rest. She calls Ralun to her aid, and together the two powerful magi create the Eres’s secret weapon: The Fifth Door, Andolem. Andolem is an inverse Anariima, black in its sheen. It does not resonate the power pushed through it, but instead it absorbs all. If enough power is fed into the gem, it will grow and swallow all. This is Eres’s goal: for the evils of the world to be engulfed in the shadow of the gem and isolated. The power required to grow the Andolem and swallow the island would be immense—and so Eres planned to release Meval and use its energy to power the Andolem, when each of the powers were present to be trapped. 
As the fateful day arrives and the one-hundred-thousand ships of the Loss’kelvar approach the shores of Naa’yamen, Eres calls one last time to Tolas and Selor for council, for she fears that her choices concern too many lives, and may lead to the destruction of all. She doesn’t know the full power of the Andolem, or what place it may hold within that she would damn all in the land to. As the wind blows through Eres’s hair, Tolas and Selor appear for a final time in the Naa’waith forms.  Tolas and Selor stand before their child, and Selor kneels to comfort her. She speaks words of wisdom to Eres: “The world will go on. This is not the fateful end, but only the beginning of a wider future, filled with many flowers, laughs, and tears. The time of our people is waning, but others will come after us in a world given truth by the events soon to follow. You, Eres, will shape the realms, and I will be forever proud. I will be watching over your world, Eres, with time as my only judge.” With that, Selor gave herself to the planet. Her arms became branches, her hair leaves, and from her eyes fell tears of shining dew. She was the first tree of knowledge, and her roots delved deep into the world. Her cuttings would influence the world long into the future. Eres turned then to her father, Tolas, for strength. “Surely, father, there must be something I can do to aid my people? I fear my path will only lead to their bitter end.” The tall Naa’waith kneeled to his daughter and spoke softly. “Your people have lived glorious lives through your guidance. The Naa’waith people have reigned more brightly than any that shall ever come, and now they give their lives for you with joy. Your path is not to preserve your people but to give way in peace for those that follow. I can give our people an enduring legacy, and I shall tell you this: Children shall awaken in our likeness, and shall be young and fleeting. Their hearts will hold fonts of love and adventure, and all colors will dance in their eyes. Their minds, though, will be untouched, born as slates of clay yet to be scratched—but for one. One child shall awaken with a message from you, Eres. What do you pass to this child?” Eres, the final queen of the Naa’waith people spoke confidently, “This child I shall give our song, the Vahalis. For in this song is the language we breathed, our passions to be held, our mistakes to be learned from, and our lives to be known. Let this child know who came before, and let her know that no matter the world she awakens to, beauty can be found in the hearts and minds of all.”
As Vash Bash’Tikish and Comenraan fall upon the city, Tolas opens his hands and dissolves into a field of golden butterflies—millions of them lilt from his fingers, shining like one million suns over the morning horizon. The magnificent sight blinds Comenraan and the Loss’kelvar but empowers the Naa’waith. Their magic becomes bolstered with the light of Tolas. The butterflies of Tolas speak alone to Ralun, and give him a new quest. He leaves Templatal and meets with Habbakuk in the waters deep below Annuntol. Ralun speaks to the ancient serpent, now grown large and furious after the death of Renno. Though Ralun leaves Templatal in its greatest time of need, he understands his destiny lies in swaying the mind of Habbakuk.

76,000 BBT—Rarely could the great figures of power be seen in the field of battle after these thousands of years of war. The long years have wearied them, and often they stayed in their strongholds. Comenraan took this moment to move on his plans to slay the other leaders and claim their power for his own. He sends Raeg to Erhanca to kill Finiel and moves to destroy Vash Bash’Tikish himself. The mighty Okemenel was proud and reckless, though, and as he moved to trap Vash Bash’Tikish beneath his claws the Loss’kelvar leader moved swiftly. Comenraan underestimated the cunning of the desperate Loss’kelvar. Vash Bash’Tikish climbed up the arms of the great being and stole the Anariima Onsinta from Comenraan’s chest. Wielding Onsinta, Vash Bash’Tikish struck the heart of Comenraan. The mighty Okemenel shivered as his power was lost in an instant—without Onsinta he could not hold his form. He fell to the ground in a shattering crash, mountains crumbling beneath his still-deteriorating weight. The strikes of the tiny Loss’kelvar dug deeper and deeper into the form of the Okemenel until Vash Bash’Tikish began to sap the very power from the core of Comenraan. The Okemenel writhed in agony and ripped Vash Bash’Tikish from him and escaped into the fog. He returned to his vast caves and wallowed, filling great caverns with his blood and causing great earthquakes with his lurches.
	Finiel sat brooding in Erhanca as the voracious being Raeg bore down upon the fortress. In a single clash of the dragon’s force against the stones the fortress was leveled. Many of the Rotasere were lost to the maw of Raeg then, sentenced to be forever digested by the monster, unable to die or be freed. Nithos was not present, for he was afield in Naa’yamen. Finiel rose from the rubble of the stronghold shrouded in the smoke of the act, and began to work his soft words into the mind of Raeg. He probed and found the words of Habbakuk earlier planted. Taking this seed, he grew a vile garden of deceit and confusion in the mind of Raeg. Finiel turned the monster against its creator, and neither were seen in the battle for Templatal for many years.
	-The Vahalis sings of this time as the Luumedag, the Time of Battle.
-It was during this time that Vash Bash’Tikish created the vilest artifact to exist. At its core was Ramalin, the blade wielded by Tilmore the Tower, forged by Arndulin Nuragon and inset with a cord of Corilya. Vash Bash’Tikish corrupted the blade, wreathing it with shadow and pain with raw belief. Thus, the righteous blade of light fell to the deepest darkness, and it was Ramalin Arpohda, Arpohda, the Thief Blade. The blade held not only a cord of Corilya then, but so too a string of Onsinta, and it stole not gold, not power nor joy, but belief. Those it slew knew not any legacy, for in the minds and hearts of those that survive it, the being never existed at all. Reality it rewrote, and it is the only blade that could turn man to god. Even Vash Bash’Tikish knew this blade to be vile and he used it only in jealously.

73,000 BBT—After a generation of war, Templatal was left in ruins. The Naa’waith still fought—never forgetting the light of Tolas. Vash Bash’Tikish used Onsinta to break the lines of Okemenel and Naa’waith alike, but Naa’waith from across the continents have taken the call and brought aid. The Neve’waith under Nestadis came from the west with great ships and wild animals. The wild Neve’waith fought ferociously alongside the Naa’waith in the land of their home, but Nestadis was slain, her throat ripped apart by a great host of Gurohtar. Many others were slain in the Luumedag as well: Rosiell by a force of Okemenel, and Aluru by a poisoned arrow. Eres took to the front lines, using her magic to hold the powerful enemies at bay. The Naa’waith have little land left but the grand temple, and they stand against the rising tide of Vash Bash’Tikish and Comenraan. In a glorious day, the forces of Habbakuk fall upon the coasts. The Onear launch attacks from the watery canals of the city, and Habbakuk unleashes a barrage of powerful magic at Comenraan. Ralun returns to Templatal at the head of the Annonear, and he rejoins his sister Eres. The two cannot rejoice long, though, as the fog parts to the return of Finiel and his Rotasere atop the back of Raeg the Beast. Finiel sets Raeg upon his creator while he and Vash Bash’Tikish make through the lines of the Naa’waith to the great temple at the center of the island. There, midst the violent battle under the soft light of the blue Memaska, the powerful Vash Bash’Tikish and Finiel meet Eres and Ralun. Behind them hovers the black gem Andolem, created in secret by Eres and Ralun to engulf the island. Now that Finiel, Comenraan, Vash Bash’Tikish, and Raeg are all there, Eres knows Meval must be released and the Andolem must be activated. She whispers to Ralun, “Tar mennai nanwenya.” (Stand until my return).
As she turns and opens the Andolem gate to the isle of Hera’roilya, the three behind her begin to circle. Vash Bash’Tikish knows the time for his victory is near, and he can sense the betrayal of Finiel. He knows that his tool plans to destroy him, so he moves to make the betrayal first. He strikes out at Finiel, but the Naa’waith was too swift. He dodged, but was struck by Ralun, his younger brother. The three then erupted in a swift combat, two Anariima among them. Multiple times the Anariima exchanged hands as each vied for their possession in the combat. Finiel knew that time was short for him to destroy Meval before it was unleashed, and his desperation led to furry. He turned to Ralun then and ducked under his attacks, pushing him towards the temple’s stairs. The vicious onslaught of Finiel caught Ralun off guard, and his magic faltered. In that single moment Finiel drove a spear through his younger brother—but as he turned, he saw Vash Bash’Tikish enter the Andolem.
	Vash Bash’Tikish emerges on the isle of Hera’roilya unexpecting of the brilliant spectacle that awaited him. The swirling stars and galaxies that waited beneath the world blinded him for a moment, and when he opened his eyes he saw Eres standing on the pedestal of Hera’roilya himself. In her hands, she held Meval, fragile in its shrunken state. The red moon pulsed with power and conscience, and as she lifted it the temple around them seemed to bend in space. “Hera’roilya, I release you and name you thus: Hera’rocoia. Hera’roba. Herya’rosintilya. Hera’rombar. Hera’rontan. Naa’roleith. These shall be your aspects, your lives, and your bindings.” Vash Bash’Tikish was stunned, paralyzed in fear as Eres strode towards the Andolem holding Meval, the universe swirling around her. He regained his mind and leapt at Eres, clawing for the moon that he died so many times for. He caught her leg and tore at her body, dealing her a mortal wound. Eres summoned forth all her power and unleashed it upon Vash Bash’Tikish there in the Rim, and she sent him reeling backwards. She staggered through the Andolem, reappearing in the chaos of Templatal. She holds the moon in her hands, cradling it like a child. Finiel sees the very object of his hatred and pauses, frozen. He reaches out to Eres, speaking softly, “Eres my sister, destroy Meval, damn our enemies to an eternity of suffering! This is your chance to save your people, our people!” As he speaks Finiel slowly moves closer, reaching to take Meval from Eres. Before he can reach her, his very own spear pierces his leg, thrown by Ralun. As the Lord of Celebtal sways in the fiery breeze, he utters his last words: “Tarnëni” (I stood). Ralun’s blood dripped onto the steps of the temple, great fire silhouetting his figure as he fell then, topping down the steps. Eres ran from Finiel and stood at the edge of the temple steps. She looked down at the city, ruined after thousands of years of war. Her people were few, the streets were filled with the bodies of the sacrificed. She threw Meval into the air. 
The world turns red, as if blood was poured over the sky. The Loss’kelvar look up and see the moon—and then Meval calls to them. Meval’s arms reach down and take its children unto it. As the red light of Meval shines over the world it takes the souls of the Loss’kelvar unto it. They are freed at last to live in eternal peace in Keleshe. In the instant that the great moon hovers above the city, Raeg falters and Comenraan strikes him a terrible blow and destroys him, absorbing his creation’s power and sending Raeg to the bottom of the seas. As Comenraan revels in his victory, Habbakuk pushes passed his cowardice and rises against Comenraan, the one who he has always lived afeard. Habbakuk wrapped himself around the Okemenel and strangled him. He forced him upwards and pressed him against the moon. Habbakuk held Comenraan unrelenting, and there the Okemenel could see the island. He saw the Loss’kelvar fall, finally, to the red light and go to Keleshe—and somewhere deep within Comenraan, he believed. The dragon sank into the moon, pulled to Keleshe forever. Habbakuk fell to the oceans, diving after the monster Raeg.
As the Loss’kelvar found their paradise, Finiel let forth a cry more terrible than any. He staggered to the open air of the steps of the temple passed Eres standing on the edge. His eyes were a frenzy of hatred, and he tore Corilya from his staff. He held the Anariima aloft, a challenge to the moon above him. He cursed Meval and all those inside, and began to unravel the fabric of Corilya. He took the power of the star-stone unto himself, and his soul became bound to its power. His eyes glowed like stars, his hair was thousands of strands of light. He called upwards to Meval, and unleashed his power in a mighty burst. A column of light flew into the sky and struck the center of the moon in its fragile state, and the unbridled power of Corilya unchained broke Meval into four pieces. As Meval broke, the red light of the world faded to a sickening gray, dousing even the light of Memaska. Vash Bash’Tikish strode from the Andolem then, only to see the power of Finiel destroy Keleshe. The burst destroyed Finiel, stripping him to his core—the very center of Corilya itself. The burst of Corilya fed into Eres’s creation, and the Andolem grew, swelling to engulf all. The power was far greater than Eres anticipated, and the Andolem grew to devour the world.
As the dark gem swallowed all, Hera’roilya opened his eyes. He breathed, moving air through his lungs once again. He stood from his pedestal and stepped forward. With each step, the world rocked. One. Earthquakes rock the surface of the globe, causing great waves to beat the shores. Two. The wisps that remain of Yvari after the sundering of the Rim rise to create a global storm, blowing away all. Three. In his last act, Hera’roilya tears the present from the Prime realm. He pulls the world covered by the Andolem away, isolating it beneath the Prime plane. He leaves the Prime plane in a state where the Andolem took only Naa’yamen, leaving the world untouched by its power. In separating the Andolem present, Hera’roilya creates a plane of Shadow lying beneath the world, and Onsinta in that present resonates this decree and forms another realm of its own will with the remnants of Corilya—an ethereal copy of the Andolem’s present. This plane Hera’roilya places above the Prime plane. Then Hera’roilya is sundered into six pieces: Good, Evil, Knowledge, Law, Creation, and Chaos, all aspects of the Architect’s power.
 This sundering of Hera’roilya spilled the world over, emptying the Universe around the nameless world. The two Anariima in the sky were joined by the countless stars, planets, and galaxies that were held beneath the world. The sun itself is spilled forth around the world. For a moment, the sun rose and was eclipsed behind Meval, a great light in the sky. The warmth touched the planet over, but only briefly before being hidden away once more. Good lays his hand upon the Andolem, the gem swollen to engulf Naa’yamen, and he soothed its fury. He took the Andolem into his possession, leaving only emptiness where Naa’yamen once sat. Eres, Vash Bash’Tikish, Raeg the Beast, Habbakuk, Nithos, and their peoples were all within the Andolem.
Hera’roba hovered above Naa’yamen, even more terrible than Meval. Doused in blood-red light, he brought the Naara’tela. This shatters Perakor, sending tentacles of fire across the world. Hera’roba rose above the fires of the Naara’tela in his terrible beauty, and he rent the souls of thousands. The fires of the Naara’tela cracked the surface of the world, raising mountains into the stars and lowering valleys to the core of the planet. The waters around Perakor boil and wretch, filling the air with steam and spilling over the land. In this doom, the flowers of Tolas unveil themselves. They give a legacy to the Naa’waith, forming creatures in the far north. These creatures are much like the Naa’waith, but they are simple—they are the elves. The first elves awaken in the violence of the Naara’tela with the language and memories of the Naa’waith in their minds. They hide as the world shakes. Ash rains down across the world, blocking out the sun that was only just born. This begins the Times of Ash.

%%%%%%%%%%%%%%%%%%%%%%%%%%%%%%%%%%%%%%%%%%%%%%%%%%%%%%%%%%%
%CHAPTER
%%%%%%%%%%%%%%%%%%%%%%%%%%%%%%%%%%%%%%%%%%%%%%%%%%%%%%%%%%%
\chapter{Chapter Two}

Yen Lith, The Times of Ash
The ash of the Naara’tela choked the skies of the world. The fiery arms of Hera’roba still moved in the clouds, burning the air of the world and causing great storms to rage on the oceans and the lands. Dunes of ash buried Perakor, the oceans turned to black acid, and the sun was not seen for thousands of years. The wrathful storms throw the ash to the wind, and great waves of acid burn the coasts. The lands of the world freeze, animals and plants perish, and the air is silent but for the rocking of thunder and whip of the wind.

72,000 BBT—The shards of Hera’roilya took root across the world. Hera’roba stayed in Perakor, exerting his influence over the shattered land. Hera’rontan emerged in the south, but he returned through his gate to the Eye of Ehlu where he began to exert his power to turn the Eye into a place of wonders. Hera’rombar emerged in the frozen lands of Numendor. There he took his seat on Tauras to watch over the world. Hera’rocoia took the Andolem and hid it in the deep places of the world. There he peered into it and watched over Selor’s growth. Naa’roleith emerged on Annuntol, and wandered about the island. He spun the colors and shapes of the island, creating massive contrasts and dynamic changes. Herya’rosintilya traveled to the last refuge of the Naa’waith: Celebtal in Rhunendor. He saw the strife and disorganization of the city and its people. With the children of Tolas and Selor gone, the people of the Silver City had no one to lead them. Herya’rosintilya took the Rhun’waith into his arms and nurtured them into stability. He guides their civilization to equilibrium, hiding them first from the aftermath of the Naara’tela.
-The new race of elves is grief stricken, barely living on the barren ash-covered landscape of Perakor’s north. One elf, on the brink of death from sickness, is touched by a flower of Tolas. A flame is kindled inside her, and she is restored to health. She names herself Liliath and leads her people on a path of unification. In her mind is the Vahalis, and she understands it and uses it to teach her company the language of the Naa’waith. She names them the Astar, or people of the dust, as their skin has been stained by the ash. She tells of the Estanesse, First Children, who came before them. Liliath leads her people towards civilization and agriculture.
-Of the Seim’waith little remained. Almost all had refused Eres’s calls to return to Templatal, and thus many had perished in the Naara’tela. Some, though, remained. In the days leading up to the Luumedag, Eres had sent out a small host of Ariel to protect the Seim’waith in Nevlonde. Among those Ariel was one of her finest students, Linarata. Linarata’s power protected the Seim’waith that she could reach, however she alone succeeded among the Ariel tasked there. Saeronder perished with the greatest cities of the Seim’waith, but with Linarata was Saeronder’s great-granddaughter, Reinien, then Queen of the Seim’waith, and yet her daughter, Enalië. After the Naara’tela, Linarata and Reinien regrouped the few Seim’waith that remained, sent many to travel the Nevlonde in search of survivors, of which few were found. The Seim’waith took on a new name around this time; Tel’waith, or End People. In time, Linarata and Reinien moved to the eastern coasts far from the fiery origin of the Naara’tela that still reeled in the sky. There the Tel’waith built a new haven, Metimacopa. There they slowly began to rebuild, protected by the shield of Linarata.

70,000 BBT— The Silver City sits under a gray cloud of grief. The Rhun’waith are held in a perpetually solemn mind, their emotions crushed by the Naara’tela. Herya’rosintilya guides them with his infinite knowledge, and they use their powers to prepare the world for those who would come after them. Among them, young Iulian became distinguished. Iulian learned quickly from Herya’rosintilya the ways of calm and grace. Iulian saw that the Rhun’waith could not further progress their people, but still could have hope for the legacy that Tolas promised. In a glorious memory of the Van’Yasseron, the Ariel of the Rhun’waith gather in the north and raise great Tuluryamar mountains with their magic, for Herya’rosintilya prophesized the coming of a mountain-loving peoples.
-Hera’rocoia, nurturing Selor in her new form, realized that she could not flourish within the Andolem, though her power is tied to it. The Shard prepares a means for Selor to live outside the Andolem. He speaks to Herya’rosintilya in Celebtal to learn of a way to release Selor from the shadows of the Andolem. Hera’rocoia spent many years in Celebtal with Herya’rosintilya. During this time, the two Shards worked together to clear the pain from the Silver City, washing the grief from the wound of the Rhun’waith. The city shone again, though in a different way. The light of the fire in the sky danced along the buildings and towers, and the Rhun’waith looked upon the disaster through the eyes of beauty, led by Iulian the Graceful. Herya’rosintilya retreated to a secret grove in the Tuluryamar range and looked over the hidden city from then on, leaving Hera’rocoia with the knowledge he wished for: Selor could be freed by combining the resonances of each of the Anariima, for their combined forms Andolem could not hold. With this, Hera’rocoia begins his quest. First, he looks to Taregil and Arangil in the sky, unable to be reached. He meets with Hera’roba of his desire for the Anariima, and to Hera’roba this task is passed. Of Corilya Hera’rocoia was troubled, for the Anariima had been sacrificed in the creation of the Ethereal Realm. He knew that the combination had to be there for its part to be fulfilled. The last Anariima was Onsinta, possessed still by Vash Bash’Tikish the Damned within the Andolem. It was there that he traveled first.
-As the Tel’waith learned to live in their new environment destroyed by the Naara’tela, Metimacopa grew. A new generation of Tel’waith came from the old, hinting at the hope that the Tel’waith may yet survive the desolation. Linarata was saddened, though, for long she reached to her old friends with her mind, but never did she receive response. While she believed the Tel’waith were the last of the Naa’waith in Perakor, she found hope in the idea that the city of Celebtal remained. Word had reached Nevlonde during the Luumedag of the aid of Ralun and the Rhun’waith, and Linarata could only hope the city remained, hidden though it was. In the many dark days that Linarata looked upwards, hoping to see the Memaska through the ash, she was graced with the sight of Taregil and Arangil. The sight of Linarata was the sharpest of all the Naa’waith at that time, and even through the smog could she spy the two Anariima’s glow. With her power, she granted the eternal sight of the Anariima to her people, and they knew it as the Blessing of Linarata. Wherever they were, they could see the light of the Anariima in the sky or on the horizon come death or the destruction of the stars.

69,000 BBT—The Blessing of Linarata led to the belief in a prophecy among the Tel’waith, that one day the Anariima would return to Tyeluum, and on that day the strife of the Tel’waith would end. Despite this prophesy, Linarata prepares for the day that she can no longer lead the Tel’waith. She begins to teach Enalië the ways of magic, and the Princess of the Tel’waith becomes the first of Saeronder’s people to become and Ariel. Enalië was the most beautiful of the Tel’waith, and from her face shone an echo of the glory of Memaska. Her voice was soft, and her singing was renown among all the Seim’waith before the Naara’tela. After the destruction of her people Enalië had saddened and sang naught once.

65,000 BBT-- Liliath’s drive to push through the horrors of the land she named Amarth brings the attention of Hera’roba. He is inspired by her resolve, and he meets with her. Hera’roba hears her story, her dreams, and her loves. Hera’roba sets before her a challenge: He shall allow her to live and to lead the Astar peoples, but she must retrieve for him the mightiest of all gifts. Above the endless grey skies, above the ash of the Naara’tela, there is a beauty surpassed by none: the Anariima, hidden among the countless points of light in the sky. Hera’roba gives Liliath the task of plucking the Anariima from the sky. Liliath smiles, and she accepts the offer.
-Long did Hera’rocoia search Andolem for Vash Bash’Tikish, and finally he found the cursed being. Vash Bash’Tikish held Onsinta close to him and was untrusting of Hera’rocoia but saw this as an opportunity. He whispered his secrets into a thread of Onsinta and extended it to Hera’rocoia but would only relinquish it on one request: That Hera’rocoia bring him a cutting of the tree. The deal was made.

62,000 BBT—The initial mass of ash and dust caused by the Naara’tela settled, lasting ten thousand years due to the furious winds of Yvari. The storms still blow the ash across the globe, and the boiling oceans clog the skies with clouds.

60,000 BBT—Ten thousand years after Liliath first led her small tribe to safety, she stands at the top of the Astalena, the mighty tower at the center of the elven city of Ithalas. Liliath brought the elves together, and now she watches over the construction of a city in the north. Ithalas shall stand as a beacon in the harshness of Amarth, a challenge to any who doubt the power of the elves. As the Astalena grows taller, it pierces the smog and ash that has choked the skies of the world. When Liliath first peers out upon the spilled Universe, the sky full of color and light, she gasps, “Coi’il deno’Nuinen!” With this phrase, she names the world Nuinen, or “Land Beneath the Night Sky”. Liliath calls to her sons and her followers to stand upon the peak of Ithalas, and they too see the stars and lights. Liliath speaks an oath on the light of the stars above her, on the names of Death and Life, that she would give her peoples the means to live under the stars, free of the ash. This oath Hera’rombar hears, and he then sets his eyes on Ithalas, the first and greatest city of the elves.

56,000 BBT—The Astar begin to use symbols to record songs, the first sense of a written history of the world.

54,991 BBT—Ithalas grew larger, and the Astalena grew taller. Each day, Liliath stood at its peak with her children, Mallorn, Mear, and Tirwen, and looked out over the scape. When the tower was complete, Liliath studied the skies: the Anariima were there, hidden, among the stars, but she did not then know the shine and glimmer of the Queen and King. Her people had developed over thousands of years, had built this tower into the stars, and grown in peace and knowledge, but she could not find her salvation. Long she thought, staring at the stars. It was a cold night, light and devoid of breeze or fog when Liliath understood: to her, each of the stars were an Anariima, for each reflected their light in turn. She reached up, and believed. When she brought her arms down, she held Taregil in her left and Arangil in her right. When she turned around she saw Hera’roba once more. The Shard drew threads of the two Anariima for Hera’rocoia before speaking: “You have done as you said you would. Your people are strong, and they walk through the wastes of Amarth without fear. Your life has not been in vain, Liliath. Now, it is time that you walked with me.” Hera’roba held forth his hand, and after Liliath handed an Anariima to each of her sons she took it, and thus the greatest of the elves parted the world. Mear and Mallorn held council upon the Astalena with the other followers of Liliath. The circle was eight strong; the first three being children of Liliath.
First and most known there was Mallorn, second son of Liliath. Mallorn was tall, with hair of shining silver. Mallorn was ever in love with the sky, and could oft be seen upon the height of the Astalena with Faranwe, his dear friend. He followed closely to Liliath always, and learned to believe in the power of the elves more than anything. The Astar were the font of all Mallorn’s love, and he gave his love back to them. Mallorn wanted naught but to save his people from the harsh scape of Amarth and deliver them to green lands to the south, rumored to him by his mother. He was strong and fiery of heart, but reckless and quick to judge.
Second was Mear, first son of Liliath. The first son of Liliath was tall as his brother, but dark of hair and bright of eyes. Mear was ever competing with Mallorn to win his mother’s praise, and he did so with great study into the means of helping the Astar. While the others saw the Naara’tela as an evil of the past, Mear saw it as the driving force of progress. Without the harsh environment, Liliath would never have built her determination to continue forward towards the sky. Mear wanted to understand how to harness the power of the Naara’tela to not only teach the future Astar of their origin, but to give them an everlasting power into the future. He saw the ash, the thunder, the fire in the sky as a force that could be tamed and used. He sought dominion over the sky, the earth, and the seas above all else to lead the Astar forward. Mear was much like his brother Mallorn, and all could see this. He was passionate, strong, and swift. His mind was keen and his heart powerful. Mear was wiser than Mallorn, though, and softer of temper.
Third was Tirwen, the third child and sole daughter of Liliath. Tirwen, though the youngest of the children of Liliath, oft took the forefront of conflict. She was strong of will and strong of heart, and she was the fairest of the three children of Liliath. Tirwen took most easily to counsel by the other followers, but in her anger could be unpredictable. Tirwen was the most ferocious, but also the most loving of the three children, and it was ever in her heart to protect those she loved. She was as a fire, chaotic in her mind but rigid yet in her beliefs. After Liliath’s death, Tirwen took to be the Lady of the Astar in Ithalas before another was appointed.
The others being followers of Liliath:
Faranwe, Chaser of Destiny. Youngest of the Counsel of Liliath, Faranwe was light of presence. Often, he could pass without others noticing, for he rarely spoke out or made himself evident. He was oft lost in his mind, for fleeting is his hold on the present. Faranwe dreamt of the future, of far off places not bleak as Amarth. A quick friend of Mallorn, Faranwe was drawn to the sky and the stars spilled forth by the shattering of Hera’roilya. The stars were endless possibilities, and each could see more of Nuinen than Faranwe ever could. For that, Faranwe wished to have the sight of the stars. Faranwe was wise in his perspective, thinking more of grand schemes than present troubles. This he would keep to himself, unless others asked of him his thoughts, which were wide and unending, for Faranwe sought knowledge more than anything, and he knew many things.
Apythia, who holds many secrets but seeks many more. Dark of skin and hair, Apythia could see the truth in any situation, but keeps much of it to herself. She trusted Faranwe with much of her knowledge, and holds him close to her as kin. With others, Apythia is less trusting—and she distrusts Mear most of all. While Faranwe would spend much of his time with Mallorn upon the peak of the Astalena, Apythia sought always for the soft mists of the coast. For though she could not wade in the acid water, in the cold mists soothed her, and she felt she could think freely in arms of fog. She was quiet and wise, and gave not freely the secrets she knew. With then she would subtly turn conversations in her favor, for she was crafty. It was in her dreams that Apythia felt most at home, for she alone could control her dreams and use them to show hidden truths in herself: in that way she knew herself best of anyone.
Leithan, who loves to free objects of their given shape and build them anew. Tall and lean, Leithan was pale of skin, hair, and eyes, though oft he was stained and painted with the ash of the forge. Leithan is the most renown craftsman of the Astar, and he alone worked with Liliath to plan Ithalas and construct the Astalena. The ash he mixed with earth, metals, and stones to make the hardest of brick, and it was he that was first among the elves to shape the forever-frozen ice of the north into tools and ornaments. Leithan loved his works, and forever tunneled downward in search of further mysteries under the earth. He was rough of body but sharp of mind. His temper was little, for in his work he accepted no mistakes and would punish those who made them. This made him proud, and this pride in his work and his people knew no bounds: if the Astar were wronged, Leithan would show no mercy on the offender.
Nalas, who forever sorrowed over the pain of the Astar, lived only for the happiness in others. Sister of Karn, Nalas oft wandered the land around Ithalas, but the streets of the city were as a second home for Nalas, who would spend much time with the other Astar. A lover of music, the company of others, and all living things, Nalas is the friend of all but herself. Her hair was soft and dark, and she could often be sought in time of distress, for she was the counsellor of many. In the eyes of all, Nalas was the softest of heart of the Astar, and she couldn’t bear conflict.
Karn, twin brother of Nalas, was rarely found in the civilized lands of Ithalas or its surroundings. The wastes of Amarth saddened Karn, and he wanted nothing more than the soft forests and rolling hills woven in the stories of Liliath. He feared that the ash was killing Nalas, if not in body then in spirit. Karn was relentless in his search through the wastes, searching in vain for a sanctuary for Nalas. In this Karn showed his endless perseverance and stubbornness. Should a challenge present itself to him, Karn would work until the challenge was met.  He rode his great elk Voronwe through the ash storms, and was a friend to all the animals.
The Council of Liliath made discussion around the fate of the Astar. This meeting and those after it the Astar called the Omen Umbar, where long the Council deliberated. Those of the circle were divided in their thoughts—Mallorn spoke most for those who wished to take the Anariima and leave Amarth, heading south for lands yet unknown. Faranwe, Tirwen, Apythia, and Karn he had behind him, but Mear was outraged. Many thousands of years the Astar spend in Amarth, building their great city of Ithalas, growing as a people. The harshness was woven into their beings now, and it was only a matter of time before their progress would lead them to a dominion and mastery of the landscape. Leithan Mear had behind him, who ever wanted to explore the earth beneath Amarth. The ground was hard and frozen, but the tools and materials reaped were uncounted an invaluable. Nalas took no side in the conflict, despite Karn’s pleas. The argument she hated, and she stepped forward to be the main mediator of the Council.
-Hera’roba delivers the threads of Arangil and Taregil to Hera’rocoia, who takes the threads to the Ethereal Realm. There he travels to the realm’s Eye of Ehlu, and upon the very pedestal that held Hera’roilya Hera’rocoia wove together the threads together. That braided chord he formed into the crystal heart of Selor. He entered Andolem and took Selor from the shadowed Naa’yamen, instead placing her upon a bed of leaves in the Shadow Realm’s Eye of Ehlu, for there she would be able to transcend the endless planes. There Hera’rocoia gave to her the heart, but in his mind, he heard the voice of Selor: “This heart is an extension of myself, and shall serve to bear upon the worthy the power of Hera’roilya. I give this to you to hold as its sentinel, for of all I know that you shall not use this power in wroth.” Hera’rocoia takes the Heart of Selor and names it the Parmasinta, the Keeper of Knowledge.
-Reinien, Queen of the Tel’waith dies of age, and the title passes to her daughter Enalië. Reinien lives to see the Taregil and Arangil leave the sky, and in her final moments she is at peace believing her people’s salvation to be soon at hand.
-As the Anariima are plucked from the sky, Linarata and Enalië, for by the Blessing of Linarata can their path be seen. Linarata grows into her years of waning, but in the coming of this long-awaited day she gathers her strength. She and Enalië begin preparations to lead a great procession of the Tel’waith to the north.

54,990 BBT|1 YE—As the Council of Liliath continues the Omen Umbar, the great ice bells of Ithalas ring out into the darkness: a host was on the horizon. Tirwen orders the gates opened, and the procession of Tel’waith enter the city. Tirwen, flanked by Mallorn and Mear each holding an Anariima, welcomes Linarata and Enalië. When Linarata bathes in the glory of Arangil and Taregil she gasps “Anwaië!” (It is true!). The leaders of each party meet, introducing themselves and their purposes. Linarata tells of the Naa’waith peoples, of their heritage and their demise by the hands of the Loss’kelvar and Okemenel. She tells of the Luumedag and the Naara’tela, but not of Hera’roilya or the origin of magic. Of magic she does tell, and she demonstrates lightly the magic she was taught by Eres long ago. When Linarata shows the Council of Liliath magic, all are held in awe. Each in their own mind could not believe what they saw—but even then, they had a desire to wield it. To Linarata and Enalië the Astar showed the endless stars from the top of the Astalena. Enalië was in such joy when she saw the heavens that she cried the first merry tears of the Tel’waith. 
In this time, the most glorious moment of the meeting came: Linarata then showed the Astar the true power of the Anariima. Holding Arangil in her right hand and Taregil in her left, Linarata became the first of the Naa’waith to wield two Anariima at once—and the power it gave her was magnificent. She held Arangil to the moon, and the light of Memaska shone through it, casting ten million rays of blue light upon the clouds below. Linarata channeled forth her power and cleared the smog and dust from Ithalas, and the infinite stars shone down upon the city of ice. The towers, walls, and streets all reflected the glory of Arangil and Memaska, and the combined peoples of the Astar and Tel’waith looked upon the heavens as one. Then Linarata brought Taregil to the stars, and with it she showed those few who stood upon the peak of the Astalena the wonders of timelessness. Linarata created then three artifacts of Taregil: the Traveling Amulet to step into the past, the Reading Glass to discern the secrets of the past, and the Chalice to grant everlasting life. To Mear she gave the Traveling Amulet, to Tirwen the Reading Glass, and to Mallorn the Chalice. She filled the chalice with water and passed it to all. The old Tel’waith was given youth once again, and the Council of Liliath and Enalië were given beauty and endless life. With this the Astar named Linarata Envintule, the Bringer of Life.
	The year was dedicated to festival and jubilee, and it was named the Yenesta. Soon after the clearing of the clouds the Yenesta was marked with the first rising of the sun. The light of Memaska gave way to the majesty of the golden light of the sun, which the elves and Tel’waith named Atarisil. With this naming, the Tel’waith name the elves the Insilaë. During the first night after the rising of Atarisil Enalië sings the Vahalis from the height of the Astalena while holding Arangil. Her voice echoed throughout the city and the land, for Arangil carried it far. Her voice carried with the water and the wind then, and it was heard by all those of weak spirit in the world. The Vahalis was recorded on the walls of the Astalena.
	This ends the Times of Ash and begins the Yen Envinyanta, the Age of Healing in the history of the world.
-Hera’rocoia follows through on his promise, taking a sapling of Selor to Vash Bash’Tikish in the Andolem.

Yen Envinyanta
The fires of the Naara’tela still burn in the skies, but a glint of light shines through. Arangil and Taregil have been recovered and are held by Linarata Envintule and the Astar of Ithalas. The majesty of Ithalas has been shown under the risen sun, and a golden age has dawned on the Astar.

54,985 BBT|5 YE—Linarata Envintule and Enalië and three thousand of the Tel’waith accompany Mallorn, Tirwen, Faranwe, Nalas, Karn, and five-hundred of the Astar on a first expedition. Among the Astar were Astaldan, renown warrior, and Quildanon, a symbol-carver brought to ascribe the journey on ice tablets. Linarata Envintule told the Council of Liliath of the city of Celebtal, the Silver City with beauty surpassing even Ithalas, and the Council decreed that a host search for the city. Leithan crafted ships of glittering ice to carry the travelers into the east, and the ships departed from the harbor of Metimacopa. At the prow of Linarata Envintule’s ship was Arangil, blazing through the smog. The party of explorers is named the Cestani.

54,984 BBT|6 YE—The Cestani landed on Annuntol first, unknowing of Celebtal’s location but for the fact that it was far east of Naa’yamen. The land was dense and wild, the powers of Naa’roleith twisting the colors and plants and mountains. The ash of the Naara’tela stuck to the dewy plants and floated thick in the humid air of the southmost parts of the continent. The constant darkness did little to sooth the nerves of the Cestani adventurers, and progress was slow. After months of trudging exploration, though, Nalas and Karn’s party passed over the mountains into what was once the Grove of Comenraan. The Grove was desecrated and burned, the lake at its center bubbled, now a boiling acid. An expansive pit fell downward from the mountains, an entrance into Comenraan’s endless caves. As the group passed through the grove they were beset upon by the Rhovari hydra Medorohm who had taken refuge in the abandoned caves. The hydra burst from the stone beneath the feet of Karn and Nalas, and its many heads devoured the Tel’waith and Astar that accompanied them. Karn hoisted Nalas upon the back of Voronwe his steed and sent the elk galloping to the mountaintops.
	Karn dodged the many heads of Medorohm, but the swift strikes of the hydra soon overpowered the elf. Medorohm slammed his body into the mountainside and sent a cascade of rocks down upon Karn, trapping him beneath them. The rocks were many and large, and the gaps between them were not unlike hallways in a maze. Karn managed to hide within the maze of rocks, though Medorohm began probing the rubble with many of his smaller heads, none any less ferocious than his seven gaping maws. Karn was weaponless, but he was swift and bright of mind. Long he explored the tunnels of the rocky maze, analyzing its structure until he dislodged a rock and bring down the stones upon Medorohm’s heads. As the Rhovari hydra wailed from its many heads, Karn called to Voronwe, and the elk stamped his hooves down upon the weak rocks of the mountain. The mountain came down upon Medorohm and sent him tumbling once again into the caves of Comenraan.
	Nalas and Karn retreated to their camp and recounted the tale to the others of the Cestani. Linarata Envintule knew that the Grove of Comenraan was far to the west of Celebtal and Rhunendor, so the Cestani departed from that land.

54,983 BBT|7 YE—The ships of the Cestani rounded the south of Annuntol and were caught in a typhoon. The storm was sudden and brutal, and the ships were unsinkable, but they were not immune to overturning. Many of the ships capsized in the harsh waters, and others became lost in the rain and the waves. Among those lost was the ship of Enalië, Queen of the Tel’waith. When the Queen’s ship was lost to the waves, Linarata Envintule was ashamed of her own fault in the protection of her student. Her decision was difficult: she could not go in search of Enalië, for that would doom the Cestani to failure in finding Celebtal—they needed her experience and knowledge of the hidden city. Linarata’s other path would be to abandon her student and queen of the Tel’waith. In that she would be betraying her people. After long thought Linarata Envintule turned away from her student, but not entirely. A ship she sent to search for the lost Ariel, captained by the Tel’waith thinker Raededir and accompanied by the Astar Astaldan. Those ships that made it through the storm continue through the sea that that name Airaum and land upon the large isle off the southern prominence of Rhunendor. The Cestani explore this isle that they name Erincaine for the remainder of the year.
-Enalië washes up upon the shores of a small island to the south of Annuntol. Her crew was killed, and the ship was shattered upon the rocks. The island was rocky and had a fey feel, covered in arches and tunnels. This island Enalië named Lónacuma, and she sat for many days upon the rough coast listening to the waves crash upon the rocks. Enalië was an Ariel, though, and new enough magic to keep her fed and healthy. She wandered the island and found its surface barren but for a small grove of trees. When the storms of the sea buffeted the island Enalië sought refuge in the caves. A powerful storm hit the island in the latter part of the season and the caves began to flood, pushing Enalië further into the island. There she stumbled upon a deep sanctuary. The water there was cold and pure, and even through its mighty depth the glittering crystals at its bottom could be seen. When Enalië stepped carefully along the slick stones a light appeared in front of her—a sapling had begun to sprout, a beautiful blue light emanating from it. The sapling grew quickly in front of her into a tall tree with leaves of luminous hues. Enalië curled herself below the boughs of the tree and slept. When Enalië woke, she heard the voice of the tree over her. The leaves whispered to her and soothed her. This was the first sprouting of a Tree of Selor in the main of the world, planted by Hera’rocoia. Enalië spoke to the Tree and learned from it the lives of Tolas and Selor. Enalië made her home in that cave, and there she learned further the ways of magic that not even Linarata Envintule could know.

54,982 BBT|8 YE—The Cestani land upon the shores of Rhunendor and begin their exploration. Linarata Envintule holds aloft Arangil to banish the ash from their path, and the long journeys through the forests and mountains of Rhunendor begin.
-After a year of sailing and seeking, the ship of Raededir and Astaldan land on Lónacuma. It is not long before they find Enalië and her sanctuary. Enalië tells the two of the power of the Tree and shows them the arts that she had learned. Raededir and Astaldan both speak to the Tree and become enraptured by its knowledge. They linger long on the island, learning all they can from Selor. To Raededir, though, the whispers were corrupting. Through the tree that Hera’rocoia had brought him, Vash Bash’Tikish spoke to Raededir. The Tel’waith was fed lies and tales of the power by Vash Bash’Tikish whom Raededir named Sana. Astaldan learned to wield his sword, carved from the ice of Amarth, with the power of magic. He became the greatest blade-wielder of the Astar in that time, for he combined body and mind in his technique. Enalië became versed in the magic of sight and knowledge and wove it into her songs. Her music echoed through the arches of Lónacuma and over the waters of the sea. Her music held the secrets of the island and her premonitions, though few could see them.

54,981 BBT|9 YE— Landbound, the adventurers of the Cestani searched ever northward into Rhunendor for the hidden city of Celebtal, but the Iantar was thick and enchanted, ever misleading the search. It was Faranwe the Dreamer wandering alone in the forest that first stumbled upon the secret stream Lencanen that led under Celebtal. He followed the water and found himself in a ravine below the city, and upon climbing the many stairs he was welcomed by the Rhun’waith. He returned to the Cestani ahead a party of tall white horses, and brought the search to an end. Iulian the Graceful welcomed the host into the city and celebrated the coming of the Tel’waith and the awakening of the Astar. Introductions were made and Linarata Envintule informed Iulian of the state of Perakor and the islands the Cestani explored. She showed to him Arangil and told him of Taregil in Ithalas. Iulian was enchanted with the Anariima, and as an Ariel himself he reveled in its power. Linarata Envintule was given new life at the discovery of Celebtal and the Rhun’waith survivors. She told Iulian of her plans to bring the Tel’waith of Metimacopa from the wastes of Nevlonde and lead them to Celebtal. Celebtal would be given light once again with Arangil and Taregil shining from its towers, a city unsurpassed in beauty by even the glorious Ithalas under the sun. Iulian the Graceful did not share her dream, though: in his wisdom he saw that the time of the Naa’waith was waning. The Rhun’waith were dwindling, their spirits turned from the world. Iulian the Graceful would be their last Lord. The purpose of the Naa’waith, he explained to Linarata, was to lead the Astar, the legacy born of the body of Tolas. The Anariima must go to them, and the secrets of the Ariel as well. Iulian told Linarata of the words Herya’rosintilya spoke to him before his departure, that there may be a way for the Rhun’waith to pass from this world and live forever among their forefathers. Linarata Envintule was aghast at what Iulian was proposing, and she was enraged. She had saved the people of Saeronder and brought them through hardships unknown to the Rhun’waith. The Tel’waith were survivors—they would not give up so easily. She left the meeting and thought long about Iulian’s words.
-Many years the Cestani spent in Celebtal. The Astar of the journey learned the ways of the Ariel under Iulian the Graceful. The renown of the Council of Liliath took fastest to the ways of magic. Iulian saw that each of the Council brought unique aspects of their personality into their magic, shaping its properties with their wills but still mastering all aspects shown to them.

54,976 BBT|14 YE—Two sons, Hundië and Insúlo, are born to Enalië and Astaldan. They are Gwenyn, or half-ones, as Enalië is Tel’waith and Astaldan is Astar. Enalië’s bearing of Hundië and Insúlo were kept secret from Raededir, as Astaldan had grown suspicious of the Tel’waith’s motive and character. Long had Raededir spent whispering to the Tree, muttering under his breath words that Astaldan only caught fragments of. Whispers of treachery and plot they were, but Astaldan knew not, but he feared for Enalië and their sons. His fears turned true, for only nights after the bearing of Hundië and Insúlo Raededir made forth from his hovel with wicked sword in hand. He burst into the hall of Astaldan and took the warrior by surprise, cutting from him his sword hand and his ear. Enalië distracted Raededir before the traitor could kill Astaldan. She ran through the caverns with Hundië and Insúlo in her arms and Raededir upon her heels. She made for the Tree of Selor, and in a hollow she placed her children. The white bark of the tree closed itself around the Gwenyn and hid them from the horrors beyond, for Raededir then slew Enalië at the foot of the tree, her blood painting its roots. Raededir knew not that Enalië had given birth, though, and thought the children dead. He left Lónacuma with his crew and made for Rhunendor, the whispers of Sana echoing in his head.

54,975 BBT|15 YE—The preacher Raededir comes to Rhunendor in the budding spring, and he wanders long without direction or knowledge of the location of Celebtal. Many things he learned from Sana, but the Silver City had evaded even the cursed spirit during its time.
-After Raededir’s betrayal, Astaldan is left to raise his two sons alone. They grow on Lónacuma, and Astaldan teaches them what he can of sword fighting, and they begin to learn magic from Selor in the tree.

54,974 BBT|16 YE—Raededir was worn and decrepit when hunters out of Celebtal found him, the last surviving of the crew. They brought him with urgency to the city, where he recounted a tale of treachery to Linarata Envintule and Iulian the Graceful. On the murder of Enalië he spoke, but Astaldan he painted as the traitor. He spoke that the Astar had raped and killed her on the isle, and Raededir had barely escaped with his life. Linarata was shaken to her knees upon hearing of Enalië’s death, and Raededir took to her side, consoling her and growing closer to her, the Queen of the Tel’waith. Raededir’s charisma was unmatched, and he quickly wove himself into the lives of the prominent Tel’waith. All this was on the behest of Sana, though, the voice in the Tree of Lónacuma. Sana spoke that should the Tel’waith and the Astar become divided, Raededir would be able to strike them and gain power unlike any other. Raededir became so close with the queen that she shared with him Iulian’s plan to let the Rhun’waith pass from memory and leave the Anariima to the Astar. With that knowledge Raededir sowed his discord: Why should the Astar be given the Anariima, the jewels of the Naa’waith? Was it not Liliath but Tolas and Selor that pulled the stars from the heavens in the ancient times? If the Rhun’waith desire to pass from the world then so be it—but the Tel’waith shall reign yet in Celebtal, not the Astar of the ashen plains. Raededir fed also into Linarata’s love for the Anariima, and the queen became ever more reliant on the jewel’s glow. She would lie with it always, never sleeping but watching and staring into the glow. Slowly, her love for the gem drove her mad and blinded her. Her everlasting youth dimmed not, and her radiant form reflected the light of the gem she held so dear. Raededir saw this corruption and encouraged it—for every day Linarata would see less, hear less, speak less, and every day Raededir could be her eyes, her ears, and her tongue. 

54,946 BBT|44 YE—As Hundië and Insúlo matured and learned the crafts of their parents, Astaldan turned to the sea. He listened to the wind, and upon it he heard the words of Enalië. She whispered to him the location of Celebtal, for in her many years on Lónacuma she learned the craft of divining the future from the Tree of Selor. Astaldan crafted a ship from the trees of the island and left to find Raededir in Celebtal and slay the traitor. As Astaldan passes through the Lencanen, though, and speaks to the guards of Celebtal he is quickly captured and brought before Linarata Envintule and Iulian. Upon the floor of the mighty Temple of Ralun, Astaldan was accused of his crime: the murder of Enalië. Astaldan pleaded his innocence and told his tale of the happenings to Linarata. The queen was conflicted—she was wise, but the words of Raededir in her ears were that of a fellow Tel’waith, not an Astar that Raededir warned of. Iulian offered a solution to Linarata: that they seek the aid of the infinitely wise: Herya’rosintilya who disappeared many years before. Should the hunters of Celebtal find him, he would surely know the innocence of Astaldan. Until the time that Herya’rosintilya could pass judgement, Astaldan would be imprisoned.
	The return of Astaldan did not trouble Raededir, for his plan was to divide the Tel’waith from the others. To divide the Tel’waith from the Rhun’waith would be the most difficult, but the first steps had already been taken. Raededir built and fed the greed of Linarata Envintule for the Anariima and fanned her pride in the survival of the Tel’waith. This pride and desperation would lead Linarata to bring her people from Metimacopa to Celebtal at any cost. Should the Tel’waith be resisted, the sack of Celebtal and the final destruction of the Rhun’waith would be inevitable. To turn the Tel’waith against the Astar would be simpler, for Raededir already had his first piece: Astaldan, come from Lónacuma, murderer of Linarata’s only student, prodigy, legacy, and Queen of the Tel’waith. Raededir would speak out on Astaldan’s crime and grow tensions between the Tel’waith and the Astar. Astaldan was but one of the Astar that were envious of the Tel’waith’s beauty and accomplishments, and their creations: the Anariima. In this movement Raededir would spark a pride in the Tel’waith and paint the Astar as hungry for power. Should Astaldan be found guilty in his trial for the murder of Enalië, Raededir’s plot would be complete. The introduction of Herya’rosintilya was disturbing to Raededir: this figure of unending wisdom would surely see through the lies. Raededir thought long about how to bring about the guilt of Astaldan.
	It was also during this time that Raededir used Linarata to create the staff Maril and the font Indofi, both artifacts imbued with cords of Arangil. Maril was a staff of glittering crystal, unbreakable and beautiful, it created mirror aspects of any magics sent through it. Indofi was a wide basin filled with a pearly water. From the water could be drawn anything a heart desires or a mind could ponder. These things Linarata used in plenty and in pride.

54,943 BBT|47 YE—Raededir’s preaching increases tension between the Tel’waith and Astar to the point of near hostility—the only thing stopping all out violence is a crucial event: the ongoing trial of Astaldan, hero of the Astar. As this happens, the training of the Council of Liliath halts. Iulian the Graceful looks down upon his city in despair and confides little in the Astar. Their knowledge of magic will only about more destruction should the two factions come to blows, and Iulian wants to prevent that at all costs. He tells the Astar this, fearing that they take his refusal to continue lessons as a refusal to share power with the elves. The Council of Liliath then take to the streets with their people and learn more of the troubles of the city.

54,938 BBT|52 YE—Raededir recruits a radical band of Tel’waith to break into Astaldan’s prison and kill him to cause war between the two peoples. The band of Tel’waith moved on a dark night, the fires in the sky dampened by an ashen cloud. They broke the bars of Astaldan’s tower prison and moved to slay him, but they underestimated the warrior. Even with his sword hand severed, Astaldan moved with a speed unmatched. He dodged each of their blows and grabbed one of their swords. In a matter of seconds, he slew each of them. The cold wind blew into his prison, covering the bodies with ash. The doomed warrior climbed from the tower and escaped into the dark night.
The morning came, and though Raededir saw that his plan failed, he did not despair: for the people did not know. He preached the death of Astaldan and called to the success of the radicals and the satiation of the Tel’waith’s justice. His words did not stop at Astaldan: he continued, warning of the dangers of the Astar and their jealousy. Riots erupted in the streets, Astar fighting Tel’waith.
The five of the Council of Liliath met during this time of strife, joined also by Iulian the Graceful. Iulian saw Raededir’s plot as provocation, and warned the Astar that should they fight, they would fall into his hands and find only pain. The Tel’waith were a broken people, living only on the memory of what once was. In this they are misguided and doomed. Should the Astar stay in Celebtal war would surely ignite. Iulian pleaded to the Astar to leave the city and return to the western waters of their homeland. He warned that the Tel’waith would surely follow, though, for Linarata Envintule now desired Taregil and Arangil above all things.
In this matter Mallorn was decided: the Astar were wronged, their hero Astaldan killed in cold blood and treachery. Raededir was turning the Tel’waith against the Astar, and he needed to be ended. The Astar should fight. Should they leave for Ithalas, the violence will only follow them.
Karn saw the riding tensions in Celebtal and knew nothing good could come of violence. Astaldan’s death was tragic and unneeded, but Karn turned to Faranwe’s wisdom in this. Faranwe reminded the five of the goal of the Cestani: to find a place of refuge and new life for the Astar. The majesty of the Tel’waith and their magic caused the Astar to lose sight of this foundation. The five of the Council had learned much in the ways of the Ariel, and they could share their knowledge with the other Astar. Should the city fall into war, though, the knowledge could be lost with the lives of its holders. Faranwe counseled Mallorn to see this wisdom and lead the Astar home to Ithalas. From there, they can seek once again a haven for their people.
Tirwen was torn in her thoughts. She was proud of her people and wanted to fight for this pride, but she knew that war would only lead to sorrow. In this matter, she sided with Faranwe, for the lives of the Astar were the most important. The Rhun’waith were innocent in this fight, and Tirwen did not want to trouble them with the matters of Astar and Tel’waith.
And so, it was decided that the Astar would leave the silver city of Celebtal. Iulian knew that, should the Astar leave for Ithalas, it would leave the Rhun’waith weakened against the power of Linarata Envintule and Raededir. Iulian spoke of this to the Astar of the Council of Liliath, and Tirwen saw that Linarata Envintule would be lost without Arangil in her possession. Should the Astar and Iulian confront her, they may be able to take Arangil and return it to Ithalas.

54,937 BBT|53 YE—The Astar, led by Nalas and Karn, marched from Celebtal. Rain muddied the tracks and greyed the sky, and thunder rocked the towers of Celebtal. Linarata Envintule sat upon the throne of Ralun in his temple, peering into Arangil, Raededir by her side. She smiled as the Astar fled, for Raededir had corrupted her mind towards power alone. Raededir’s plot was building: without the Astar to aid the Rhun’waith, the Tel’waith of the Cestani would be able to lead a coup and take the power of the city. Then Raededir schemed to turn Linarata to Ithalas, to Taregil. Raededir turned in surprise when the great hall of the temple echoed approaching footsteps.
	Iulian the Graceful, Mallorn, Tirwen, and Faranwe approached the throne of Linarata Envintule.
	“That throne is not yours, Linarata. Stand from it, and relinquish your false power,” called Iulian. Mallorn and Tirwen drew their blades, and Faranwe readied his magic.
	Raededir laughed, for the opportunity to end the Astar and Rhun’waith had presented itself. He whispered counsel to Linarata, and she spoke thus:
	“Are you threatening me, Iulian? What is your power compared to mine? What is your rule compared to mine? What are your people compared to mine? Celebtal is wasted upon the Rhun’waith. The people of Metimacopa shall raise it to new heights and grow to rule all! If you desire my power, then step forward those who dare and take it.”
	Mallorn lunged, and a great fight ensued. Linarata Envintule unleashed the power of Arangil, but Tirwen and Iulian worked to douse her. Together they created mighty shields, and Faranwe channeled his magic to give wisdom to Mallorn in his attacks and Tirwen in her defense. Raededir hid away, trusting the power of Linarata to engulf the attackers. The corrupt philosopher underestimated the power of the Astar, though, and Faranwe’s foresight allowed Mallorn to strike where Linarata was weakest. The Astar gained the upper hand, and as Linarata became desperate her power began to engulf the temple. In this moment of greatest power Faranwe saw her in her weakest, but Mallorn was knocked back, Tirwen was pinned, and Iulian was the only one able to prevent the destruction of the temple with his shields. Faranwe moved forward and grabbed Linarata Envintule, wrestling Arangil from her hands. In this moment, she struck him through with a dagger, and he fell. Iulian the Graceful pushed with all his power in Linarata’s moment of distraction and sent her across the hall. Mallorn went to Faranwe and carried his friend from the temple with Arangil in his arms. Mallorn glanced back at Iulian, lord of the Rhun’waith, before leaving from the temple with Tirwen his sister. Iulian understood his duty now was to allow the Astar to escape with the Anariima. He nodded once to Mallorn before turning back to Linarata. The queen was enraged and broken, and her power weakened without Arangil. Even without the gem, though, Linarata Envintule’s power surged. She battled with Iulian long in the temple of Ralun, but it was Raededir from the shadows that came forward and plunged a sword into Iulian’s back. The Rhun’waith died, then, but not in vain. The Astar had left the city, and Arangil with them.
Iulian had left the Rhun’waith to his student Miranion, whom he had taught for many thousands of years. Miranion was born the son of a jewel smith, the finest in the land. Miranion’s father was tasked to making Iulian the Graceful a ring, the finest piece to shine through the ashes. Soon after being tasked to this, Miranion’s father fell ill and died, and his burden was left to his young son. Miranion knew little in jewel craft, and each attempt in crafting a ring for Iulian ended in dismay. Tirelessly, Miranion worked. When the day came for Iulian to be gifted the ring, Miranion arrived at the temple with his creation. The young smith knelt before the lord of the Rhun’waith and placed upon his finger a simple bronze band, roughly hammered and soot stained. “My lord, this is my finest work, humble it may be.” Iulian saw the ring and knew Miranion’s wisdom—for what could a lord do with a ring?
“Humble it is, though it shall be my greatest possession, for in it I can find more than beauty,” responded Iulian. “It seems you have little skill in jewel crafting. It may be that your calling is elsewhere—let me aid you in your search, if you would have me as your teacher.” And so Miranion trained under Iulian and was his finest student. Now, as Iulian bled upon the steps of the Temple of Ralun, Miranion strode at the front of the Rhun’waith guard. The streets of Celebtal broke out in violence, the Tel’waith of the Cestani fighting the guard of the city. Linarata called out to her followers to escape the city and follow the cowardly Astar, for already she felt a painful hunger for Arangil. The Tel’waith fought brutally in their escape, and many were killed by the Rhun’waith soldiers. The Tel’waith fled down the many stairs of the river canyon, and many were slain by arrows upon the slopes. Fewer than five hundred of the Tel’waith survived the canyon pass, but those who did were led by Raededir and Linarata upon the tracks of the Astar. With them were the artifacts Maril and Indofi. This violence, betrayal, and murder comes to be known as Rasestani.
The rain of the storm cleared the streets of the silver city, and on the next day Iulian the Graceful was placed within the tombs of the Rhun’waith, their ancient halls surrounding the very cavern that Hera’roilya awoke in. Iulian, student, was placed next to the empty sarcophagus of Ralun his teacher, and his next to the two dedicated to Tolas and Selor. Miranion slipped from Iulian’s finger the rough bronze ring that he had crafted many years before and wore it proudly on his finger. The ring became known as the Risilus. As Miranion wore it, he felt that Iulian had bestowed upon it powers of his ancient magic, and from it Miranion continued his lessons and learning.

54,936 BBT|54 YE—The Astar made for the south, towards the original landing of the Cestani upon Rhunendor. The journey was cold and perilous, with storms rocking the skies and rain chilling the skin of the Astar. The elves made their encampment upon the southern shores of the land, unearthing the frozen ships of the Cestani from the caves of the cliffs. As they prepared for the journey west, the war cries of the Tel’waith echoed along the cliffs. Linarata Envintule and the surviving Tel’waith bore down upon the Astar and began a massacre that would never be forgotten. Quildanon ascribed this event as the Thoryollo, and the cliffs ever after held that name. The swords of the Tel’waith were fierce and fell, but Tirwen rallied the icy spears of the Astar against the onslaught and stood in formation upon the cliffs as those unable to fight boarded the ships. Linarata’s power was dampened without Arangil, but she was yet powerful. Her influence pushed the Astar back, and the Tel’waith’s eyes shone red with fury and hatred. The Astar were pushed off the precipices, falling upon the sands far below. The macabre scene was light by the fire in the sky and the pale light of Arangil with Mallorn upon the sandy shores below. Nalas looked up as the last of the force of Tirwen was slain, and the shield lady herself was felled and thrown from Thoryollo by Linarata. Nalas found the body of Tirwen and brought it forth upon the ships, and Karn preserved it as he preserved the body of Faranwe. Mallorn pushed the last ship from the shores as the Tel’waith descended the steep cliffs and charged into the waters. The desperate and corrupted Tel’waith drowned in the rough waters, and Linarata screamed curses after the ships as they drifted from the shore.
	In the weeks to follow the Thoryollo, Linarata crafted ships for the remaining one-hundred Tel’waith, and they made pursuit of the Astar.

54,935 BBT|55 YE—…The cold wind blew into Astaldan’s prison, covering the bodies with ash. The doomed warrior climbed from the tower and escaped into the dark night. Astaldan ran from the city, thunder roaring in the skies. He moved throughout the wilderness, avoiding all contact. There was only one that he sought, one who could prove his innocence and bring about the fall of Raededir’s lies…and Astaldan knew them not. Herya’rosintilya he was called, a man of infinite knowledge and wisdom. Herya’rosintilya had disappeared into the mountains millennium ago, and none had seen him since, yet he was the man that would give Astaldan the means of avenging Enalië. Astaldan pulled his cloak tighter about him and pushed forward, for the night was cold.
-Linarata’s hunger knew no bounds and could not be controlled any longer by Raededir. Linarata knew she could not return to Metimacopa without the Anariima in her hands, for if she did, her people would see that her prophesy was false. She landed upon the frozen lands of the Halcinaline, and her one-hundred followers constructed the mighty fortress Tarullume. There Linarata sat in darkness and in cold, looking ever to the western light of Ithalas. Her soldiers grew mighty and hateful, and they trained for tens of years in magic and martial warfare. Raededir sat beside Linarata always during this time, but his council fell on deaf ears and blind eyes. Linarata saw only the Anariima and felt only hatred for the Astar and for the Rhun’waith. In her mind, she began to scheme a means to bring the downfall of both in one vile strike.

54,934 BBT|56 YE—The city of the Astar was silent as the survivors of the Thoryollo marched from the east. Nalas walked at the head of the procession, an ice candle alight with magic in her hand. Behind her were carried the bodies of Tirwen and Faranwe, preserved by Karn. Mear stood to meet Mallorn and show his sorrows. The Council of Liliath met that night upon the heights of the Astalena once again, and Quildanon shared the tragedy of the Cestani. Mallorn knew that Linarata Envintule would not stop her pursuit until the Anariima were hers, and so he spoke of fortifying Ithalas and training the first army of the Astar. Of this the Council agreed: the city would not fall to the Tel’waith. Mallorn held Taregil and Arangil to his chest and felt the old power flow through him. He opened his eyes, and the training of Mear, Apythia, and Leithan began. The three of the Council learned quickly, and most powerful grew Mear. Mear saw magic as the one tool he had waited for, the one power able to control the world around the Astar and command the fires in the sky. His mastery of the art surprised even Mallorn, and the brothers were drawn closer by this power. Mallorn’s mourning for his sister and friend turned to a thirst for revenge as he witnessed Mear’s power. Mallorn spoke long with Mear, discussing methods of manipulating magic to create life—or to reimbue life in the dead. Leithan used his magic to craft the weapons and armor of the Astar from the ever-frozen ice of the north. He recorded his methods and works on his hammer of ice, and when he had finished his final work, he placed his hammer upon a pedestal in his forge in the ice mines below Ithalas.
- Miranion, Lord of the Rhun’waith and King of the Naa’waith after Iulian the Graceful’s death, stood watch over the grey city of Celebtal. He knew what Iulian wanted of him—to follow the destiny of the Naa’waith. They were a legacy to the Astar and races to come after—but how could the Rhun’waith give to the Astar when the Tel’waith sought to destroy them? Miranion held this mindset after the Rasestani and gave the Rhun’waith a new goal: to destroy the Tel’waith before they can destroy the Astar. The Rhun’waith prepared for war—Miranion reopened the Temple of Ralun to teach his people magic, using the Risilus to show them the power of Iulian, student of Ralun.

54,927 BBT|63 YE—Years passed as Astaldan searched, but the Tuluryamar were tall and winding and full of creatures. Astaldan had grown tired and came to rest upon the roots of a mighty tree. As he sat, the fog parted for a moment, and he spied a path through the crags. The path was narrow and grassy, and it wove among the mountains, under arches that reminded Astaldan of Lónacuma. The path climbed upwards, far higher than any mountain rose. Astaldan fell many times on the path upwards, but each time he would rise again and push through his pain. This path he named Oria, and he climbed it long. Upon Oria, Astaldan faced many trials of the mind and body. In each of these tests, Astaldan’s heart shone through in glory, and his climb was not slowed.
-Tarullume’s walls echoed with the sounds of combat and training. Swords clattered as the one-hundred followers of Linarata battled to be her champion. Ten distinguished themselves to her, and those ten she named the Arati of the Tel’waith.

54,925 BBT|65 YE—Astaldan reached the heights of Oria as the mountain peaks below were frozen in winter. At its end, the path opened to a wondrous grove filled with tall frosted pines and soft grasses. There Astaldan slept long, and when he woke he spied by a pond a pearl-white elk with tall antlers of seven-point. The elk turned to Astaldan and spoke: “You walk a road of despair, my one-handed friend. Naught will you gain from this quest, and it shall end you.”
	“This I know, and this I accept. Raededir slew my Enalië, and I will have my revenge. This I swear by the towers of Ithalas and the endless fires above,” cursed Astaldan. The warrior stood and approached the elk. “You are no elk, but Herya’rosintilya the Wise. Raededir is the enemy of the Rhun’waith—your people—and stands to tear apart the bonds that have been built between all peoples. Surely you must act!”
	“The Rhun’waith are not my people. The paths of the Insilaë and the Naa’waith have been further broken and stained in blood since your escape, warrior. Linarata Envintule holds the power to destroy the Insilaë and stands upon her mighty towers to the north awaiting a champion. Iulian the Graceful is dead, his power passed to his pupil Miranion, who now prepares for the final march of the Rhun’waith to war. The Rhun’waith reach into the darkness blind, though, for they know not the location of Ithalas or the Insilaë. Your people, warrior, have felt the sting of Naa’waith swords and ready themselves now to defend Ithalas. The world will burn yet before the skies clear, warrior, and you shall aid in its destruction. Know that.” Herya’rosintilya spoke softly, but his voice was intense and pointed. “You have sought me to know a means to this end. I tell you this: follow Miranion, join him in his fate. His path shall lead you to Raededir.”
	“Then it is to war,” whispered Astaldan as he turned. The warrior descended Oria and made for Celebtal.
-After years of study with Raededir her thinker, Linarata Envintule found within herself a twisted magic, a dark magic. Shadows she learned to shape, bending them around beings and disguising them under false guise.  Across the walls of Tarullume, Linarata ascribed her fell art for her students to learn. Raededir smiled as he watched her spin the shadowy pillars of her power, for he knew his fell plot was nearing its end. He had, through his research, learned to open a well to the Andolem within Linarata. From this well she drew the shadowy power of that realm and projected it. Raededir knew that, in time, this power would consume her—and the well to Andolem would open to a gate. In front of him was Linarata, the Queen of Kings, and a seed for Sana to return to the world.

54,924 BBT|66 YE—Astaldan returns to the city in which he was imprisoned. Celebtal felt different—no longer was it a shining city of hope, but instead a cloud hung low over the towers. The warrior Astar climbed the stairs of the Temple of Ralun hooded and cloaked, and as he entered the great hall of Miranion, Lord of the Rhun’waith, he threw back his hood.
“I shall bring you to the Nevlonde. To Ithalas. To Metimacopa. To victory.”

54,909 BBT|81 YE—The gates of Celebtal opened, and the mighty host of the Rhun’waith marched forth into the fires of the night. Miranion rode at their head, a tall helm of black upon his brow and the thin band Risilus around his finger. With him rode Astaldan, the one-handed warrior of the Astar. The Rhun’waith marched west, bringing with them their fleet upon sleds. When they reached the waters of the Airaum, their path was clear. The ships sailed for the Nevlonde.

54,908 BBT|82 YE—Miranion’s fleet landed upon the shores of Amarth beneath the crest of the Melehtor peaks. There Miranion’s men began the construction of Hravumbas, the greatest fortress of the age. Miranion and Astaldan rode to Ithalas, and when the gates opened Mallorn was shocked into disbelief—he met with Astaldan immediately, asking “My friend you have returned form the land of our forefathers—how is this possible?”
	Astaldan smirked as he replied, “I have yet to meet my ancestors in their land, my lord. I was never lost, merely gone.” With this Mallorn spoke thusly:
	“I am glad that you felt not the pain of death, though I am left further in search of a means of returning the lost to this world.” Mallorn and Mear spoke longer with Astaldan and Miranion of their research to breathe life once again into Tirwen and Faranwe. Once their tale was recounted, Astaldan thought on the problem. He returned to the brothers in a week’s time. “Selor would know…yes, there may be a means by which you can speak to her. Sail with me to the island Lónacuma, and I shall show you.”
	Soon after, Mallorn, Mear, and Astaldan left for Lónacuma while Miranion watched over the construction of Hravumbas and Karn and Nalas continued the training of the Astali.

54,907 BBT|83 YE—Mallorn, Mear, and Astaldan reached Lónacuma, and Astaldan led Mallorn and Mear to the Tree of Selor. The two brothers knelt before the Tree and speak to it. Mallorn called to Selor for the knowledge to give life, to manipulate magic to heal. She told the brothers the only truth of magic:
“Magic is manipulation of power, power is drawn from the ambient universe, which is all. See the truth.”
Selor put the brothers under a spell, taking their consciousnesses throughout the planes of existence, showing them the manifestations of power and magic. This journey showed them all aspects of magic as Tolas and Selor saw them, and in learning the origins of the power the Ariel wielded, Mallorn and Mear could better control it themselves.

54,904 BBT|86 YE—The gwenyn of Lónacuma lived alone after the departure of Astaldan. He spoke to them before he left, and he did not promise his return: instead, he spoke of life, and how his was now focused. He would not return to Lónacuma, for his quest would surely end in his death. To Hundië and Insúlo he gave his wisdom, and they learned from it. The day came that Insúlo heard the words of his mother upon the wind. Enalië whispered to Insúlo his destiny and the destiny of Hundië: the gwenyn would lead both their peoples and end the Naara’tela. They must head the construction of a great pyramid, a Conduit of the Rhun’waith power. The reason behind this is unclear to the gwenyn, but they obey.
	And so, the gwenyn started upon their quest, crafting a slim ship from the tough trees of the island. They left for Rhunendor to find their father and the fabled silver city he spoke of. The ship of the brothers broke the waves of the warm waters and landed on the southern shore of the continent at the cliffs Thoryollo. As they walked the warm sands, the atrocities of the event met them: skeletons still wearing the armor of the Tel’waith and Astar, hands that still grasped the spears of war. The brothers knew no words to describe the scene, and it weighed on their minds as they moved north, following the tracks of the war parties. The paths brought them to the river Lencanen, and they climbed the solemn stairs to the city that they sought.
Hundië and Insúlo tell the tale of their life and the death of Enalië. The Rhun’waith are shocked and angered as the treachery of Raededir is unveiled. The cause of Miranion is strengthened—and the Rhun’waith of Celebtal come together and agree: they must do whatever they can to aid in the survival of the Astar. They were not a people of battle, yet they were still Ariel of Rhunendor, the most renown of the mages of old. Hundië and Insúlo knew that, should the Rhun’waith come together in their power, anything would be possible. They began crafting plans to build a conduit of the Rhun’waith’s power. Convinced by the brothers’ innate charisma and influence, the Rhun’waith follow them as leaders.
-Mallorn, Mear, and Astaldan return to Ithalas. Mallorn and Mear burry themselves in their research, attempting to uncover the locked secrets of magic. Astaldan takes his seat at the head of the Astar army. He oversees their martial training.

54,902 BBT|88 YE—Miranion, The Lord of the Rhun’waith, stood at the base of the Astalena, training the Astar of Ithalas the ways of magic with the Risilus around his finger. Miranion worked alongside those remaining of the Council of Liliath to train the armies of Astar for war, as each knew that the Astar would confront Linarata’s Tel’waith.
-Linarata and Raededir-Linarata sent the three most trusted of her Atari—chief among them Dravion—and Raededir to Metimacopa to raise the people in an army against the Astar and Rhun’waith. Her assault on Amarth would be two pronged—her army from Metimacopa would move north and lay siege to Hravumbas as Linarata and her one-hundred followers would land on the eastern shores and cross the Melehtor peaks in secret. Under Linarata’s shadows, they would enter Ithalas itself and throw down the city. If Linarata could seize the Anariima, the city would fall.


54,923 BBT|109 YE—Dravion and Raededir marched from the gates of Metimacopa ahead the army of the Tel’waith. Raededir had preached to the people of the city, telling them of the traitorous deeds of the Astar and Rhun’waith. The Astar had attempted to slay Linarata their queen and steal the Anariima from her—while she lived, and lives now, the Anariima were taken. The Tel’waith must fight for the jewels that their people forged!

54,922 BBT|110 YE—Dravion lays siege to Hravumbas, drawing Miranion to the frozen fortress with a large garrison of Astar from Ithalas. The siege acts as a distraction for Linarata to sneak over the Melehtor under a guise of shadow. Her force entered the city of Ithalas, and she threw down the disguise. Her one-hundred soldiers rampaged through the streets, slaying all that they saw. The guards of Ithalas battled them valiantly, but none saw Linarata as she entered the Astalena. As she battled below, Karn rushed to Mallorn’s study where he found the son of Liliath in a mess of tablets and drawings. Mallorn spoke,
“She’s here, I know…I need time, Karn. I need time. Slow her ascent to the heights, for I figured it out, Karn. I’ve finished my research.” With that, Mallorn packed the carved tablets into a knapsack and handed it to Karn. “Keep these with your life, my friend. I’ll see you in time.” He ran up the stairs of the Astalena, disappearing around a bend.
Karn took Mallorn’s tablets and sought the others of the Council of Liliath. He informed them of Mallorn’s request, and they gathered on the center platform of the Astalena, awaiting their fate. As they stood to face Linarata, Karn took Mallorn’s tablets and flew from Ithalas as Mallorn requested. Karn escaped the city by ship. Linarata Envintule’s steps echoed through the tower as she climbed, and when she and three of her Atari met the Astar on the platform, she smiled. Silently, she cast her shadows around the room and battled the five Astar. The fight was swift and ruthless, as powerful magic was thrown in the circle from all combatants. The Atari fell one by one, but the Astar were weakened and brought to kneel as well. Linarata struck down Nalas, Apythia and Leithan fell to Atari. Mear was left among them, and his power was great. His anger and desperation grew greater as his friends fell around him, and he found his height when he alone stood against Linarata and one of her Atari. Mear’s stand was glorious, and he slew the final Atari before being felled by Linarata Envintule.
Linarata entered the cold night at the height of the Astalena. The falling ash was below her now, the clouds and storms surrounded the tower but only the stars above offered light to the scene. Mallorn knelt at the center of the tower, an intricate spell circle carved around him. In his left hand shone Taregil, and in his right Arangil. He looked up, and Linarata saw in his eyes a power that rivaled her own—and a wisdom that far surpassed hers. Linarata Envintule was afraid. She stood, frozen—then she felt a pulse within her. Linarata’s connection to the Andolem grew wide and hungry—opened from within the dark realm. The power of Onsinta poured through Linarata’s chest, emanating from her heart. The power overwhelmed her—and she reveled in it. She brought her hands to the sky and called forth a rain of fire upon Ithalas.
Below, Astaldan rallied the Astar. Great streaks of fire shattered the city of the elves, and in the chaos Astaldan shone. He led the elves from the destruction to a hill outside the city. The ancient city of the elves crumbled in the flames falling from the sky like stars to the earth.
Mallorn stood, and moved toward her. With the power of Taregil and Arangil, Mallorn’s power was unbounded, and he faced Linarata down. Linarata Envintule moved quickly—none of Mallorn’s attacks could hit her. The immortal Tel’waith pushed Mallorn back until he could back no longer, pressed against the parapets of the tower. Astaldan looked up at the lights of the fight breaking through the clouds like lightning. He felt the powers of Mallorn and Linarata, and he knew the aura of Linarata’s new power. It was the power that imbued Raededir as he slew Enalië. A whisper came over the wind to Astaldan, a whisper from Enalië. Astaldan listened, and then named the power that he felt:
“I know you. You are false, and weak. Sana. Bash’Tikish, the Damned.”
Linarata’s power faltered as Vash Bash’Tikish was distracted by Astaldan below. In that moment Mallorn pressed his hand against Linarata’s forehead and activated the carved circle below him. In his research, Mallorn understood that magic is not only all things, but also the connection and flow between them. Each that uses magic draws their own energy or energy from some other source to their purpose, manipulating it and shaping it to their needs. Mallorn learned first to draw on the power of life, then to shape the energies that others drew from. In this way, the Astar perfected a method to halt the flow of energy and magic. He blocked Linarata’s draw from the Andolem, and halted her magic entirely. Linarata was an Ariel no longer—but her final act was not yet finished. A terrible star fell to the Astalena, greater than any that had fallen on the city. Mallorn saw it and saw the doom of the city of the Astar. He left Linarata Envintule slumped against the wall of the tower and moved to its center. There, he invoked a magical art not seen before: Arangil and Taregil glowed to an unfaltering brightness that shone even through the thickest of the ash clouds below. Mallorn vanished in the moment that the fire hit the Astalena, and the tower was destroyed. The rubble and fire and smoke rained down on the city and filled the surrounding hills. Astaldan and the surviving Astar looked on at the destruction and saw that the city was truly gone. Mallorn was dead, as was Linarata Envintule, queen of the Tel’waith.
Hate in their eyes and sadness in their hearts, Astaldan led the Astar on a final march to Hravumbas. There, the Astar flowed from the valleys of the Melehtor and fought the Tel’waith under Dravion. Miranion opened the gates of Hravumbas, and his Rhun’waith charged into the field. Raededir and Dravion called a retreat against the glorious charge, and the Tel’waith escaped to the south, persued all the way by the vengeful enemies.
-Karn flew from Ithalas upon the back of Voronwe to a harbor. Suddenly, the world was enveloped in a brightness likening to ten-thousand suns, and a cacophonous boom echoed as the Astalena was destroyed. Something appeared in Karn’s hand—and when he opened it, the dull light of Taregil shone in his palm. Mallorn had given him an Anariima, and Karn knew where it should be taken: to Celebtal. 

54,921 BBT|111 YE—The Tel’waith were caught just as they entered the city of Metimacopa. The streets were filled with slaughter and mayhem—the Tel’waith forces had no chance against the combined strength of the Rhun’waith and the vengeful Astar. The battle was a bloodbath on both sides, but the Tel’waith under Dravion mounted their fleet in the harbor and flew from Metimacopa. Astaldan followed Raededir midst the chaos to the great pier that overlooked the swirling dark waters below. Metimacopa was once a haven for the Tel’waith, now it seemed their grave. Astaldan cut his sword across Raededir’s back, knocking the preacher to the ground. Astaldan leaned over him,
“You destroyed my city.”
He ran his sword through Raededir’s arm just as the Tel’waith rolled, revealing a hidden dagger. Raededir stabbed the dagger into Astaldan’s chest. The old Astar warrior pulled his blade from Raededir’s arm.
“You slew my people.”
He ran his sword into the Tel’waith’s stomach,
“You killed my Enalië.”
He pulled back his sword, and thrust it into Raededir’s ribs once more.
“Die.”
His sword twisted in Raededir’s heart.
“Die,”
He jerked his sword back, and stabbed it for a final time into Raededir’s chest.
“and may your last vision be of the Astar that bested you.”
Astaldan pushed Raededir off the stone pier into the waters far below. The elf warrior sighed, and fell to his knees. He pulled the dagger from his stomach and dropped it by his side. Miranion came then to the pier and saw his friend. Miranion took Astaldan’s hand in his, and pressed the asta’s wound. Astaldan died, then, in the burning city of Metimacopa. Dravion and the Tel’waith sailed east with one goal in their eye—the Gift of Linarata still corrupted their minds, and they saw Taregil in Rhunendor. It was for Celebtal that Dravion left.
-Karn crosses the plains of Rhunendor to find the city of Celebtal encased in a pyramid of stone. The structure was unlike anything that the Rhun’waith had ever seen—its pyramid base spanned a city and its height rose well above the mountain peaks. This was the Conduit of Hundië and Insúlo, and the two gwenyn spied from the peak of the Conduit the lone rider approaching. As Karn rode Voronwe towards the Conduit, he held high Taregil, and the light of the elder stars shone across the grasses of Rhunendor once again. The gate of the Conduit opened, and the gwenyn welcomed Karn. The Astar was worn and tired from his swift journey, and Voronwe as well. The Conduit’s inside was shorn crystal and silver, and reflected wonderfully the light of the city. A new light reflected as well: a soft blue. Karn’s eyes widened in wonder as the gwenyn led him up the stairs of the temple of Ralun. The temple’s majesty had been destroyed by Linarata in the Rasestani, but in its open dais was planted new life—Hundië and Insúlo had brought a cutting of Selor with them, and it had grown in Celebtal tall and glorious. The Rhun’waith were solemn, but Karn could see a determination in their eyes. They were led by the gwenyn to a doom they knew not, but Selor assured them that it was just.
	With Taregil came a new thought: Insúlo isolated himself and meditated under Selor. After four days’ time, he stood, and sought Hundië and Karn.
“I know the end of this Conduit.”
	He explained his thoughts, but knew that the magic he proposed was unheard of—the manipulation not of reality directly, but by the movement of resonances in Corilya. The gwenyn began their meditation and research on the idea while Karn took leadership over the Rhun’waith.

54,920 BBT|112 YE—Dravion led the last of the Tel’waith in a march across the plains of Rhunendor, the Conduit on the horizon. The fires of the Naara’tela still writhed in the sky, and the sight of Taregil still burned in the eyes of the Tel’waith. Dravion was flanked by Caralach and Gweria, and the three Atari knew only hunger. The army began a siege of the Conduit’s gate, the thick stone hammered by the Atari’s magic—for Dravion, Caralach, and Gweria were the strongest of the Atari, and each had opened an Andolem Well within themselves alongside Linarata. As the siege unfolded, the grasses bent under the horns of Miranion, come from across the seas. The Rhun’waith and Astar together met the Tel’waith on the field of battle once again under the fires of the sky. Miranion held the Risilus aloft, and called to the soldiers beneath him.
	“On this day, we fight! No longer are they our sisters or brothers, for they have given themselves to the shadows! Astali, fight for the light of Ithalas lost! Rhun’waith, fight for the Insilaë beside you! Astali, Rhun’waith…My people, fight for a day yet to come, a day midst a time of light, when the fires above no longer burn, when from the ash springs leaf and grass, a time of the Insilaë! May the light of Tolas shine yet from our eyes, for on this day we fight yet!”
	And the plains thundered under the hooves of the horses of the Rhun’waith and the feet of the Astar. The Tel’waith turned, and they cowered not from the charge, for the Andolem Wells within the Atari opened then, and unleashed the power of Onsinta. The darkness of the Andolem devoured the Atari, and their bodies turned to shadow. Their eyes shone sickeningly with the light of Onsinta, and each smiled as the Tel’waith braced their spears against the charge of the Rhun’waith. As their focus was turned, the gates of the Conduit opened, and from it washed the light of Taregil and Selor. Karn alone atop Voronwe charged the backs of the Tel’waith, and he called out the clearest that day.
	In the thick of the battle, Hundië and Insúlo climbed to the peak of the Conduit. They looked down on the glorious scene, and Miranion looked up and met their gaze. Miranion saw the coming doom in their eyes, and knew that all would sacrifice themselves for the coming fate. Miranion took from his finger the Risilus, and tied it to an arrow. He fired the arrow from his mighty bow, and it soared above the field, sticking itself within the stones of the city within the Conduit. With that, Miranion looked at the foes in front of him. The monster Dravion swung a vile mace towards the Rhun’waith king, and Miranion dodged, moving forward and cutting the calves of the shadowy beast. Miranion battled Dravion on the ashen fields, but saw not the archers of the Tel’waith behind him. Miranion took four arrows as he fought Dravion, but with each hit the king stood once more. Miranion severed the mace-hand of Dravion, and spilled the shadowy-light of Onsinta from his bowels, but then he was caught by the claws of the Atari. Dravion lifted him high and crushed the king, but then the clawed hand of Dravion was severed by Karn, who rode up on Voronwe. Dravion was pushed back, and Karn dismounted and lay with Miranion upon the grass.
In that moment, Hundië and Insúlo set Taregil in place on the Conduit’s spire, and the light cascaded down and blinded the Tel’waith. Each gwenyn looked at the eyes of the other, and then activated the monument. Taregil slowed time, it seemed, as it absorbed the power of the Rhun’waith within its walls. The power melted from the walls of the Conduit in beads of light, then it seemed that all things stopped—the ash ceased to move in the sky, the wind ceased to move across the grass. The beads of light moved and broke from the stone of the Conduit, spinning around the pyramid, circling upwards like a river of light around the structure. The beads of power met at the peak, then were unleashed in a mighty beam of energy into the skies. The gwenyn called into the air:
“Ea’imnë hi a’tanaránë! Tyel’ulcocca narë ar lavme rainë hela lavme effírië!”
(Here I show the fault of our people fulfilled! End your fell flames and grant us peace or grant us death!”)
	The beam of power disappeared into the heights of the world, then burst forth in a spectacle of light, a dome of power encircling the sky. The light burned through the clouds of ash and overpowered the fires of the Naara’tela—then the gwenyn enacted their power. Grasping Taregil, they reached into Corilya, and used the power of the Rhun’waith to shape the resonance of the Naara’tela. They brought the fires and clouds to Rhunendor, a circle maelstrom of hell. The gwenyn knew that the resonance of the Naara’tela could be trapped, but only within a stronger resonance—Nuinen itself. The maelstrom fell, concentrated, upon the Conduit. The catastrophe to follow brought the world to its knees, shattering the surface of Rhunendor in its entirety. The Conduit was protected by the power of the gwenyn, but it was forced swiftly into the broken ground. The Naara’tela collided with Rhunendor like a meteor, pushing up the plates of the world like a bowl around it. The rim of the bowl pushed outward, the peaks were miles high as they rolled outward. The peaks came to a halt only after they had run hundreds of miles in each direction, leaving only devastation in their wake. The resonance of the Naara’tela was trapped within the crust of the world as a pale crystal, expanding as a network of lightning into the stone. The stone of the Conduit was melted, but the city of the Rhun’waith was preserved in a cage of crystal forever more. The armies in the fields were gone with the fields themselves. As the dust settled on the world and the stones ceased their shuttering, the gwenyn climbed from the smoldering peak of crystal, now in a crater where it once looked out over the mountains. The sky that met them was clear, and the stars gleamed brightly. The moon Memaska peered down, its light soft and clear. The Naara’tela was over, and the world was reborn. The Rhun’waith within the Conduit lived still, but their magic was gone, siphoned by Taregil which now saw the light of the moon once again.
	Insúlo laughed then, and smiled, and cried for the sacrifices of all. The pain and loss of the Tel’waith, Rhun’waith, and Astar showed true its flowers, then, as the world anew. The rebirth he named the Atanosta. In that moment did the world see again a familiar face: Mallorn stepped into the world again, upon a mountaintop far-away. Behind him walked a host of Astali, at their head Mear, Tirwen, Faranwe, Apythia. Mallorn had tethered himself to Nuinen before his death, and trailed behind him Arangil as a string as he journeyed to the paradise of his people. He came there and saw again the Astar who passed: Liliath, his mother, Mear and Tirwen his kin, Nalas with her brother Karn again, Astaldan again with Enalië, and all those who had fought against the Tel’waith and died. He offered to them a second life to grace Nuinen, a second chance to live. Those that followed walked with Mallorn in a great pilgrimage back to the world, tracing all the way the string of Arangil. As Mallorn brought them back to Nuinen, though, he smiled sadly, and Mear knew his melancholy: Mallorn could not return with them, for he had bound himself to Arangil in the void. Mallorn spoke to those behind him:
“Here I cannot stay. I am bound to the void, but there I have found purpose. With Arangil I will weave paths between the existences: roads, bridges, seas, and rivers of light shall connect all things, and I shall be but a shepherd to all who are lost. Vanemer, I will be called, and I will see you all again, in time.”
And so, the Astar stood on the unfamiliar mountaintop and looked out upon the world with new eyes.

54,919 -54,879 BBT|113-153 YE—The Astar awoke in a high and dark mountain range, the sky open above them. A clear spring lake was set in the peaks, and around it grew luscious trees and grasses. The moon rose silently over the garden, and the water shimmered under its light. The Astar named the moon Memas, after Memaska, and the lake Dinaelin. The paradise in which the Astar awoke they named Tellin. In surveying the oasis, the peaks rose sharp to the north and south, and a heavy fog sat over the western cliffs. From Dinaelin issued the river Beluin, flowing east across the endless desert Palanim. Beluin snaked through the hills and dunes of the white sands, carrying with it the green of Tellin. This new land the Astar named Nomenes, and themselves the Astar Nomeni. Upon the hills in Tellin the Astar Nomeni created their city, Anatelea. The stone towards the issue of Beluin was light sandstone, this they used for their houses and temples. For many years the Astar Nomeni simply rebuilt their lives, starting anew in Nomenes. During this time, the beliefs of the Astar Nomeni began to solidify.
	During the early times, the Astar had only the cold, ashen ground. They worked to survive. They built up a belief of The People, referred to as Ilie. They now do not pray to any deity, for they believe that their loyalty to their people and their daily work to the strengthening of the group are devotion enough to the Astar. Each Astar knows they are a part of a larger whole—the sacrifices they make are for the bettering of all; Ilie always comes first. One among them they held above the others: Liliath, the first leader of the Astar, the one that brought them from the dust to the heights of the sky. She they know as the Mother of the Astar, Amilie. They turn to Amilie in times of sorrow, when their devotion or strength wavers. Amilie turned not from the fires of the sky, but instead rose to meet them in a battle of ten-thousand years. The Astar’s symbol for her is a seven-pointed star, for she brought low the stars unto Nuinen.
In the time before Ithalas, many of the original Astar died in the wastes of Amarth. These Astali were viewed as heroes—they gave their lives for Ilie, and it became thought that even after death the Astali work to the betterment of the others. In this way, the Astar built up strong ties to their ancestors, for they believe that their forefathers look over them always. After the Atanosta, the rise of Vanemer gave the Astar a direct connection to Ilie, and so oft they will look to him as a messenger to the dead. As the Astar live for the strength of others, it is times of lonesome that bear upon them hardest, when they have none to lean on and none to aid them. It is then too that they can turn to Vanemer, for even in the darkest of times, the Astar passed remain with them, and Vanemer is the bridge. While Vanemer exists as a figure in the pantheon of the Astar Nomeni, those Astar that were not reborn in the Atanosta do not hold Vanemer among the holy figures.

54,919 -53,919 BBT|113-1113 YE—After the Atanosta, the Rhun’waith were in awe of the new world—the sky, the stars and galaxies. These things fascinated them and instilled a beauty and hope that the legacy of Tolas—the Astar—were yet alive, somewhere. In the first decades after the destruction of Conduit, the Rhun’waith lived yet in the crystal city of Celebtal. Here Hundië and Insúlo kept the artifacts of old: the Risilus, the Tablets of Mallorn, and Taregil. It quickly became known that in the trapping of the Naara’tela within the crystal of the world, the Rhun’waith lost their magic. Hundië and Insúlo alone retained their magic, and in the time after, as the Rhun’waith lived in Celebtal yet, the two gwenyn developed the form of magic used upon the Conduit: Corilyan Shaping.
	Hundië and Insúlo build from the ashes of the Atanosta a new religious based society. The world was new to them, the sky unknown. As the gwenyn learned more of flow magic and Corilyan Shaping, they saw the intersections of soul and body and world. Thus, the beliefs of the Rhun’waith developed around harmony with the world, and with the souls of nature and peoples. The desert was not devoid of life, as the Rhun’waith first thought as they rose from the crater. Insúlo saw the resonant traces of life in all—the plants, cactuses, insects, and creatures that lived still in the land. The desert gave the people of Celebtal proof of life’s ever presence. The Rhun’waith considered the desert around Celebtal a holy-land, and named it the Elerume. The society lives on the rough and few animals and crops that can be found in the wastes.
Some people are brought before Hundië and Insúlo to learn Corilyan Shaping. These peoples are taken as monks to understand the nature of life as they study under the gwenyn and from Mallorn’s tablets. Their teachings covered the flow of physical energy and its covenant with life as described by Mallorn; the resonations of all things in Tyeluum, and the intersection of these magics: the understanding of the resonant echoes of life upon Corilya. This last artifact of knowledge is the most obscure and the most powerful, for should one understand how to derive a living thing’s physical energy resonant effects on its soul, then the chaos of its soul can be nullified. Thus, a living soul can be manipulated as easily as an inanimate object. Only the most practiced and masterful of Corilyan Shapers could manage this feat; however, many shapers can derive this ability for their own soul after years of practice. This combination of Corilyan Shaping and Mallorn’s Flow magic is a subfield known as Corilyan Flow. These people that learn from Hundië and Insúlo are named Istamb and use their secrets only for the gain of knowledge. All knowledge that the Istamb uncover is stored within light-like resonations accessible only to Corilyan Shapers, thus, the Istamb become advisors, shamans, and leaders throughout society.
As the language of the Rhun’waith evolves, they begin to call themselves only the “Rhun”. The Rhun beliefs begin to develop around the sky. Never having seen the stars before, and thus they were enraptured.
Celebtal is known for its Tuvem, those of the Istamb dedicated to the expansion of knowledge. They journeyed far from Celebtal and gathered knowledge of the world.

	As the Rhun settle in Celebtal, they grow and develop. The crystal city cannot support the growth, and Hundië and Insúlo give leave for their Istamb to lead many of the Rhun on journeys to find new lives. The first of these societies to be born were the tribal nomad society, active in the grasslands and forests north of the Rambel mountains.
These Rhun were led by the Istamb Nimben and Achar. Nimben and Achar were each of like mind: they felt that Corilyan Flow should be used for more than just the gaining of knowledge, as Hundië and Insúlo use it. Through his studies of life, Achar mastered the art of soul-throwing. This technique involves powerful Corilyan Flow, requiring the user to have an innate knowledge of their own soul to shape or “throw” its resonance into another form—a living form is most accepting of a soul, with nonliving and incorporeal forms being the most difficult. Soul-throwing can also be used to replicate one’s own resonance within an object or living thing already present on the prime plane. This form of soul-throwing, or skin-dancing, as it is more commonly called, is much more difficult, dangerous, and taboo. To send one’s soul into another object already formed forces the thrower’s soul to push out or overpower the soul of the object or person they are throwing into, which can often lead to the permanent destruction of one or both souls. Achar taught his followers the specifics of soul-throwing, and they oft took the form of creatures or plants to ambush their prey. Nimben knew best the souls of the natural world. She could shape and turn the wind, knew the souls of fire and lightning, the names of the clouds and the whispers of the grass. Nimben could call the wind to carry her horse faster than any other and knew the memory of the earth. To her shamans she taught these secrets, and they were one with the hills and grasslands and coasts. Nimben could feel the echoes of the souls of stars, left from ancient times within the crust of the earth. Thus, she knew even the stars of the sky, and for her they brightened and danced.
From these two tribes of Achar and Nimben branch others, but each held to the ideals of the Atar, Achar’s tribe, and the Amil, Nimben’s tribe. A sub-culture of skin dancers emerges. They are alienated by the tribes but hold themselves firm in the standings as an official tribe.
The next Rhun society to develop was that of the totemic monks of the Tuluryamar. Hithu, most gifted in the animist arts of Corilyan Flow under Hundië and Insúlo, led a sect of the Rhun to the Tuluryamar area. Hithu could give life to the lifeless, motion to the motionless, thought to the thoughtless: these gifts he gave to the world and his people. He founded monasteries in the mountains, hills, and forests, and each of his monasteries he bestowed with a purpose and a doctrine. For these monasteries he held Taregil and gave souls to the trees upon the hills, the water of the rivers, the wind of the valleys, and the mountains themselves. Hithu was gifted Taregil by Hundië and Insúlo to begin his monasteries, as Hithu led the largest group of Istamb from Celebtal in the days after the Atanosta.
Each of these monasteries was named a Coave, a house of spirits dedicated to the understanding of the world and the principles of the Rhun. Among the Coave was that of Strength, headed by Sarniell; that of Stone, headed by Mirel; that Wind, headed by Lend; and that of the Trees, headed by Orthenon. 

54,919 -54,182 BBT|113-850 YE—There was a time after that the Astar of Ithalas unable to join Miranion to Celebtal remained in Metimacopa after the destruction of the city. Long they waited after the Atanosta and the end of the Naara’tela, but vainly. The hope for the return of Miranion would not avail them, as they knew not of his demise. These Astar grew disconnected and discontented and took the name of Vanar. In time the Vanar split, for some still clung to the hope that, in time, their loved ones would return from the east. Those that threw from their minds the warriors of Miranion were led by Rawel. 
Rawel brought her people away from the eastern shores that gave them only bitterness, to the far west by means of the northern bridges. This western land lay untouched by the Astar entirely, for the only eyes to see the mountains and trees of the west were the Naa’waith before. This land the Vanar named Romendor, and themselves they named the Vanico, “Those forgotten, but living”. The Vanico created the city of Oronlan upon the cliffsides of the northern half of Romendor, a land by which the Vanico surveyor Arasben completed his great works. Arasben explored the lands of northern Romendor, which he named Telmello. Much of Telmello is a cold land, though it bleeds into rocky hills and temperature forests as the prevailing westerly winds sweep the south. Telmello is marked by massive mountain ranges. These streaked mountains Arasben named the Menirea. The rocky and harsh bluff lands south of the Menirea he named the Mavambo, the “Shepherd’s Hills,” as the long grass attracted sheep and herd animals. The forests he spent long in—for the hopelessly wide forest hid many secrets. The forest spanned, unbroken, from the hills to the coast, and much was obscured by fog or dark canopy. This land Arasben named Ezelan. Upon the southern brim of the Ezelan did Arasben end Telmello and name a new land: Sundo, which is the continent south of Telmello. It was on the cold slopes of Menirea that Oronlan was carved, and long it took—though the Vanico were not afraid of long work.
The Vanico held firmly to the traditions of the Astar—they were Ilie, and it was in Ilie that they believed and worked. None were above Ilie, as the other Astar developed. The Vanico knew Ilie to not be perfect, though—and soon the term applied less to mean “the people” and more “the life.” Rawel saw her people as flawed, for Ilie had been betrayed by itself.  Each strove towards a loyal and honorable lifestyle, which cemented the traditions of old and began to build a new hierarchy of the civilization.
Those of the Vanar that remained in the Nevlonde at first were led by Idhel. Idhel lingered longer in Metimacopa, but the memories of the battles and the lost drove her to guide the Vanar south, to the hot desert coasts upon the tip of the Nevlonde. These lands were dry but for the river valleys were these Vanar, named then the Esthela, made their settlements. The coasts were ripe with fish and sun, and the Esthela were soon tanned from their work on the waters and dunes. This land they named Fanalan, and it was their home. Upon the coast Idhel gathered many of the Esthela and build the city Layen.
Idhel bestowed the Esthela with the belief that the followers of Miranion, or others, will return yet still from the east. This flowered a prophesy of sorts, that upon the day that the warriors sail from the east, Liliath will be reborn to lead Ilie. Liliath the Esthela held above all, and to them she was Amalie, like unto the Astar Nomeni. Idhel guided with wisdom, and in her hands was Maril, the staff of Raededir. Maril she used to bring bounty to her people.

54,879-54,750 BBT|153-282 YE—As Anatelea became more founded, the question of rule arose. Before the Atanosta, the Astar had been ruled by Liliath’s dynasty and their council. Two of the original dynasty live still and have exerted their power in the construction of Anatelea. Yet the changing times and solidifying beliefs of the Astar Nomeni have brought some among them to desire a rule by Ilie, The People. The woman Eliadis spoke out on this movement to form a theocratic republic, aided by her champion Ruiwen.
	Eliadis represented the generation of Astar that experienced the strife of the Tel’waith interactions. Born 200 years before the first Tel’waith emissaries arrived in Ithalas, Eliadis worked as an apprentice in Leithan’s school of craft. She worked to forge the tools and weapons of the Astar and was chosen to accompany the Cestani to repair the ships should they be damaged. Eliadis grew in reputation as a strong-minded woman, talented, and far reaching. She made a point to be known, to be helpful to Ilie. She fought alongside Tirwen upon the cliffs of Thoryollo and was felled by the warriors of the Tel’waith. She took the call of Mallorn as he led the pilgrimage of the Astar back to Nuinen and felt that single event spoke more of the capabilities of the Astar than any other. In that moment, she knew that she would dedicate her new life to the exploration of belief. Now slightly over 400 years old, Eliadis had recently taken the man Fededir into her home in Anatelea and had the girl Reisil.
	Eliadis met Ruiwen on the cliffs of Thoryollo, a trained warrior since birth. After the Atanosta, the two rejoined in common mind. During the construction of Anatelea, Eliadis worked to carve and build the first temple of Ilie, its architecture reminiscent of the lost city of Ithalas yet revitalized with a new style birthed with the Atanosta. This temple was magnificent, the center of Anatelea. It became the main for the speeches and meetings that Eliadis and Ruiwen called. Tens, first, then hundreds would gather outside the doors of the beautiful temple. Soon, thousands joined in Eliadis’s teachings, her lessons, and her sermons.


54,750-54,300 BBT|282-732 YE—At first, the Sect of Ilie grew underneath the oligarchy. For a generation it grew and solidified--within it, two positions found shape: the Maron and Arato. Each brought respective social change. The Maron of the sect is the leader of prayer and sacrifice, the model of the Ilie. They are the head of the sect. The first Maron was Eliadis, the second her daughter Reisil. Any followers of Ilie saw the Maron and aspired towards that idea of an Astar. The growth of the Sect brought the Astar closer and encouraged the growth of Ilie. Children were raised by the community, as family groups did not appear. Astar in this time did not take a single other for life, but instead offered themselves to Ilie. Sexuality in the society was encouraged, and the population boomed.
	The Arato is the guardian of the sect. The position ritualizes and symbolizes the protection of Ilie. They exemplify the martial archetype--the Arato brought rise to a new social movement, Calloism, named after the martial traditionalists, the Callo, the Guardians of Ilie. Calloism came with memory and unfounded fears of the wars with the Tel'waith. The People would not suffer in war again. As the population grew and Anatelea expanded, the normal problems of society appeared: crime, though low, was present and punished. The highest offense in the eyes of Calloists was dissention against Ilie. Removing one’s self from society, or not fulfilling one’s contribution to Ilie was heresy. Calloists who contribute to Ilie through the protection of all and the inquisition of dissenters were known as Callo. Callo were vigilantes at first, but before long became an organized part of the Sect.
	The growth of the Sect created a strong dichotomy of social views during this time. The oligarchy was still strong above the Sect, headed by figures from before the Atanosta: Mear, Tirwen, Faranwe, and Apythia. These Four Ancients, the Canyara, were figures that stood above all Astar. Mear and Tirwen were yet children of Liliath, Amilie, and siblings of Mallorn, Vanemer. Thus, at early times in Anatelea with the growth of the Sect, the Canyara were held above the Maron or Arato in Ilie.
	With magic still prevalent in Anatelea during these early times, it dominated society. Their masons were mages, their doctors witches, their warriors wizards and warlocks. Their buildings were smooth and perfect, clay and limestone shaped by magic upon the heights of mountains. Their bridges were beautiful and intricate, impossible feats of engineering rising over the valleys. Wood and rocks and metals were transported on floating platforms, powered by the wills of the Astar. Advanced and powerful magic the Astar wielded, and it became integrated within their society.
	The Canyara moved the Astar towards reconstruction. Anatelea was built up and the population grew, and the Canyara took a step back, and focused on a goal of their own. They, of all the Astar, knew that magic was the cause of the war between the Astar and the Tel'waith. Linarata was consumed by greed and a hunger for the power of the Anariima. The Canyara held still the mighty power of the Ariel but knew that it could not be passed down as it once had been.
	It came to pass that the Canyara met to decide the future of the Astar. The war with Linarata stemmed from the misuse of magic, and the greed for power that it created in those with corrupted minds. The plot of Raededir was a mysterious one, for his fell id grew upon the island of Lónacuma, with only Astaldan and Enalië to witness it. The cause of Raededir’s fall to darkness had yet to be uncovered. Until the time at which the source be discovered and stemmed, the power of Hera’roilya could not lie in the hands of those that walk upon Nuinen, for while it brings progress and glorious unimaginable, the machinations of the eternal world remain balanced with incredible sorrows. These sorrows were fresh in the minds of the Canyara, and thus they decided that magic must be destroyed. The Canyara were the ones to bring magic to the Astar, and now their shoulders must bear the burden of its removal.
So, the Canyara hid their secrets and pondered a road that would destroy magic as it had been known. The removal of magic from a being had never been accomplished by any but Mallorn upon the peak of the Astalena, and none knew his secrets, but the tablets carried from Ithalas by Karn before the Atanosta. These tablets were brought to Celebtal, last the Canyara knew, and there they would go in search of them. During the time, the Canyara ascribed upon tablets of stone the story of the Astar and the fall of the Tel’waith. The tablets held also the knowledge of the Canyara: their lives, teachings, and magical legacy. Finally, the tablets told of the motivation of the Canyara in their plot: to pull magic apart, so that none may hold the absolute power that Linarata and the Naa’waith found. These tablets they called the Kentilia, the People’s Story.
The Kentilia they sent with Faranwe the Dreamer to hide far away, so that none may find them even in search. Faranwe strapped the tablets to his back and left for the western shores, where he stayed long as he built a ship to sail the world.
Three Canyara remained in Anatelea. Two remained as Apythia set on a journey to further aid the purpose of the Canyara: to find the Tablets of Mallorn. She met with and joined Faranwe upon the western shores. Mear and Tirwen alone stayed in Anatelea, where they ruled the city as two. Then begins the Canyara’s end and the odyssey of Apythia and Faranwe the Dreamer. 

54,300- 54,200 BBT|732- 832 YE—Apythia and Faranwe worked upon the coast to build a fine ship. This time brought the two closer, though they were close already from their long experiences in Ithalas. Apythia was worldly, looking always at the people and places around her and seeing the truth in their existence. She knew the secrets of many and the dreams of all she saw—but Faranwe eluded her, for he dreamed for all, and saw more. Long did she aspire to know the heart of Faranwe. Apythia was quiet, with eyes that tore through shrouds. She revealed little to those she knew, but his time drew her close to the other Canyara. Faranwe was young, at least younger than the other Canyara. He looked up at the skies and the stars and realized the scope of Nuinen. He was driven by curiosity, and wished for the sight of the stars, for they were ancient, spilled from the bowl of Nuinen in the birth of the Astar. Faranwe was often lost in dreams, and his eyes rarely saw what he looked at. In this way Apythia and Faranwe greatly differed, but together they were a pair that communicated through their silence.

54,200- 54,100 BBT|832-932 YE—Apythia and Faranwe left Nomenes for the west, for that was where the currents took them. Long they sailed, and by many lands they stopped. The journey was mapped by Faranwe such that they Astar would know the world and their place within it. In time, Faranwe and Apythia came upon Perakor, and this place they recognized. They landed upon the land of the Esthela, and for a time they lingered, though never did they reveal themselves to the Esthela. Faranwe knew that they were a people of strong, good hearts. They had waited long for the return of Miranion, and thus had shown their belief in Ilie. Faranwe saw the future of the Astar in the Esthela, and thus remained with them as Apythia journeyed on to Rhunendor for the tablets of Mallorn.

54,100-53,000 BBT|932-2032 YE—Apythia sailed to Rhunendor, and saw the destruction caused by the Atanosta. The land was distorted, and Celebtal was entrapped beneath the Elerume. Apythia was unhesitant as she wandered the desert, searching for any remnants of the Rhun’waith. She was found by a Tuvem, Fim. Fim was slender and young, agile and curious. She had learned Corilyan Flow from Hundië and Insúlo, and her talent had surpassed the other pupils of the time. She took mastery under Nimben and was returning to Celebtal on pilgrimage from the Amil.
This meeting between Fim and Apythia was a fateful one, and as they camped the night they quickly became friends. Apythia saw much of Faranwe within Fim, and that comforted her. It was from Fim that Apythia first learned of the development of the Rhun, and the changes brought by the gwenyn. An experienced shaper, Hithu, had just left for the Tuluryamar mountains of the north. Apythia was guided by Fim to the crystal city of Celebtal, and there she met with Hundië and Insúlo. She revealed herself as one of the Council of Liliath, and explained the miracle of Vanemer after the Atanosta, and the existence of the Astar Nomeni and Esthela. She told the gwenyn that she had come to claim the tablets of Mallorn. She explained that they were needed to better understand the final works of Vanemer and his ideas on the direction of the Astar after the Atanosta. This Insúlo saw through—Apythia could keep many secrets, but the wisdom of Insúlo was great. Apythia, confronted by Insúlo of her intentions, came forward. She revealed the purposes of the Canyara. The meeting was silent for a time, but Insúlo saw the reasoning in the Canyara’s decision, though he did not agree with it. Insúlo saw other paths that the Astar could tread—and thought it best to explain to Apythia the magic of Corilyan Flow. He took her to the mountaintop Accalaicalë, and in the cold he told her the secrets of Corilyan Shaping and its mesh with Mallorn’s Flow magic: they do not work through the manipulation of energy as origin magic does, but instead they are based on the understanding of the nature of magical existence and the body’s connection to Corilya. It was this understanding—not the power of origin magic—that was the true gift of Hera’roilya.
It was then that Insúlo enlightened Apythia to a path by which magic would not be destroyed, but instead controlled. Magic reaches the physical world through the connections between the soul and the body. The chains that connect these soul and body are channels by which the magic flows, and the places upon which the channels connect act as gates. The most significant of the gates is at the heart and should the other gates’ channels be instead connected to the heart rather than their original gates, then the gates of the physical body would be closed to the soul until the heart is focused to allow flow into those channels. This focus can come from words, actions, or objects that evoke an emotion or idea within the caster.
This form of magic Hundië and Insúlo proposed to Apythia, and it was then that Insúlo opened the doorway at the peak of Accalaicalë. Within the heart of the mountain was held Taregil, whose light spilled forth unto the mountainside when the doors were opened. Insúlo offered to Apythia Taregil to hold and use in her quest, should she choose that path. However, before he left her, he spoke thusly:
“Go from here and seek the stairs above the northern mountains. Speak to the water, the wind, and the trees, for they would guide you. Climb the stairs that once my father ascended and find the wisdom that lies at the precipice. Then, return here. Should you still further your quest, then you may take this light and use it to level us fleeting fools upon the world. Go, north to the mountains.”
Apythia left Accalaicalë and traveled north to the Tuluryamar. Many years Apythia wandered the mountains with Fim, for the stairs to Oria were hidden to her as they were hidden to Astaldan. In that time, Hithu used Taregil to awaken the mountains, rivers, trees, and wind of the land. Thus, the soil was quiet and humble, the trees mighty and strong and brash, the rivers swift and full of laughter and passionate anger, the wind soft and cool and kind. Each of these saw the Rhun and protected them, and they guided the Rhun and lived beside them. The land saw too Apythia and Fim, determined in their search, a misfit duo. Then the wind whispered in their ears, the trees guided their steps, the mountains opened their valleys to the stairsteps of Oria, for each of these things held wisdom unparalleled, and could see the honesty of Apythia’s quest.
Apythia and Fim reached the vale of Oria, though Apythia alone entered. The mists of the glade guided Apythia through the soft grass to rest beside a pool of glassy water. A white-furred fox stepped beside her and drank from the water, and it turned to her.
“Your shoulders are heavy, child. Rest, and speak of the troubles of the Astar.” Apythia spoke to Herya’rosintilya then for three days and three nights. The fox warned Apythia that, should she follow Insúlo’s plan, Mear would betray this cause in its execution to save the life of his son, Toran. If this should come to pass, Apythia must not stop it, for this betrayal would set in motion a sequence of events that would set Toran on his tragic path. Mear must not be slain, for his time would not yet be at an end. However, Apythia’s burden would not end with the Astar Nomeni. The world must bear this burden equally, for the Astar Nomeni alone do not hold Hera’roilya’s gift. Apythia must take the aspect of dark times and travel the world.
With this knowledge, Apythia departed from Oria. She traveled with Fim to Accalaicalë and took Taregil from the pedestal within. She donned her cloak and returned to Celebtal. She met once more with Insúlo, who waited atop the crystal stairs with the tablets of Mallorn.
“You go in wisdom, though not all will feel so. This peace and balance of the world will be shattered, and many will be sorrowed. You will be cursed and hated. You will be the breaker of souls and the destroyer of destinies. Your legacy will be one of evil, Child of the Flowers. Should this burden be too great to bear, or should you become lost upon your quest, return to us, and rest.”
With that, Apythia left Celebtal. Fim with her held the mighty tablets of Mallorn, and the pair returned to Layen, the city of the Esthela, where they met with Faranwe.
During this time, Faranwe was watching the Esthela with great insight. They were a happy people, quiet and hopeful. Mariners and agrarians, they had found harmony with the world. They waited, also, for the return of Miranion, or any from the east. Faranwe divined the importance of the Esthela, knowing they would greatly shape the future of the Astar.
Upon meeting with Faranwe, Apythia and Fim taught him what they knew of Corilyan Flow, for he knew it could be used for his purposes with the Esthela. After learning sufficiently, the words of Mallorn, the three mages carried out a feat to hide Kentilia: they created a body and a soul for the story of the Astar. Kentilia she was named, born a child on the sands of a small village. Her body was written over with the words of the Canyara, though the language had been broken in her transition to life. The words upon her skin held little power, unless combined with the words upon her bones. Together, as they were with Kentilia, they granted her the power of the Canyara. Faranwe placed Kentilia in a deep slumber upon the peaks of the western Fanalan until her time of awakening.
When Apythia, Faranwe, and Fim returned to Anatelea once again, much had developed. The culture of the Callo had birthed and grown, and the Sect of Ilie grew prominent. Tirwen and Mear met with the other of the Canyara and Fim and heard Apythia’s wisdom from Herya’rosintilya. She told them naught of the betrayal prophesized, keeping the secret within her heart. The plot was taken by the Canyara, and Fim teaches Mear and Tirwen the magics of Corilyan Flow. The Canyara meet with Eliadis and Ruiwen and enlighten them to the plot—for the people must know of the means to their fate. Over the next centuries, the Canyara built the bonding technique by which they would link the channels of all Astar Nomeni to the channels of one, such that the shaping of the chosen Astar’s channels may then propagate to the rest of the Astar. This technique was developed and finished, and the time came to change the course of history.
As the ritual began upon the heights of Anatelea, an Asta among them needed to be chosen. This Asta would be the bond of the race, though the breaking of the link would require the breaking of the soul: this Asta would be sacrificed for the preservation of the race. This bond would be chosen at random. It came to pass that when the chosen Asta stepped within the ritual circle, the sun sinking low on the horizon of Nomenes, it was revealed to be Toran, son of Mear. As Mear saw this, he was frozen with fear and pain. Tirwen saw this, but she pushed forward with the binding ritual. Toran alone was bound to the lives of each Astar Nomeni, and he could feel their souls connected to his as he looked over the city, painted by sunset lights. The binding completed, Tirwen stepped forward, shaping the channels by which Toran’s magic flowed into his body, connected each to the channel of his heart. The sun was all but below the horizon when Tirwen placed her hand upon Toran’s forehead to complete the ritual—the breaking of the bond. As Tirwen began, Mear could watch no more. His son’s soul was being torn apart in front of his very eyes, broken like glass under a hammer. Mear unleashed his mighty power, second only to that of Mallorn himself, and thrust many of the bonds held within Toran into the nearest soul—Tirwen’s. Tears streaming down his face, Mear completed the ritual. He ripped Tirwen’s soul to uncountable pieces and broke the bond between her and the Astar Nomeni. The sky lighted with streaks as he tore his own sister’s soul.
Ruiwen, witness to this betrayal, lunged at Mear, her sword long and furious. The two battled atop the spire, and Eliadis fled below, casting her voice to the people, telling them of the betrayal of Mear and Toran, and the death of Tirwen. She knew the results of the ritual and had studied it long before its execution: at the point of interruption, some would still be bonded in soul to Toran, thus they were aberrations to her, monsters. She called,
“Nomeni! Hear my voice from high and listen! Our peoples are divided tonight, for we are betrayed. Our queen is dead, slain by the very brother of Vanemer, whose aberrant son lives on. There are some that may still be one with the betrayer, and should they flee they should be slain!”
During this chaos, Apythia approached Toran, and whispered to the dazed child,
“Run, child, with your people. You have no life here now.”
She drew from beneath her cloak the brilliant light of the Taregil, and pulled from it a single slim cord, which she bestowed upon Toran.
Mear was in a blind rage as he fought with Ruiwen, and he was gravely wounded. His eyes were lost, though he could still sense the movements of Ruiwen. He could feel the flow of magic from her soul, and in that moment, knew her every thought. He smiled, and dodged her attack, moving forward beneath her defenses. He thrust his hand through her body and slew her in the darkness. He tarried not, stumbling down the many stairs to Eliadis, calling to the people below. He came behind her and cast her from the stairs to the valley below. Thus ended Eliadis and Ruiwen, the first Maron and Arato of the Sect of Ilie. Faranwe witnessed this in silence, though now moved and brought Mear low with a deep slumber. He took the son of Liliath and met with Apythia, who stood crying upon the tower peak. She and Fim would watch over Toran, the tragic son, and Faranwe would protect Mear in exile.
In the following times, Toran led his people in haste and sorrow from Anatelea, though many were slain in flight. They traveled down the Beluin to its mouth upon the Niullumea. His people were yet connected in soul, and built for themselves a mighty city, Nemora, upon the banks. This marks the end of the Canyara in Anatelea, as the rule of the Astar passes to the leaders of the Sect of Ilie. Reisil, daughter of Eliadis, takes over as Maron and leader of Anatelea.

53,000-42,000 BBT|2032-13,032 YE—The city Nemora grew and prospered under the rule of Toran. After the ritual, Toran was left shattered. His soul was a half-soul, which left his body weak and crippled. His soul could not support magic any longer on its own, but his only solace lay in the cord of Taregil that Apythia bestowed upon him. The cord was a window to his past, drawing his soul together as it once was. Through it, Toran gave his people the bounty of his magic. Long did the Calloists of Anatelea leave them in peace, though tensions forever grew between the cities. Nemora was ruled by Toran alone, though his connection to his people propagated throughout generations, and he knew the wills of all. This bond became known as the Fealimil, or soul chain. He educated his people in magic and logic, and farming and fishing. Toran’s was a beautiful society. Apythia and Fim became close to Toran, though they were separate from his people, never appearing midst them. They learned from Toran, and taught him many wisdoms, and they became friends.
	During this time, Reisil enflamed the betrayal of Toran and Mear. She led her people to a growth of Calloism, emphasizing the hatred of heretics and those who dissented from the path of Ilie’s wisdom. Her sermons were focused on one foe: Toran. He who had taken to cowardice and fled from Ilie’s judgement. As Reisil passed in time, and her daughter Minis took the title of Maron, the aggression began in earnest. Minis organized the Callo into bands and forces and began fiery crusades into the desert. Soon began great wars between the Anateleans and the Nemorans. In the first great battle, the Anatelean force greatly outnumbered those fighters from Nemora, but Toran strode forward and pulled forth the cord of Taregil and used its power to stop the flow of time, allowing he and his people to destroy the Anateleans. In this battle he revealed to his opponent that he held a cord of an Anariima, the mightiest weapons.
	As the wars continued, it became clear that even with the cord of Taregil, the Nemorans were simply too few to defeat the Anateleans. The end came in a mighty battle: Toran was scarred from his many fights, and his warriors were tired and defeated. The Anateleans came in a host larger than ever before, and the besieged Nemora for a generation. The people of the city hungered and suffered, and Toran could no longer watch his people die. He alone would stand against the force beyond his gates. The son of Mear walked from the gates alone, spear and sword brandished against his dull worn armor. The sand dared not blow as he stepped, and the field fell silent. Toran met the army and fought for many days. Scores fell to his blades, and his people saw this from the walls. Toran fought until he could no longer lift his blade, and in that time, he fell to his knees. The Callo came upon him and surrounded the defeated hero. The clouds broke then, and the sun shone down upon the vast fields. The wind coursed and knocked the warriors from their feet, and it carried from the sky Fim, who had long watched over Toran. She came now, as Toran was true of heart, and most pure of any Astar she had seen. She would not let him die this day. She came down and fought the Anateleans, a Rhun from far away. She had trained in swordsmanship from Hundië himself, who had trained from Astaldan, the greatest swordsman. Fim brought low hundreds of Anateleans and thinned their ranks. It was this moment that brought the Nemorans to their most loyal. Though they were sick and weak, the people of Nemora, the women and children and elderly, saw their leader fight for them, and they gained hope. The gates of Nemora opened and they came to the battle with whatever weapons they had. Hope cannot win a battle, though. The tide was turned, and the Anateleans had taken losses beyond reckoning, but they were trained and fueled by the zeal of Ilie.
	Toran could only kneel as he watched his people die, slain, all of them. Nemora was emptied, and Fim too was slain by the spears of the Callo. As she lay dying, Apythia descended and lay next to her. Apythia cried and grieved for her friend and took it upon herself to carry out Fim’s wish—that Toran be protected and guided, for it was Toran that would see the mistakes of the Astar Nomeni rectified, and it was he that would, in time, give hope to the Esthela. Fim pressed into Apythia’s palm the VanvataThis prophesy Fim left to Apythia, and she lay there on the field midst the dead. Toran broke in that moment, though, and satisfied the tragedy that was predicted by Herya’rosintilya. He pulled once more from his cloak the cord of Taregil, and he reached into Corilya. He witnessed there the souls of his people leaving their bodies, the bonds between Corilya and the physical world breaking. He took hold of the cord of Taregil and tied the departing souls forever to his, for he could not bear to see them leave. Toran knew not the consequences of his actions. His people were souls without bodies, the most painful existence. In their pain, they could do only one thing—reach for the living, and tear into the bodies of those still walking. The Nemorans ripped open the bodies and souls of the Anateleans and rushed into their forms. The living bodies could not handle the trauma, and they would attempt to evacuate the unknown souls through the extrusion of blood and organs through any means, until there was no more blood left. The new bodies satisfied the Nemoran souls not, and they were cursed to forever hunger for a complete physical form.
	The Anateleans were ousted in horror, leaving the battlefield of blood and death. The cursed they name the Saith, the hungry. The Saith are the first true undead, and in their hunger, they spread their affliction by binding others’ bodies to their hungry soul. This has led to some of the original Saith having hundreds or thousands of bodies attached to them, but they are still hungry for more. Apythia crossed to him and laid her hand upon his shoulder, but Toran was gone. Only a shell of pain remained, but Apythia was determined to save Toran from this path of evil. She spoke to him:
	“You must leave this place, child, and run. Run to a place where none can find you or your people! You are good, I know this, but your sorrows have overwhelmed you! Turn from them and come with me.”
	It did no good, though, and Toran turned his back on Apythia. He said no words to her and cried not at the death of Fim. He led his people from Nomenes, construction a great fleet of ships in the Niullumea and sailing north to Amarth and Ithalas, the first home of the Astar.
-During this time, Apythia began her long travels of the world to perform the ritual of Tirwen upon all peoples. This was an impossible task, though, for the world was ever changing and there were many peoples scattered across its surface. Apythia needed a clairvoyance, a knowledge of the many existences of the world. She pondered long with Fim on this, and in her mind, she conceived of a lantern inlaid with a cord of Arangil. This lantern she would name Inehn, but for it she would need a cord of Arangil, lost with Vanemer long before. She pulled then three cords of Taregil and fashioned them into three rings, the Vanvata. If worn, the Vanvata can create a door to any memory or past truth. Apythia used the Vanvata to create a door into Vanemer’s realm and saw again the face of Mallorn. She spoke with him, told him of her quest and her need. Mallorn drew a cord of the Anariima and bestowed it upon her. She forged it into the lantern Inehn, which she took with her on her quest. To Fim she gave the Vanvata. Fim stayed and watched over Toran, and she used the Vanvata to open doors to the knowledge of the Tuvem in Rhunendor. She continued Toran’s teachings and her accruement of knowledge using the rings.

42,000-29,000 BBT|13,032-25,532 YE—The Saith rebuilt Ithalas, though it could never reach its former grandeur. They lived their lives apart and hunted all living things. Many times did Apythia approach Toran to convince him to leave the Saith, to realize his mistake and rectify it. Toran would not listen, and he sat upon his throne of ice in vile tyranny. Toran was consumed by shame and pride and hatred. He took then his father’s name, Mear, for it was a name of power. Mear Toran was often silent upon his throne of ice, pondering only to distract himself from his pain. In this distraction he sought power and guidance. From the vaults below the ruined city he retrieved Tirwen’s Reading Glass. Long he spent peering into the past of the ruined city that his father and his father’s kin governed. He watched the Astalena grow tall. He listened to Enalië sing the Vahalis, he watched the skies part under Linarata’s power. His mind lingered for years on the light of Arangil and Taregil mingling upon the heights of the Astalena. He was enraptured by the power of Linarata and watched her fateful return to Ithalas with a dark hunger. In that past he saw something that none others saw—when the mighty fires rained down on the city and destroyed the tower of Liliath, when Mallorn and Linarata Envintule were destroyed, one thing lingered. A dark heart, shadowy and vile. The heart lingered only a moment before being destroyed in the flames, but that moment Mear Toran saw as an opportunity. He went again to the vaults below the city and drew from his father’s archive the Traveling Amulet. Toran poured his power into it, and he traveled back to that moment over fifteen thousand years prior. He flew for a moment over the burning city. Below him the screams of the dying. Above him the light of fire and star. In front of him, the black heart of Linarata Envintule. He seized the macabre artifact and returned at once to his present. The black heart beat in his hand, and Toran felt its power wash over him—but he also saw its cost. The Andolem Gate began to devour the vault around Toran. The walls began to crumble to dust, the air turned thin and cold, the flesh of Toran’s hand wrinkled and cracked. Toran retreated to his throne once more and let the heart take him. To him, the visions of Linarata Envintule and her dark heart were a means to Toran’s only want—the protection of his people. The Saith were hungry, so he gave them food. They were weak, so he gave them power. They died, but he gave them life. Toran’s life was his no longer. He lived for his people. Should he find a way to satiate their hunger, their need for body and blood, then he would be at peace. He dove into the black heart in search of these answers.
	The heart of Linarata was created under the guidance of Raededir, servant of Sana, and to only Sana is the heart loyal. 


The Saith grew hungrier, though, and the cold lands of Amarth could no longer satisfy them. Farther and farther south they reached and hunted, until after many years they found the outskirts of the Esthela far below and the Vanico to the west. The Esthela were ill-used to the death that the Saith brought. When the first of these skirmishes began, Faranwe awoke the child Kentilia from her slumber, and let her grow in the southern reaches of the Esthela. As she grew, she heard tales of the remnants left behind by the passing of the Saith. Kentilia knew the life of a poor farmer, though before long her family noticed the differences in her—the magic she performed was reminiscent of the times of old. In time, the Saith became bolder. They saw the Esthela as a perfect source of sustenance to satiate their hunger, and Toran bade them to war. The first war was not much of a war as a massacre. The Esthela were savagely slain and the Saith marched south in numbers, armed by the weapons of old Ithalas. As the Esthela marshalled all that they could and rode to meet the Saith force, one strode ahead. It was Kentilia, a child among them. She raised her hand towards the Saith, and her eyes scanned over them. Kentilia surged with the power bestowed upon her by the words of the Canyara, an unconscious use of origin magic. The wind blew in a gale past her, and the earth shook as if a volcano were erupting. The energy of the world was at her command. Toran saw her and was struck. He could feel her resonance and knew it to be different, some sort of memory. He hid away from the battle and ordered the Saith to return north. The Saith were routed, and Kentilia had revealed herself to the Esthela and completed their old prophesy. The Esthela saw her as Liliath reborn, and in time they incorporated her into their belief of Ilie. With the Saith pushed back, a time of peace came over the Fanalan. In her unveiling, Kentilia found no solace in her existence—who was she? What was the meaning of the markings on her skin? Why could she command the magics of old while others could not?
	Though many attempted to decipher her tattoos, the meaning could not be uncovered. Those monks and visionaries that copied the words from Kentilia’s skin compiled the strange writing into what came to be known as the Kentilian Scriptures. One among the monks met with Kentilia in secret—he was hunched and shrouded

Kentilia was matriarch of the Esthela during the time when the Saith were quiet, and she taught them many things, ushering them into a golden age.
Apythia attempted once more to convince Toran to leave Ithalas and the Saith, but this time she first approached Faranwe, and awoke Mear from his long slumber. She brought Mear before Toran in Ithalas, unbeknownst to the Saith, and he spoke to his son. Toran was shocked by the appearance of his father, the one who had saved his life so long ago. Toran knew the pull of age now but was a child again under his father’s eyes. Mear saw the sorrow of his son, and the pain the Saith had wreathed unto the Esthela and Vanico. He convinced Toran into exile, and both retreated to the isles of the east. The Saith were without their bond, and were lost in this time. Apythia now took to the burden set upon her by Herya’rosintilya. She must make the peoples of the world equal in magic. She traveled the world cloaked in black, and performed the dark ritual upon all the cultures of the world, be them Astar or Naa’waith in nature. She reformed the channels of magic so that none may wield fully the power of the soul, but instead all may know the gift of understanding that Hera’roilya gave. 
The night came that Kentilia was approached by a wolf, slender and quick. The wolf spoke to her, warning her that the Saith would not tarry long, only long enough to rebuild their force and attack once more. The wolf was Fim, a memory of her created by Apythia. This wolf was the world’s first illusion, and was sent to Kentilia by Apythia. She had spoken with Faranwe, and knew that Kentilia would be the one to break Toran from his bonds of sorrow. Fim spoke to Kentilia, and told her of Toran, the creator of the Saith and the one who may be able to destroy them.
	Soon, Kentilia set upon this quest: to find this Toran, and convince him to aid the Esthela. She was guided by Fim, but long they traveled to find Toran and Mear. The island they found, guided by flowers of Tolas in the night, and they spoke to Toran and Mear. By Kentilia’s words, Toran felt the shame and sadness of what the Saith had done well into him at last, and he broke from his sorrows. Toran, Mear, and Kentilia joined together and returned to Layen. There they formed the Faranax, those trained to fight against the Saith in any capacity. The Faranax took to the north, and destroyed many Saith. After many generations, the Faranax had grown in power enough to fully battle the Saith in the north, and fearsome wars began again in Amarth. In a gruesome battle, the Faranax had surrounded Ithalas and broken into the gates, and they fought the Saith in the crystal streets. Toran was close to his redemption, cleansing the world of the Saith, and he slew those he long ago created. Ithalas was sacked, and the Faranax were victorious. The Saith were destroyed, save for some that escaped into the night. The battle did not come free, for Toran himself was slain with his father Mear, and Kentilia fell in the fighting as well. Their bodies were recovered, and Toran and Mear were buried upon the island in which they found exile. Kentilia was brought to Layen, where the words upon her bones were discovered. The mages of the Esthela discovered that the words of Kentilia’s flesh and bones were words of power, and combined could summon forth the ancient magics of the Canyara. This magic is known as Esthelan Scripture. After the fall of the Saith, the Esthela continued their growth, and their culture spread across the continents, along with that of the Vanico and Astar Nomeni. These three cultures became the roots of many Astar peoples. Thus ended the Yen Envinyanta and began the Yen Alasta.
%%%%%%%%%%%%%%%%%%%%%%%%%%%%%%%%%%%%%%%%%%%%%%%%%%%%%%%%%%%
%CHAPTER
%%%%%%%%%%%%%%%%%%%%%%%%%%%%%%%%%%%%%%%%%%%%%%%%%%%%%%%%%%%
\chapter{Chapter Three}
%%%%%%%%%%%%%%%%%%%%%%%%%%%%%%%%%%%%%%%%%%%%%%%%%%%%%%%%%%%
%CHAPTER
%%%%%%%%%%%%%%%%%%%%%%%%%%%%%%%%%%%%%%%%%%%%%%%%%%%%%%%%%%%
\chapter{Chapter Four}
%%%%%%%%%%%%%%%%%%%%%%%%%%%%%%%%%%%%%%%%%%%%%%%%%%%%%%%%%%%
%CHAPTER
%%%%%%%%%%%%%%%%%%%%%%%%%%%%%%%%%%%%%%%%%%%%%%%%%%%%%%%%%%%
\chapter{Chapter Five}
%%%%%%%%%%%%%%%%%%%%%%%%%%%%%%%%%%%%%%%%%%%%%%%%%%%%%%%%%%%
%CHAPTER
%%%%%%%%%%%%%%%%%%%%%%%%%%%%%%%%%%%%%%%%%%%%%%%%%%%%%%%%%%%
\chapter{Chapter Six}
% begin back matter

\end{document}
% END THE DOCUMENT