\documentclass[12pt]{article}
\usepackage[margin=1in]{geometry}
\usepackage[all]{xy}
\usepackage{setspace}

\usepackage{enumitem}
\usepackage{amsmath,amsthm,amssymb,color,latexsym}
\usepackage{geometry}        
\geometry{letterpaper}    
\usepackage{graphicx}
\usepackage{float}
\usepackage{siunitx}
\usepackage{soul}
\usepackage{bold-extra}
%\usepackage[ampersand]{easylist}
\usepackage[at]{easylist}
%\usepackage{SIunits}
\newcommand{\expnumber}[2]{{#1}\mathrm{e}{#2}}
\newcommand{\degree}{$^{\circ}$}
\newtheorem{problem}{Question}
\newenvironment{solution}[1][\it{Response}]{\textbf{#1. } }{$\square$}

\newlength\mytemplen
\newsavebox\mytempbox
\setlength{\fboxrule}{2pt}

\newlength{\bibitemsep}\setlength{\bibitemsep}{.2\baselineskip plus .05\baselineskip minus .05\baselineskip}
\newlength{\bibparskip}\setlength{\bibparskip}{0pt}
\let\oldthebibliography\thebibliography
\renewcommand\thebibliography[1]{%
	\oldthebibliography{#1}%
	\setlength{\parskip}{\bibitemsep}%
	\setlength{\itemsep}{\bibparskip}%
}

\usepackage{titlesec} 
\titleformat{\section}[runin]{\large \bfseries}{}{}{}[]
\titleformat{\subsection}[runin]{\normalfont \bfseries}{}{}{}[]
%\titleformat{\section}[runin]{}{}{}{}[

%%%%%%%%%%%%%%%%%%%%%%%%%%%%%%%%%%%%%%%%%%%%%%%%%%%%%%%%%%%%%%%
%BEGIN DOCUMENT
%%%%%%%%%%%%%%%%%%%%%%%%%%%%%%%%%%%%%%%%%%%%%%%%%%%%%%%%%%%%%%%
\begin{document}
	
\Large \textbf{Culture Library}

\normalsize

\noindent\makebox[\linewidth]{\rule{0.75\paperwidth}{1pt}}

Example Progression:
\begin{easylist}
	\ListProperties(Numbers=R,Progressive=2ex,Style*=\bf)
	@ \textsc{\textbf{ Parent culture}}
	
	\normalfont Spoken Language: language (analogue)
	
	Written Language: alphabet
	
	Morphology: (social, economic, religious, magical)
	
	Notes: This culture likes cats.
	@@ \textsc{\textbf{ Child culture}}
		
	\normalfont Spoken Language: Childic (French)
	
	Written Language: Childic heiroglyphic
	
	Morphology: (tribal chiefdom, hunter-gatherer, animist, nature magic)
	
	Notes: This culture likes dogs.
	
	@ \textsc{\textbf{ Parent culture 2...}}
\end{easylist}

\noindent\makebox[\linewidth]{\rule{0.75\paperwidth}{1pt}}

\vspace{1in}

\begin{easylist}
	\NewList
	\ListProperties(Numbers=R,Progressive=3ex,Style*=\bf,Style1*=\Large\bf)
	@ \textsc{\textbf{ Waith of the Four Cities}}
	
	\normalfont Spoken Language: Inivatic (Tolkien Elvish)
	
	Written Language: Inivatic
	
	Morphology: (Magocratic, agrarian, centralized polytheistic, social supplication \& pre-Apythian arcane)
	
	{\footnotesize Notes: The Four Cities of this culture were connected by the Anariima Gates, allowing them to influence the entire northern hemisphere.}
	@@ \textsc{\textbf{Post-Apythian Vanhar}}
	
	\normalfont Spoken Language: Inivatic (Tolkien Elvish)
	
	Written Language: Inivatic
	
	Morphology: (Warrior state, subterrain agrarian, deceased polytheism, no supplication)
	
	{\footnotesize Notes: The Vanhar are the first fully subterranean waith culture.}
	@@ \textsc{\textbf{Post-Apythian Rhun}}
	
	\normalfont Spoken Language: Inivatic (Tolkien Elvish)
	
	Written Language: Inivatic
	
	Morphology: (tribal egalitarian, hunter-gatherer, deceased polytheistic, no supplication)
	
	{\footnotesize Notes: The Rhun underwent a diaspora from Rhunendor after the Flood of Revelation.}
	@ \textsc{\textbf{Prehistoric Loss'kelvar}}
	
	\normalfont Spoken Language: Proto-Keleshi (Proto-Niger-Congo)
	
	Written Language: N/A
	
	Morphology: (egalitarian, hunter-gatherer, animist, nature magic)
	
	%{\footnotesize Notes: The Four Cities of this culture were connected by the Anariima Gates, allowing them to influence the entire northern hemisphere.}
	@@ \textsc{\textbf{Modern Loss'kelvar}}
	
	\normalfont Spoken Language: Keleshi (Igbo)
	
	Written Language: N/A
	
	Morphology: (tribal chiefdom, hunter-gatherer, pseudopolytheism, nature magic)
	
	%{\footnotesize Notes: The Vanhar are the first fully subterranean waith culture.}
	@@ \textsc{\textbf{Prehistoric Orun}}
	
	\normalfont Spoken Language: Proto-Ebo (Proto-Yoruba)
	
	Written Language: N/A
	
	Morphology: (egalitarian, hunter-gatherer, animist, nature magic)
	
	%{\footnotesize Notes: The Vanhar are the first fully subterranean waith culture.}
	@@@ \textsc{\textbf{Isikal}}
	
	\normalfont Spoken Language: Isikali (Zulu)
	
	Written Language: N/A
	
	Morphology: (egalitarian, hunter-gatherer, monotheist \& pseudopolytheist, nature magic)
	
	{\footnotesize Notes: Isikali lizard-men perform violent sacrificial supplication directed towards the moon, their primary deity, which they believe slew all other deities. The goal of their supplication is natural balance and preservation.}
	@@@ \textsc{\textbf{Nla}}
	
	\normalfont Spoken Language: Ebo (Yoruba)
	
	Written Language: Ebo
	
	Morphology: (centralized city-state theocracy, agrarian \& slavery, Apotheism \& centralized polytheism, apotheosis)
	
	%{\footnotesize Notes: The Vanhar are the first fully subterranean waith culture.}
	@@@@ \textsc{\textbf{Nlashikir}}
	
	\normalfont Spoken Language: Ebo (Yoruba)
	
	Written Language: Ebo
	
	Morphology: (egalitarian, hunter-gatherer \& mercantile, polytheism, nature magic)
	
	%{\footnotesize Notes: The Vanhar are the first fully subterranean waith culture.}
	@@@ \textsc{\textbf{Kekere}}
	
	\normalfont Spoken Language: Yaltalese (dia. Niger-Congo)
	
	Written Language: N/A
	
	Morphology: (tribal chiefdom, hunter-gatherer, pseudopolytheism, nature magic)
	
	%{\footnotesize Notes: The Vanhar are the first fully subterranean waith culture.}
	@ \textsc{\textbf{Usadan Yfanald Culture}}
	
	\normalfont Spoken Language: dia. Inivatic (Tolkein Elvish + PIE)
	
	Written Language: N/A
	
	Morphology: (tribal theocracy, hunter-gatherer \& horticultural, animist, health magic)
	
	%{\footnotesize Notes: The Vanhar are the first fully subterranean waith culture.}
	\vspace{0.25in}
	\textbf{\textit{Atanostan}}
	@@ \textsc{\textbf{Tellin Culture}}
	
	\normalfont Spoken Language: Wesaxian (Proto-Indo-Aryan) by 49,700 BBT
	
	Written Language: Okos logographic
	
	Morphology: (centralized theocracy, agrarian, centralized polytheistic [war, afterlife], prophetic magic)
	
	%{\footnotesize Notes: The Vanhar are the first fully subterranean waith culture.}
	@@@ \textsc{\textbf{Eyasolian}}
	
	\normalfont Spoken Language: dia. Wesaxian (mod. PIA)
	
	Written Language: N/A
	
	Morphology: (centralized theocracy, horticulturalist, decentralized animist-polytheistic, spirit-binding magic)
	
	{\footnotesize Notes: The Eyasolian magic uses some spirits to bind others to objects and people, allowing the spirit to share or control the new form.}
	@@ \textsc{\textbf{Proto Huroman}}

	\normalfont Spoken Language: Proto-Huroman (PIE+PIA+Bangala)
	
	Written Language: N/A
	
	Morphology: (gerontocracy, fishing horticultural, animist \& ancestor worship, health magic)
	
	{\footnotesize Notes: Proto-Huroman magic takes a `witch-doctor' like focus, with tribal practitioners supplicating spirits for healing, cures, and curses. They also supplicate their ancestors for guidance and guardianship.}
	@@@ \textsc{\textbf{Nomadic Huroman}}
	
	\normalfont Spoken Language: Huroman (PIA+Bangala) by 48,800 BBT
	
	Written Language: N/A
	
	Morphology: (egalitarian, mercantile \& hunter-gatherer, animist, centralized health magic)
	
	{\footnotesize Notes: The more mature nomadic Huromans, living primarily in the north, central, and dry southeast, practice a more consolidated form of the health magic of their ancestors. Rather than a single witch-doctor from any tribe being able to supplicate to the spirits, these nomads now require a gestalt of tribal spiritualists to work together. The mercantile and social nature of these nomads brings them generally closer than others, making these gestalts not incredibly rare.}
	@@@ \textsc{\textbf{Coastal Huroman}}
	
	\normalfont Spoken Language: Huroman (PIA+Bangala) by 48,800 BBT
	
	Written Language: Huroman logographic
	
	Morphology: (monarchical, horticultural \& slavery, distributed polytheistic [social, nature], decentralized health magic)
	
	{\footnotesize Notes: The coastal Huromans are a settled culture with a separation between the religious authorities and state authorities, making them one of the most progressive cultures around in this respect. The decentralized and heirarchical nature of Huroman Dzers allows more levels of society to access supplication rituals.}
	@@ \textsc{\textbf{Proto Eastern}}
	
	\normalfont Spoken Language: Proto-Eastern (PIE + mod. Indo-Iranian)
	
	Written Language: N/A
	
	Morphology: (tribal egalitarian, hunter-gatherer, animist \& ancestor worship, decentralized ancestor magic)
	
	{\footnotesize Notes: The Proto-Easterners were a widely nomadic people practicing local and cosmic animism with the addition of tribal ancestral guardians. The primary goal of their spiritual supplication was on these ancestral guardians, which were manifested in fetish objects carried by the tribes.}
	@@@ \textsc{\textbf{Ghahakan}}
	
	\normalfont Spoken Language: Ghahakan (mod. Indo-Iranian) by 47,500 BBT
	
	Written Language: N/A
	
	Morphology: (chiefdom, hunter-gatherer \& pastoral, centralized polytheism \& ancestor worship, centralized ancestor magic)
	
	{\footnotesize Notes: The Ghahakans call upon spirits and deities to gather the spirits of the dead in regional monoliths.}
	@@@ \textsc{\textbf{North Samisian}}
	
	\normalfont Spoken Language: Samisian (mod. Indo-Iranian + Ossetic)
	
	Written Language: N/A
	
	Morphology: (chiefdom, fishing horticultural, pseudopolytheism \& ancestor worship, nature magic)
	
	{\footnotesize Notes: These early inhabitants of the Suhairi region called upon local nature gods for the main purpose of natural preservation.}
	@@@ \textsc{\textbf{East Samisian}}
	
	\normalfont Spoken Language: Samisian (mod. Indo-Iranian + Kurdish)
	
	Written Language: N/A
	
	Morphology: (chiefdom, hunter-gatherer \& pastoral, decentralized pseudopolytheist, decentralized ancestor magic)
	
	{\footnotesize Notes: Territorial totems imbued with ancestral spirits similar to but more decentralized than Ghahakan monoliths}
	@@@@ \textsc{\textbf{Gairiatic}}
	
	\normalfont Spoken Language: \_\_\_\_\_ (mod. Indo-Iranian + Kurdish)
	
	Written Language: N/A
	
	Morphology: (egalitarian, hunter-gatherer, decentralized pseudopolytheist, spirit-totem magic)
	
	{\footnotesize Notes: The Gairians lived in generally smaller, more egalitarian groups than their ancestral Samisians. They were exclusively hunter-gatherers and fishers, and believed in unique local anthropomorphized spirits. The Gairians continued to make totems like their ancestors, though these totems acted as houses and conduits for spirits rather than ancestral guardians. Tribes would create totems for important spirits, and any tribe could supplicate to that totem to gain the spirits boons or use it to curse others. Generally, supplication was used for social and natural ends.}
	@@@@ \textsc{\textbf{Darungian}}
	
	\normalfont Spoken Language: ????????????? (mod. Indo-Iranian + Kurdish)
	
	Written Language: N/A
	
	Morphology: (chiefdom, hunter-gatherer \& pastoral, decentralized pseudopolytheist, decentralized ancestor magic)
	
	{\footnotesize Notes: Territorial totems imbued with ancestral spirits similar to but more decentralized than Ghahakan monoliths}
	@@@ \textsc{\textbf{Vinaltakhan}}
	
	\normalfont Spoken Language: Proto-Eastern (PIE + mod. Indo-Iranian) until 38,800 BBT
	
	Written Language: N/A
	
	Morphology: (egalitarian, hunter-gatherer, pantheist, \& ancestor worship, spirit-line social magic)
	
	{\footnotesize Notes: The Vinaltakh cultures practiced a unique form of animism and magic. They believed that, rather than the ancestral animistic belief of all things having spirits, all things exist as part of a single spirit. This single spirit, one with the material world itself, is marked by spiritual trails and sites. Supplication can allow tribes to communicate or travel along these spiritual lines, and general supplication goes towards sustaining this network.}
\end{easylist}

\end{document}